\hypertarget{concurrent-programming-with-async}{%
\section{Concurrent Programming with
Async}\label{concurrent-programming-with-async}}

The logic of building programs that interact with the outside world is
often dominated by waiting; waiting for the click of a mouse, or for
data to be fetched from disk, or for space to be available on an
outgoing network buffer. Even mildly sophisticated interactive
applications are typically \emph{concurrent}, needing to wait for
multiple different events at the same time, responding immediately to
whatever happens first. \index{interactive input/concurrent programming
for}\index{concurrent programming}\index{programming/concurrent
programming with Async}

One approach to concurrency is to use preemptive system threads, which
is the dominant approach in languages like Java or C\#. In this model,
each task that may require simultaneous waiting is given an operating
system thread of its own so it can block without stopping the entire
program. \index{threads/preemptive
vs. single-threaded programs}

Another approach is to have a single-threaded program, where that single
thread runs an \emph{event loop} whose job is to react to external
events like timeouts or mouse clicks by invoking a callback function
that has been registered for that purpose. This approach shows up in
languages like JavaScript that have single-threaded runtimes, as well as
in many GUI toolkits. \index{event loops}\index{system threads}

Each of these mechanisms has its own trade-offs. System threads require
significant memory and other resources per thread. Also, the operating
system can arbitrarily interleave the execution of system threads,
requiring the programmer to carefully protect shared resources with
locks and condition variables, which is exceedingly error-prone.

Single-threaded event-driven systems, on the other hand, execute a
single task at a time and do not require the same kind of complex
synchronization that preemptive threads do. However, the inverted
control structure of an event-driven program often means that your own
control flow has to be threaded awkwardly through the system's event
loop, leading to a maze of event callbacks.

This chapter covers the Async library, which offers a hybrid model that
aims to provide the best of both worlds, avoiding the performance
compromises and synchronization woes of preemptive threads without the
confusing inversion of control that usually comes with event-driven
systems. \index{Async library/benefits
of}

\hypertarget{async-basics}{%
\subsection{Async Basics}\label{async-basics}}

Recall how I/O is typically done in Core. Here's a simple example.

\begin{lstlisting}[language=Caml]
# open Core
# In_channel.read_all
- : string -> string = <fun>
# Out_channel.write_all "test.txt" ~data:"This is only a test."
- : unit = ()
# In_channel.read_all "test.txt"
- : string = "This is only a test."
\end{lstlisting}

From the type of \passthrough{\lstinline!In\_channel.read\_all!}, you
can see that it must be a blocking operation. In particular, the fact
that it returns a concrete string means it can't return until the read
has completed. The blocking nature of the call means that no progress
can be made on anything else until the call is complete.
\index{blocking}

In Async, well-behaved functions never block. Instead, they return a
value of type \passthrough{\lstinline!Deferred.t!} that acts as a
placeholder that will eventually be filled in with the result. As an
example, consider the signature of the Async equivalent of
\passthrough{\lstinline!In\_channel.read\_all!}. \index{Deferred.t}

\begin{lstlisting}[language=Caml]
# #require "async"
# open Async
# Reader.file_contents
- : string -> string Deferred.t = <fun>
\end{lstlisting}

We first load the Async package in the toplevel using
\passthrough{\lstinline!\#require!}, and then open the module. Async,
like Core, is designed to be an extension to your basic programming
environment, and is intended to be opened.

A deferred is essentially a handle to a value that may be computed in
the future. As such, if we call
\passthrough{\lstinline!Reader.file\_contents!}, the resulting deferred
will initially be empty, as you can see by calling
\passthrough{\lstinline!Deferred.peek!}. \index{Deferred.peek}

\begin{lstlisting}[language=Caml]
# let contents = Reader.file_contents "test.txt"
val contents : string Deferred.t = <abstr>
# Deferred.peek contents
- : string option = None
\end{lstlisting}

The value in \passthrough{\lstinline!contents!} isn't yet determined
partly because nothing running could do the necessary I/O. When using
Async, processing of I/O and other events is handled by the Async
scheduler. When writing a standalone program, you need to start the
scheduler explicitly, but \passthrough{\lstinline!utop!} knows about
Async and can start the scheduler automatically. More than that,
\passthrough{\lstinline!utop!} knows about deferred values, and when you
type in an expression of type \passthrough{\lstinline!Deferred.t!}, it
will make sure the scheduler is running and block until the deferred is
determined. Thus, we can write:

\begin{lstlisting}[language=Caml]
# contents
- : string = "This is only a test."
\end{lstlisting}

Slightly confusingly, the type shown here is not the type of
\passthrough{\lstinline!contents!}, which is
\passthrough{\lstinline!string Deferred.t!}, but rather
\passthrough{\lstinline!string!}, the type of the value contained within
that deferred.

If we peek again, we'll see that the value of
\passthrough{\lstinline!contents!} has been filled in.

\begin{lstlisting}[language=Caml]
# Deferred.peek contents
- : string option = Some "This is only a test."
\end{lstlisting}

In order to do real work with deferreds, we need a way of waiting for a
deferred computation to finish, which we do using
\passthrough{\lstinline!Deferred.bind!}. Here's the type-signature of
\passthrough{\lstinline!bind!}.

\begin{lstlisting}[language=Caml]
# Deferred.bind
- : 'a Deferred.t -> f:('a -> 'b Deferred.t) -> 'b Deferred.t = <fun>
\end{lstlisting}

\passthrough{\lstinline!bind!} is effectively a way of sequencing
concurrent computations. In particular,
\passthrough{\lstinline!Deferred.bind d \~f!} causes
\passthrough{\lstinline!f!} to be called after the value of
\passthrough{\lstinline!d!} has been determined. \index{Deferred.bind}

Here's a simple use of \passthrough{\lstinline!bind!} for a function
that replaces a file with an uppercase version of its contents.

\begin{lstlisting}[language=Caml]
# let uppercase_file filename =
    Deferred.bind (Reader.file_contents filename)
      (fun text ->
         Writer.save filename ~contents:(String.uppercase text))
val uppercase_file : string -> unit Deferred.t = <fun>
# uppercase_file "test.txt"
- : unit = ()
# Reader.file_contents "test.txt"
- : string = "THIS IS ONLY A TEST."
\end{lstlisting}

Again, \passthrough{\lstinline!bind!} is acting as a sequencing
operator, causing the file to be saved via the call to
\passthrough{\lstinline!Writer.save!} only after the contents of the
file were first read via
\passthrough{\lstinline!Reader.file\_contents!}.

Writing out \passthrough{\lstinline!Deferred.bind!} explicitly can be
rather verbose, and so Async includes an infix operator for it:
\passthrough{\lstinline!>>=!}. Using this operator, we can rewrite
\passthrough{\lstinline!uppercase\_file!} as follows:

\begin{lstlisting}[language=Caml]
# let uppercase_file filename =
    Reader.file_contents filename
    >>= fun text ->
    Writer.save filename ~contents:(String.uppercase text)
val uppercase_file : string -> unit Deferred.t = <fun>
\end{lstlisting}

In the preceding code, we've dropped the parentheses around the function
on the righthand side of the bind, and we didn't add a level of
indentation for the contents of that function. This is standard practice
for using the infix \passthrough{\lstinline!bind!} operator.
\index{bind function}

Now let's look at another potential use of
\passthrough{\lstinline!bind!}. In this case, we'll write a function
that counts the number of lines in a file:

\begin{lstlisting}[language=Caml]
# let count_lines filename =
    Reader.file_contents filename
    >>= fun text ->
    List.length (String.split text ~on:'\n')
Line 4, characters 5-45:
Error: This expression has type int but an expression was expected of type
         'a Deferred.t
\end{lstlisting}

This looks reasonable enough, but as you can see, the compiler is
unhappy. The issue here is that \passthrough{\lstinline!bind!} expects a
function that returns a \passthrough{\lstinline!Deferred.t!}, but we've
provided it a function that returns the result directly. What we need is
\passthrough{\lstinline!return!}, a function provided by Async that
takes an ordinary value and wraps it up in a deferred.
\index{return function}

\begin{lstlisting}[language=Caml]
# return
- : 'a -> 'a Deferred.t = <fun>
# let three = return 3
val three : int Deferred.t = <abstr>
# three
- : int = 3
\end{lstlisting}

Using \passthrough{\lstinline!return!}, we can make
\passthrough{\lstinline!count\_lines!} compile:

\begin{lstlisting}[language=Caml]
# let count_lines filename =
    Reader.file_contents filename
    >>= fun text ->
    return (List.length (String.split text ~on:'\n'))
val count_lines : string -> int Deferred.t = <fun>
\end{lstlisting}

Together, \passthrough{\lstinline!bind!} and
\passthrough{\lstinline!return!} form a design pattern in functional
programming known as a \emph{monad}. You'll run across this signature in
many applications beyond just threads. Indeed, we already ran across
monads in
\href{error-handling.html\#bind-and-other-error-handling-idioms}{Bind
And Other Error Handling Idioms}. \index{monads}

Calling \passthrough{\lstinline!bind!} and
\passthrough{\lstinline!return!} together is a fairly common pattern,
and as such there is a standard shortcut for it called
\passthrough{\lstinline!Deferred.map!}, which has the following
signature:

\begin{lstlisting}[language=Caml]
# Deferred.map
- : 'a Deferred.t -> f:('a -> 'b) -> 'b Deferred.t = <fun>
\end{lstlisting}

and comes with its own infix equivalent, \passthrough{\lstinline!>>|!}.
Using it, we can rewrite \passthrough{\lstinline!count\_lines!} again a
bit more succinctly:

\begin{lstlisting}[language=Caml]
# let count_lines filename =
    Reader.file_contents filename
    >>| fun text ->
    List.length (String.split text ~on:'\n')
val count_lines : string -> int Deferred.t = <fun>
# count_lines "/etc/hosts"
- : int = 10
\end{lstlisting}

Note that \passthrough{\lstinline!count\_lines!} returns a deferred, but
\passthrough{\lstinline!utop!} waits for that deferred to become
determined, and shows us the contents of the deferred instead.

\hypertarget{using-let_syntax-with-async}{%
\subsubsection{\texorpdfstring{Using \texttt{Let\_syntax} with
Async}{Using Let\_syntax with Async}}\label{using-let_syntax-with-async}}

As was discussed in
\href{error-handling.html\#bind-and-other-error-handling-idioms}{Error
Handling}, there is a special syntax designed for working with monads,
called \passthrough{\lstinline!Let\_syntax!}. Here's what the
\passthrough{\lstinline!bind!}-using version of
\passthrough{\lstinline!count\_lines!} looks like with that syntax.

\begin{lstlisting}[language=Caml]
# let count_lines filename =
    let%bind text = Reader.file_contents filename in
    return (List.length (String.split text ~on:'\n'))
val count_lines : string -> int Deferred.t = <fun>
\end{lstlisting}

And here's the \passthrough{\lstinline!map!}-based version of
\passthrough{\lstinline!count\_lines!}.

\begin{lstlisting}[language=Caml]
# let count_lines filename =
    let%map text = Reader.file_contents filename in
    List.length (String.split text ~on:'\n')
val count_lines : string -> int Deferred.t = <fun>
\end{lstlisting}

The difference here is just syntactic, with these examples compiling
down to the same thing as the corresponding examples written using infix
operators. What's nice about \passthrough{\lstinline!Let\_syntax!} is
that it highlights the analogy between monadic bind and OCaml's built-in
let-binding, thereby making your code more uniform and more readable.

\passthrough{\lstinline!Let\_syntax!} works for any monad, and you
decide which monad is in use by opening the appropriate
\passthrough{\lstinline!Let\_syntax!} module. Opening
\passthrough{\lstinline!Async!} also implicitly opens
\passthrough{\lstinline!Deferred.Let\_syntax!}, but in some contexts you
may want to do that explicitly.

To keep things simple, we'll use the infix notation for map and bind for
the remainder of the chapter. But once you get comfortable with Async
and monadic programming, we recommend using
\passthrough{\lstinline!Let\_syntax!}.

\hypertarget{ivars-and-upon}{%
\subsubsection{Ivars and Upon}\label{ivars-and-upon}}

Deferreds are usually built using combinations of
\passthrough{\lstinline!bind!}, \passthrough{\lstinline!map!} and
\passthrough{\lstinline!return!}, but sometimes you want to construct a
deferred where you can programmatically decide when it gets filled in.
This can be done using an \emph{ivar}. (The term ivar dates back to a
language called Concurrent ML that was developed by John Reppy in the
early '90s. The ``i'' in ivar stands for incremental.)
\index{ivars}\index{Async library/ivars}

There are three fundamental operations for working with an ivar: you can
create one, using \passthrough{\lstinline!Ivar.create!}; you can read
off the deferred that corresponds to the ivar in question, using
\passthrough{\lstinline!Ivar.read!}; and you can fill an ivar, thus
causing the corresponding deferred to become determined, using
\passthrough{\lstinline!Ivar.fill!}. These operations are illustrated
below:

\begin{lstlisting}[language=Caml]
# let ivar = Ivar.create ()
val ivar : '_weak1 Ivar.t =
  {Async_kernel__.Types.Ivar.cell = Async_kernel__Types.Cell.Empty}
# let def = Ivar.read ivar
val def : '_weak2 Deferred.t = <abstr>
# Deferred.peek def
- : '_weak3 option = None
# Ivar.fill ivar "Hello"
- : unit = ()
# Deferred.peek def
- : string option = Some "Hello"
\end{lstlisting}

Ivars are something of a low-level feature; operators like
\passthrough{\lstinline!map!}, \passthrough{\lstinline!bind!} and
\passthrough{\lstinline!return!} are typically easier to use and think
about. But ivars can be useful when you want to build a synchronization
pattern that isn't already well supported.

As an example, imagine we wanted a way of scheduling a sequence of
actions that would run after a fixed delay. In addition, we'd like to
guarantee that these delayed actions are executed in the same order they
were scheduled in. Here's a signature that captures this idea:

\begin{lstlisting}[language=Caml]
# module type Delayer_intf = sig
    type t
    val create : Time.Span.t -> t
    val schedule : t -> (unit -> 'a Deferred.t) -> 'a Deferred.t
  end
module type Delayer_intf =
  sig
    type t
    val create : Time.Span.t -> t
    val schedule : t -> (unit -> 'a Deferred.t) -> 'a Deferred.t
  end
\end{lstlisting}

An action is handed to \passthrough{\lstinline!schedule!} in the form of
a deferred-returning thunk (a thunk is a function whose argument is of
type \passthrough{\lstinline!unit!}). A deferred is handed back to the
caller of \passthrough{\lstinline!schedule!} that will eventually be
filled with the contents of the deferred value returned by the thunk. To
implement this, we'll use an operator called
\passthrough{\lstinline!upon!}, which has the following signature:
\index{thunks}

\begin{lstlisting}[language=Caml]
# upon
- : 'a Deferred.t -> ('a -> unit) -> unit = <fun>
\end{lstlisting}

Like \passthrough{\lstinline!bind!} and
\passthrough{\lstinline!return!}, \passthrough{\lstinline!upon!}
schedules a callback to be executed when the deferred it is passed is
determined; but unlike those calls, it doesn't create a new deferred for
this callback to fill.

Our delayer implementation is organized around a queue of thunks, where
every call to \passthrough{\lstinline!schedule!} adds a thunk to the
queue and also schedules a job in the future to grab a thunk off the
queue and run it. The waiting will be done using the function
\passthrough{\lstinline!after!}, which takes a time span and returns a
deferred which becomes determined after that time span elapses:

\begin{lstlisting}[language=Caml]
# module Delayer : Delayer_intf = struct
    type t = { delay: Time.Span.t;
               jobs: (unit -> unit) Queue.t;
             }

    let create delay =
      { delay; jobs = Queue.create () }

    let schedule t thunk =
      let ivar = Ivar.create () in
      Queue.enqueue t.jobs (fun () ->
        upon (thunk ()) (fun x -> Ivar.fill ivar x));
      upon (after t.delay) (fun () ->
        let job = Queue.dequeue_exn t.jobs in
        job ());
      Ivar.read ivar
  end
module Delayer : Delayer_intf
\end{lstlisting}

This code isn't particularly long, but it is subtle. In particular, note
how the queue of thunks is used to ensure that the enqueued actions are
run in the order they were scheduled, even if the thunks scheduled by
\passthrough{\lstinline!upon!} are run out of order. This kind of
subtlety is typical of code that involves ivars and
\passthrough{\lstinline!upon!}, and because of this, you should stick to
the simpler map/bind/return style of working with deferreds when you
can.~

\hypertarget{understanding-bind-in-terms-of-ivars-and-upon}{%
\paragraph{\texorpdfstring{Understanding \texttt{bind} in terms of ivars
and
\texttt{upon}}{Understanding bind in terms of ivars and upon}}\label{understanding-bind-in-terms-of-ivars-and-upon}}

Here's roughly what happens when you write
\passthrough{\lstinline!let d' = Deferred.bind d \~f!}.

\begin{itemize}
\item
  A new ivar \passthrough{\lstinline!i!} is created to hold the final
  result of the computation. The corresponding deferred is returned
\item
  A function is registered to be called when the deferred
  \passthrough{\lstinline!d!} becomes determined.
\item
  That function, once run, calls \passthrough{\lstinline!f!} with the
  value that was determined for \passthrough{\lstinline!d!}.
\item
  Another function is registered to be called when the deferred returned
  by \passthrough{\lstinline!f!} becomes determined.
\item
  When that function is called, it uses it to fill
  \passthrough{\lstinline!i!}, causing the corresponding deferred it to
  become determined.
\end{itemize}

That sounds like a lot, but we can implement this relatively concisely.

\begin{lstlisting}[language=Caml]
# let my_bind d ~f =
    let i = Ivar.create () in
    upon d (fun x -> upon (f x) (fun y -> Ivar.fill i y));
    Ivar.read i
val my_bind : 'a Deferred.t -> f:('a -> 'b Deferred.t) -> 'b Deferred.t =
  <fun>
\end{lstlisting}

Async's real implementation has more optimizations and is therefore more
complicated. But the above implementation is still a useful first-order
mental model for how bind works under the covers. And it's another good
example of how \passthrough{\lstinline!upon!} and ivars can useful for
building concurrency primitives.

\hypertarget{examples-an-echo-server}{%
\subsection{Example: An Echo Server}\label{examples-an-echo-server}}

Now that we have the basics of Async under our belt, let's look at a
small standalone Async program. In particular, we'll write an echo
server, \emph{i.e.}, a program that accepts connections from clients and
spits back whatever is sent to it.\protect\hypertarget{echo}{}{echo
servers}\protect\hypertarget{ALecho}{}{Async library/echo server
example}

The first step is to create a function that can copy data from an input
to an output. Here, we'll use Async's \passthrough{\lstinline!Reader!}
and \passthrough{\lstinline!Writer!} modules, which provide a convenient
abstraction for working with input and output channels: \index{Writer
module}\index{Reader module}\index{I/O (input/output) operations/copying
data}

\begin{lstlisting}[language=Caml]
open Core
open Async

(* Copy data from the reader to the writer, using the provided buffer
   as scratch space *)
let rec copy_blocks buffer r w =
  Reader.read r buffer
  >>= function
  | `Eof -> return ()
  | `Ok bytes_read ->
    Writer.write w (Bytes.to_string buffer) ~len:bytes_read;
    Writer.flushed w
    >>= fun () ->
    copy_blocks buffer r w
\end{lstlisting}

Bind is used in the code to sequence the operations: first, we call
\passthrough{\lstinline!Reader.read!} to get a block of input. Then,
when that's complete and if a new block was returned, we write that
block to the writer. Finally, we wait until the writer's buffers are
flushed, waiting on the deferred returned by
\passthrough{\lstinline!Writer.flushed!}, at which point we recurse. If
we hit an end-of-file condition, the loop is ended. The deferred
returned by a call to \passthrough{\lstinline!copy\_blocks!} becomes
determined only once the end-of-file condition is hit.
\index{end-of-file condition}

One important aspect of how \passthrough{\lstinline!copy\_blocks!} is
written is that it provides \emph{pushback}, which is to say that if the
process can't make progress writing, it will stop reading. If you don't
implement pushback in your servers, then anything that prevents you from
writing (e.g., a client that is unable to keep up) will cause your
program to allocate unbounded amounts of memory, as it keeps track of
all the data it intends to write but hasn't been able to yet.

\hypertarget{tail-calls-and-chains-of-deferreds}{%
\subsubsection{Tail-calls and chains of
deferreds}\label{tail-calls-and-chains-of-deferreds}}

There's another memory problem you might be concerned about, which is
the allocation of deferreds. If you think about the execution of
\passthrough{\lstinline!copy\_blocks!}, you'll see it's creating a chain
of deferreds, two per time through the loop. The length of this chain is
unbounded, and so, naively, you'd think this would take up an unbounded
amount of memory as the echo process continues.

Happily, it turns out that this is a special case that Async knows how
to optimize. In particular, the whole chain of deferreds should become
determined precisely when the final deferred in the chain is determined,
in this case, when the \passthrough{\lstinline!Eof!} condition is hit.
Because of this, we could safely replace all of these deferreds with a
single deferred. Async does just this, and so there's no memory leak
after all.

This is essentially a form of tail-call optimization, lifted to the
Async monad. Indeed, you can tell that the bind in question doesn't lead
to a memory leak in more or less the same way you can tell that the tail
recursion optimization should apply, which is that the bind that creates
the deferred is in tail-position. In other words, nothing is done to
that deferred once it's created; it's simply returned as is.
\index{tail calls}

\passthrough{\lstinline!copy\_blocks!} provides the logic for handling a
client connection, but we still need to set up a server to receive such
connections and dispatch to \passthrough{\lstinline!copy\_blocks!}. For
this, we'll use Async's \passthrough{\lstinline!Tcp!} module, which has
a collection of utilities for creating TCP clients and servers:
\index{TCP
clients/servers}

\begin{lstlisting}[language=Caml]
(** Starts a TCP server, which listens on the specified port, invoking
    copy_blocks every time a client connects. *)
let run () =
  let host_and_port =
    Tcp.Server.create
      ~on_handler_error:`Raise
      (Tcp.Where_to_listen.of_port 8765)
      (fun _addr r w ->
         let buffer = Bytes.create (16 * 1024) in
         copy_blocks buffer r w)
  in
  ignore (host_and_port : (Socket.Address.Inet.t, int) Tcp.Server.t Deferred.t)
\end{lstlisting}

The result of calling \passthrough{\lstinline!Tcp.Server.create!} is a
\passthrough{\lstinline!Tcp.Server.t!}, which is a handle to the server
that lets you shut the server down. We don't use that functionality
here, so we explicitly ignore \passthrough{\lstinline!server!} to
suppress the unused-variables error. We put in a type annotation around
the ignored value to make the nature of the value we're ignoring
explicit.

The most important argument to
\passthrough{\lstinline!Tcp.Server.create!} is the final one, which is
the client connection handler. Notably, the preceding code does nothing
explicit to close down the client connections when the communication is
done. That's because the server will automatically shut down the
connection once the deferred returned by the handler becomes determined.

Finally, we need to initiate the server and start the Async scheduler:

\begin{lstlisting}[language=Caml]
(* Call [run], and then start the scheduler *)
let () =
  run ();
  never_returns (Scheduler.go ())
\end{lstlisting}

One of the most common newbie errors with Async is to forget to run the
scheduler. It can be a bewildering mistake, because without the
scheduler, your program won't do anything at all; even calls to
\passthrough{\lstinline!printf!} won't reach the terminal.

It's worth noting that even though we didn't spend much explicit effort
on thinking about multiple clients, this server is able to handle many
clients concurrently connecting and reading and writing data.

Now that we have the echo server, we can connect to the echo server
using the netcat tool, which is invoked as \passthrough{\lstinline!nc!}.
Note that we use \passthrough{\lstinline!dune exec!} to both build and
run the executable. We use the double-dashes so that Dune's parsing of
arguments doesn't interfere with argument parsing for the executed
program.

\begin{lstlisting}[language=bash]
$ dune exec -- ./echo.exe &
$ echo "This is an echo server" | nc 127.0.0.1 8765
This is an echo server
$ echo "It repeats whatever I write" | nc 127.0.0.1 8765
It repeats whatever I write
$ killall echo.exe
\end{lstlisting}

\hypertarget{functions-that-never-return}{%
\subparagraph{Functions that Never
Return}\label{functions-that-never-return}}

You might wonder what's going on with the call to
\passthrough{\lstinline!never\_returns!}.
\passthrough{\lstinline!never\_returns!} is an idiom that comes from
Core that is used to mark functions that don't return. Typically, a
function that doesn't return is inferred as having return type
\passthrough{\lstinline!'a!}:
\index{Scheduler.go}\index{loop\_forever}\index{never\_returns}\index{functions/non-returning}

\begin{lstlisting}[language=Caml]
# let rec loop_forever () = loop_forever ()
val loop_forever : unit -> 'a = <fun>
# let always_fail () = assert false
val always_fail : unit -> 'a = <fun>
\end{lstlisting}

This can be surprising when you call a function like this expecting it
to return \passthrough{\lstinline!unit!}. The type-checker won't
necessarily complain in such a case:

\begin{lstlisting}[language=Caml]
# let do_stuff n =
    let x = 3 in
    if n > 0 then loop_forever ();
    x + n
val do_stuff : int -> int = <fun>
\end{lstlisting}

With a name like \passthrough{\lstinline!loop\_forever!}, the meaning is
clear enough. But with something like
\passthrough{\lstinline!Scheduler.go!}, the fact that it never returns
is less clear, and so we use the type system to make it more explicit by
giving it a return type of \passthrough{\lstinline!never\_returns!}.
Let's do the same trick with \passthrough{\lstinline!loop\_forever!}:

\begin{lstlisting}[language=Caml]
# let rec loop_forever () : never_returns = loop_forever ()
val loop_forever : unit -> never_returns = <fun>
\end{lstlisting}

The type \passthrough{\lstinline!never\_returns!} is uninhabited, so a
function can't return a value of type
\passthrough{\lstinline!never\_returns!}, which means only a function
that never returns can have \passthrough{\lstinline!never\_returns!} as
its return type! Now, if we rewrite our
\passthrough{\lstinline!do\_stuff!} function, we'll get a helpful type
error:

\begin{lstlisting}[language=Caml]
# let do_stuff n =
    let x = 3 in
    if n > 0 then loop_forever ();
    x + n
Line 3, characters 19-34:
Error: This expression has type never_returns
       but an expression was expected of type unit
       because it is in the result of a conditional with no else branch
\end{lstlisting}

We can resolve the error by calling the function
\passthrough{\lstinline!never\_returns!}:

\begin{lstlisting}[language=Caml]
# never_returns
- : never_returns -> 'a = <fun>
# let do_stuff n =
    let x = 3 in
    if n > 0 then never_returns (loop_forever ());
    x + n
val do_stuff : int -> int = <fun>
\end{lstlisting}

Thus, we got the compilation to go through by explicitly marking in the
source that the call to \passthrough{\lstinline!loop\_forever!} never
returns.

\hypertarget{improving-the-echo-server}{%
\subsubsection{Improving the Echo
Server}\label{improving-the-echo-server}}

Let's try to go a little bit farther with our echo server by walking
through a few improvements. In particular, we will:

\begin{itemize}
\item
  Add a proper command-line interface with
  \passthrough{\lstinline!Command!}
\item
  Add a flag to specify the port to listen on and a flag to make the
  server echo back the capitalized version of whatever was sent to it
\item
  Simplify the code using Async's \passthrough{\lstinline!Pipe!}
  interface
\end{itemize}

The following code does all of this:

\begin{lstlisting}[language=Caml]
open Core
open Async

let run ~uppercase ~port =
  let host_and_port =
    Tcp.Server.create
      ~on_handler_error:`Raise
      (Tcp.Where_to_listen.of_port port)
      (fun _addr r w ->
        Pipe.transfer (Reader.pipe r) (Writer.pipe w)
           ~f:(if uppercase then String.uppercase else Fn.id))
  in
  ignore (host_and_port : (Socket.Address.Inet.t, int) Tcp.Server.t Deferred.t);
  Deferred.never ()

let () =
  Command.async_spec
    ~summary:"Start an echo server"
    Command.Spec.(
      empty
      +> flag "-uppercase" no_arg
        ~doc:" Convert to uppercase before echoing back"
      +> flag "-port" (optional_with_default 8765 int)
        ~doc:" Port to listen on (default 8765)"
    )
    (fun uppercase port () -> run ~uppercase ~port)
  |> Command.run
\end{lstlisting}

Note the use of \passthrough{\lstinline!Deferred.never!} in the
\passthrough{\lstinline!run!} function. As you might guess from the
name, \passthrough{\lstinline!Deferred.never!} returns a deferred that
is never determined. In this case, that indicates that the echo server
doesn't ever shut down. \index{Deferred.never}

The biggest change in the preceding code is the use of Async's
\passthrough{\lstinline!Pipe!}. A \passthrough{\lstinline!Pipe!} is an
asynchronous communication channel that's used for connecting different
parts of your program. You can think of it as a consumer/producer queue
that uses deferreds for communicating when the pipe is ready to be read
from or written to. Our use of pipes is fairly minimal here, but they
are an important part of Async, so it's worth discussing them in some
detail. \index{pipes}

Pipes are created in connected read/write pairs:

\begin{lstlisting}[language=Caml]
# let (r,w) = Pipe.create ()
val r : '_weak4 Pipe.Reader.t = <abstr>
val w : '_weak4 Pipe.Writer.t = <abstr>
\end{lstlisting}

\passthrough{\lstinline!r!} and \passthrough{\lstinline!w!} are really
just read and write handles to the same underlying object. Note that
\passthrough{\lstinline!r!} and \passthrough{\lstinline!w!} have weakly
polymorphic types, as discussed in
\href{guided-tour.html\#imperative-programming}{Imperative Programming},
and so can only contain values of a single, yet-to-be-determined type.

If we just try and write to the writer, we'll see that we block
indefinitely in \passthrough{\lstinline!utop!}. You can break out of the
wait by hitting \textbf{\passthrough{\lstinline!Control-C!}}:

\begin{lstlisting}[language=Caml]
# Pipe.write w "Hello World!";;
Interrupted.
\end{lstlisting}

The deferred returned by write completes on its own once the value
written into the pipe has been read out:

\begin{lstlisting}[language=Caml]
# let (r,w) = Pipe.create ()
val r : '_weak5 Pipe.Reader.t = <abstr>
val w : '_weak5 Pipe.Writer.t = <abstr>
# let write_complete = Pipe.write w "Hello World!"
val write_complete : unit Deferred.t = <abstr>
# Pipe.read r
- : [ `Eof | `Ok of string ] = `Ok "Hello World!"
# write_complete
- : unit = ()
\end{lstlisting}

In the function \passthrough{\lstinline!run!}, we're taking advantage of
one of the many utility functions provided for pipes in the
\passthrough{\lstinline!Pipe!} module. In particular, we're using
\passthrough{\lstinline!Pipe.transfer!} to set up a process that takes
data from a reader-pipe and moves it to a writer-pipe. Here's the type
of \passthrough{\lstinline!Pipe.transfer!}:

\begin{lstlisting}[language=Caml]
# Pipe.transfer
- : 'a Pipe.Reader.t -> 'b Pipe.Writer.t -> f:('a -> 'b) -> unit Deferred.t =
<fun>
\end{lstlisting}

The two pipes being connected are generated by the
\passthrough{\lstinline!Reader.pipe!} and
\passthrough{\lstinline!Writer.pipe!} call respectively. Note that
pushback is preserved throughout the process, so that if the writer gets
blocked, the writer's pipe will stop pulling data from the reader's
pipe, which will prevent the reader from reading in more data.

Importantly, the deferred returned by
\passthrough{\lstinline!Pipe.transfer!} becomes determined once the
reader has been closed and the last element is transferred from the
reader to the writer. Once that deferred becomes determined, the server
will shut down that client connection. So, when a client disconnects,
the rest of the shutdown happens transparently.

The command-line parsing for this program is based on the Command
library that we introduced in
\href{command-line-parsing.html\#command-line-parsing}{Command Line
Parsing}. Opening \passthrough{\lstinline!Async!}, shadows the
\passthrough{\lstinline!Command!} module with an extended version that
contains the \passthrough{\lstinline!async!} call:

\begin{lstlisting}[language=Caml]
# Command.async_spec
- : ('a, unit Deferred.t) Async.Command.basic_spec_command
    Command.with_options
= <fun>
\end{lstlisting}

This differs from the ordinary \passthrough{\lstinline!Command.basic!}
call in that the main function must return a
\passthrough{\lstinline!Deferred.t!}, and that the running of the
command (using \passthrough{\lstinline!Command.run!}) automatically
starts the Async scheduler, without requiring an explicit call to
\passthrough{\lstinline!Scheduler.go!}.~~

\hypertarget{example-searching-definitions-with-duckduckgo}{%
\subsection{Example: Searching Definitions with
DuckDuckGo}\label{example-searching-definitions-with-duckduckgo}}

DuckDuckGo is a search engine with a freely available search interface.
In this section, we'll use Async to write a small command-line utility
for querying DuckDuckGo to extract definitions for a collection of
terms. \index{cohttp
library}\index{uri library}\index{textwrap library}\index{DuckDuckGo search
engine/additional libraries needed}\index{search engines}

Our code is going to rely on a number of other libraries, all of which
can be installed using opam. Refer to \href{install.html}{the
installation instructions} if you need help on the installation. Here's
the list of libraries we'll
need:\protect\hypertarget{ALduckduck}{}{Async library/DuckDuckGo
searching example}

\begin{description}
\tightlist
\item[\texttt{textwrap}]
A library for wrapping long lines. We'll use this for printing out our
results.
\item[\texttt{uri}]
A library for handling URIs, or ``Uniform Resource Identifiers,'' of
which HTTP URLs are an example.
\item[\texttt{yojson}]
A JSON parsing library that was described in
\href{json.html\#handling-json-data}{Handling Json Data}.
\item[\texttt{cohttp}]
A library for creating HTTP clients and servers. We need Async support,
which comes with the \passthrough{\lstinline!cohttp-async!} package.
\end{description}

Now let's dive into the implementation.

\hypertarget{uri-handling}{%
\subsubsection{URI Handling}\label{uri-handling}}

HTTP URLs, which identify endpoints across the Web, are actually part of
a more general family known as Uniform Resource Identifiers (URIs). The
full URI specification is defined in
\href{http://tools.ietf.org/html/rfc3986}{RFC3986} and is rather
complicated. Luckily, the \passthrough{\lstinline!uri!} library provides
a strongly typed interface that takes care of much of the hassle.
\index{RFC3986}\index{Uniform Resource Identifiers (URIs)}\index{DuckDuckGo search
engine/URI handling in}

We'll need a function for generating the URIs that we're going to use to
query the DuckDuckGo servers:

\begin{lstlisting}[language=Caml]
open Core
open Async

(* Generate a DuckDuckGo search URI from a query string *)
let query_uri query =
  let base_uri = Uri.of_string "http://api.duckduckgo.com/?format=json" in
  Uri.add_query_param base_uri ("q", [query])
\end{lstlisting}

A \passthrough{\lstinline!Uri.t!} is constructed from the
\passthrough{\lstinline!Uri.of\_string!} function, and a query parameter
\passthrough{\lstinline!q!} is added with the desired search query. The
library takes care of encoding the URI correctly when outputting it in
the network protocol.

\hypertarget{parsing-json-strings}{%
\subsubsection{Parsing JSON Strings}\label{parsing-json-strings}}

The HTTP response from DuckDuckGo is in JSON, a common (and thankfully
simple) format that is specified in
\href{http://www.ietf.org/rfc/rfc4627.txt}{RFC4627}. We'll parse the
JSON data using the Yojson library, which was introduced in
\href{json.html\#handling-json-data}{Handling Json Data}.
\index{Yojson library/parsing JSON with}\index{DuckDuckGo search engine/parsing
JSON strings in}\index{RFC4627}

We expect the response from DuckDuckGo to come across as a JSON record,
which is represented by the \passthrough{\lstinline!Assoc!} tag in
Yojson's JSON variant. We expect the definition itself to come across
under either the key ``Abstract'' or ``Definition,'' and so the
following code looks under both keys, returning the first one for which
a nonempty value is defined:

\begin{lstlisting}[language=Caml]
(* Extract the "Definition" or "Abstract" field from the DuckDuckGo results *)
let get_definition_from_json json =
  match Yojson.Safe.from_string json with
  | `Assoc kv_list ->
    let find key =
      begin match List.Assoc.find ~equal:String.equal kv_list key with
      | None | Some (`String "") -> None
      | Some s -> Some (Yojson.Safe.to_string s)
      end
    in
    begin match find "Abstract" with
    | Some _ as x -> x
    | None -> find "Definition"
    end
  | _ -> None
\end{lstlisting}

\hypertarget{executing-an-http-client-query}{%
\subsubsection{Executing an HTTP Client
Query}\label{executing-an-http-client-query}}

Now let's look at the code for dispatching the search queries over HTTP,
using the Cohttp library: \index{query-handlers/executing an HTTP client
query}\index{client queries}\index{HTTP client queries}\index{DuckDuckGo
search engine/executing an HTTP client query in}

\begin{lstlisting}[language=Caml]
(* Execute the DuckDuckGo search *)
let get_definition word =
  Cohttp_async.Client.get (query_uri word)
  >>= fun (_, body) ->
  Cohttp_async.Body.to_string body
  >>| fun string ->
  (word, get_definition_from_json string)
\end{lstlisting}

To better understand what's going on, it's useful to look at the type
for \passthrough{\lstinline!Cohttp\_async.Client.get!}, which we can do
in \passthrough{\lstinline!utop!}:

\begin{lstlisting}[language=Caml]
# #require "cohttp-async"
# Cohttp_async.Client.get
- : ?interrupt:unit Deferred.t ->
    ?ssl_config:Conduit_async.V2.Ssl.Config.t ->
    ?headers:Cohttp.Header.t ->
    Uri.t -> (Cohttp.Response.t * Cohttp_async.Body.t) Deferred.t
= <fun>
\end{lstlisting}

The \passthrough{\lstinline!get!} call takes as a required argument a
URI and returns a deferred value containing a
\passthrough{\lstinline!Cohttp.Response.t!} (which we ignore) and a pipe
reader to which the body of the request will be streamed.

In this case, the HTTP body probably isn't very large, so we call
\passthrough{\lstinline!Cohttp\_async.Body.to\_string!} to collect the
data from the connection as a single deferred string, rather than
consuming the data incrementally.

Running a single search isn't that interesting from a concurrency
perspective, so let's write code for dispatching multiple searches in
parallel. First, we need code for formatting and printing out the search
result:

\begin{lstlisting}[language=Caml]
(* Print out a word/definition pair *)
let print_result (word,definition) =
  printf "%s\n%s\n\n%s\n\n"
    word
    (String.init (String.length word) ~f:(fun _ -> '-'))
    (match definition with
    | None -> "No definition found"
    | Some def ->
      String.concat ~sep:"\n"
        (Wrapper.wrap (Wrapper.make 70) def))
\end{lstlisting}

We use the \passthrough{\lstinline!Wrapper!} module from the
\passthrough{\lstinline!textwrap!} package to do the line wrapping. It
may not be obvious that this routine is using Async, but it does: the
version of \passthrough{\lstinline!printf!} that's called here is
actually Async's specialized \passthrough{\lstinline!printf!} that goes
through the Async scheduler rather than printing directly. The original
definition of \passthrough{\lstinline!printf!} is shadowed by this new
one when you open \passthrough{\lstinline!Async!}. An important side
effect of this is that if you write an Async program and forget to start
the scheduler, calls like \passthrough{\lstinline!printf!} won't
actually generate any output!

The next function dispatches the searches in parallel, waits for the
results, and then prints:

\begin{lstlisting}[language=Caml]
(* Run many searches in parallel, printing out the results after they're all
   done. *)
let search_and_print words =
  Deferred.all (List.map words ~f:get_definition)
  >>| fun results ->
  List.iter results ~f:print_result
\end{lstlisting}

We used \passthrough{\lstinline!List.map!} to call
\passthrough{\lstinline!get\_definition!} on each word, and
\passthrough{\lstinline!Deferred.all!} to wait for all the results.
Here's the type of \passthrough{\lstinline!Deferred.all!}:

\begin{lstlisting}[language=Caml]
# Deferred.all
- : 'a Deferred.t list -> 'a list Deferred.t = <fun>
\end{lstlisting}

The list returned by \passthrough{\lstinline!Deferred.all!} reflects the
order of the deferreds passed to it. As such, the definitions will be
printed out in the same order that the search words are passed in, no
matter what order the queries return in. It also means that no printing
occurs until all results arrive.

We could rewrite this code to print out the results as they're received
(and thus potentially out of order) as follows:

\begin{lstlisting}[language=Caml]
(* Run many searches in parallel, printing out the results as you go *)
let search_and_print words =
  Deferred.all_unit (List.map words ~f:(fun word ->
    get_definition word >>| print_result))
\end{lstlisting}

The difference is that we both dispatch the query and print out the
result in the closure passed to \passthrough{\lstinline!map!}, rather
than wait for all of the results to get back and then print them out
together. We use \passthrough{\lstinline!Deferred.all\_unit!}, which
takes a list of \passthrough{\lstinline!unit!} deferreds and returns a
single \passthrough{\lstinline!unit!} deferred that becomes determined
when every deferred on the input list is determined. We can see the type
of this function in \passthrough{\lstinline!utop!}:

\begin{lstlisting}[language=Caml]
# Deferred.all_unit
- : unit Deferred.t list -> unit Deferred.t = <fun>
\end{lstlisting}

Finally, we create a command-line interface using
\passthrough{\lstinline!Command.async!}:

\begin{lstlisting}[language=Caml]
let () =
  Command.async_spec
    ~summary:"Retrieve definitions from duckduckgo search engine"
    Command.Spec.(
      empty
      +> anon (sequence ("word" %: string))
    )
    (fun words () -> search_and_print words)
  |> Command.run
\end{lstlisting}

And that's all we need for a simple but usable definition searcher:~

\begin{lstlisting}[language=bash]
$ dune exec -- ./search.exe "Concurrent Programming" "OCaml"
Concurrent Programming
----------------------

"Concurrent computing is a form of computing in which several
computations are executed during overlapping time
periods—concurrently—instead of sequentially. This is a property
of a system—this may be an individual program, a computer, or a
network—and there is a separate execution point or \"thread of
control\" for each computation. A concurrent system is one where a
computation can advance without waiting for all other computations to
complete."

OCaml
-----

"OCaml, originally named Objective Caml, is the main implementation of
the programming language Caml, created by Xavier Leroy, Jérôme
Vouillon, Damien Doligez, Didier Rémy, Ascánder Suárez and others
in 1996. A member of the ML language family, OCaml extends the core
Caml language with object-oriented programming constructs."
\end{lstlisting}

\hypertarget{exception-handling}{%
\subsection{Exception Handling}\label{exception-handling}}

When programming with external resources, errors are everywhere.
Everything from a flaky server to a network outage to exhausting of
local resources can lead to a runtime error. When programming in OCaml,
some of these errors will show up explicitly in a function's return
type, and some of them will show up as exceptions. We covered exception
handling in OCaml in \href{error-handling.html\#exceptions}{Exceptions},
but as we'll see, exception handling in a concurrent program presents
some new challenges. \index{exceptions/in concurrent
programming}\index{concurrent programming}\protect\hypertarget{ALexcept}{}{Async
library/exception handling in}

Let's get a better sense of how exceptions work in Async by creating an
asynchronous computation that (sometimes) fails with an exception. The
function \passthrough{\lstinline!maybe\_raise!} blocks for half a
second, and then either throws an exception or returns
\passthrough{\lstinline!unit!}, alternating between the two behaviors on
subsequent calls:

\begin{lstlisting}[language=Caml]
# let maybe_raise =
    let should_fail = ref false in
    fun () ->
      let will_fail = !should_fail in
      should_fail := not will_fail;
      after (Time.Span.of_sec 0.5)
      >>= fun () ->
      if will_fail then raise Exit else return ()
val maybe_raise : unit -> unit Deferred.t = <fun>
# maybe_raise ()
- : unit = ()
# maybe_raise ()
Exception: (monitor.ml.Error Exit ("Caught by monitor block_on_async"))
\end{lstlisting}

In \passthrough{\lstinline!utop!}, the exception thrown by
\passthrough{\lstinline!maybe\_raise ()!} terminates the evaluation of
just that expression, but in a standalone program, an uncaught exception
would bring down the entire process.

So, how could we capture and handle such an exception? You might try to
do this using OCaml's built-in \passthrough{\lstinline!try/with!}
statement, but as you can see that doesn't quite do the trick:

\begin{lstlisting}[language=Caml]
# let handle_error () =
    try
      maybe_raise ()
      >>| fun () -> "success"
    with _ -> return "failure"
val handle_error : unit -> string Deferred.t = <fun>
# handle_error ()
- : string = "success"
# handle_error ()
Exception: (monitor.ml.Error Exit ("Caught by monitor block_on_async"))
\end{lstlisting}

This didn't work because \passthrough{\lstinline!try/with!} only
captures exceptions that are thrown in the code directly executed within
it, while \passthrough{\lstinline!maybe\_raise!} schedules an Async job
to run in the future, and it's that job that throws an exception.

We can capture this kind of asynchronous error using the
\passthrough{\lstinline!try\_with!} function provided by Async:
\index{exceptions/asynchronous errors}

\begin{lstlisting}[language=Caml]
# let handle_error () =
    try_with (fun () -> maybe_raise ())
    >>| function
    | Ok ()   -> "success"
    | Error _ -> "failure"
val handle_error : unit -> string Deferred.t = <fun>
# handle_error ()
- : string = "success"
# handle_error ()
- : string = "failure"
\end{lstlisting}

\passthrough{\lstinline!try\_with f!} takes as its argument a
deferred-returning thunk \passthrough{\lstinline!f!} and returns a
deferred that becomes determined either as \passthrough{\lstinline!Ok!}
of whatever \passthrough{\lstinline!f!} returned, or
\passthrough{\lstinline!Error exn!} if \passthrough{\lstinline!f!} threw
an exception before its return value became determined.

\hypertarget{monitors}{%
\subparagraph{Monitors}\label{monitors}}

\passthrough{\lstinline!try\_with!} is a useful tool for handling
exceptions in Async, but it's not the whole story. All of Async's
exception-handling mechanisms, \passthrough{\lstinline!try\_with!}
included, are built on top of Async's system of \emph{monitors}, which
are inspired by the error-handling mechanism in Erlang of the same name.
Monitors are fairly low-level and are only occasionally used directly,
but it's nonetheless worth understanding how they work. \index{monitors}

In Async, a monitor is a context that determines what to do when there
is an unhandled exception. Every Async job runs within the context of
some monitor, which, when the job is running, is referred to as the
current monitor. When a new Async job is scheduled, say, using
\passthrough{\lstinline!bind!} or \passthrough{\lstinline!map!}, it
inherits the current monitor of the job that spawned it.

Monitors are arranged in a tree---when a new monitor is created (say,
using \passthrough{\lstinline!Monitor.create!}), it is a child of the
current monitor. You can explicitly run jobs within a monitor using
\passthrough{\lstinline!within!}, which takes a thunk that returns a
nondeferred value, or \passthrough{\lstinline!within'!}, which takes a
thunk that returns a deferred. Here's an example:

\begin{lstlisting}[language=Caml]
# let blow_up () =
    let monitor = Monitor.create ~name:"blow up monitor" () in
    within' ~monitor maybe_raise
val blow_up : unit -> unit Deferred.t = <fun>
# blow_up ()
- : unit = ()
# blow_up ()
Exception: (monitor.ml.Error Exit ("Caught by monitor blow up monitor"))
\end{lstlisting}

In addition to the ordinary stack-trace, the exception displays the
trace of monitors through which the exception traveled, starting at the
one we created, called ``blow up monitor.'' The other monitors you see
come from \passthrough{\lstinline!utop!}'s special handling of
deferreds.

Monitors can do more than just augment the error-trace of an exception.
You can also use a monitor to explicitly handle errors delivered to that
monitor. The
\passthrough{\lstinline!Monitor.detach\_and\_get\_error\_stream!} call
is a particularly important one. It detaches the monitor from its
parent, handing back the stream of errors that would otherwise have been
delivered to the parent monitor. This allows one to do custom handling
of errors, which may include reraising errors to the parent. Here is a
very simple example of a function that captures and ignores errors in
the processes it spawns.

\begin{lstlisting}[language=Caml]
# let swallow_error () =
    let monitor = Monitor.create () in
    Stream.iter (Monitor.detach_and_get_error_stream monitor)
      ~f:(fun _exn -> printf "an error happened\n");
    within' ~monitor (fun () ->
      after (Time.Span.of_sec 0.25)
      >>= fun () -> failwith "Kaboom!")
val swallow_error : unit -> 'a Deferred.t = <fun>
\end{lstlisting}

The deferred returned by this function is never determined, since the
computation ends with an exception rather than a return value. That
means that if we run this function in \passthrough{\lstinline!utop!},
we'll never get our prompt back.

We can fix this by using \passthrough{\lstinline!Deferred.any!} along
with a timeout to get a deferred we know will become determined
eventually. \passthrough{\lstinline!Deferred.any!} takes a list of
deferreds, and returns a deferred which will become determined assuming
any of its arguments becomes determined.

\begin{lstlisting}[language=Caml]
# Deferred.any [ after (Time.Span.of_sec 0.5)
               ; swallow_error () ]
an error happened
- : unit = ()
\end{lstlisting}

As you can see, the message ``an error happened'' is printed out before
the timeout expires.

Here's an example of a monitor that passes some exceptions through to
the parent and handles others. Exceptions are sent to the parent using
\passthrough{\lstinline!Monitor.send\_exn!}, with
\passthrough{\lstinline!Monitor.current!} being called to find the
current monitor, which is the parent of the newly created monitor.

\begin{lstlisting}[language=Caml]
# exception Ignore_me
exception Ignore_me
# let swallow_some_errors exn_to_raise =
    let child_monitor  = Monitor.create  () in
    let parent_monitor = Monitor.current () in
    Stream.iter
      (Monitor.detach_and_get_error_stream child_monitor)
      ~f:(fun error ->
        match Monitor.extract_exn error with
        | Ignore_me -> printf "ignoring exn\n"
        | _ -> Monitor.send_exn parent_monitor error);
    within' ~monitor:child_monitor (fun () ->
      after (Time.Span.of_sec 0.25)
      >>= fun () -> raise exn_to_raise)
val swallow_some_errors : exn -> 'a Deferred.t = <fun>
\end{lstlisting}

Note that we use \passthrough{\lstinline!Monitor.extract\_exn!} to grab
the underlying exception that was thrown. Async wraps exceptions it
catches with extra information, including the monitor trace, so you need
to grab the underlying exception if you want to depend on the details of
the original exception thrown.

If we pass in an exception other than
\passthrough{\lstinline!Ignore\_me!}, like, say, the built-in exception
\passthrough{\lstinline!Not\_found!}, then the exception will be passed
to the parent monitor and delivered as usual:

\begin{lstlisting}[language=Caml]
# exception Another_exception
exception Another_exception
# Deferred.any [ after (Time.Span.of_sec 0.5)
               ; swallow_some_errors Another_exception ]
Exception:
(monitor.ml.Error (Another_exception) ("Caught by monitor (id 69)")).
\end{lstlisting}

If instead we use \passthrough{\lstinline!Ignore\_me!}, the exception
will be ignored, and the computation will finish when the timeout
expires.

\begin{lstlisting}[language=Caml]
# Deferred.any [ after (Time.Span.of_sec 0.5)
               ; swallow_some_errors Ignore_me ]
ignoring exn
- : unit = ()
\end{lstlisting}

In practice, you should rarely use monitors directly, and instead use
functions like \passthrough{\lstinline!try\_with!} and
\passthrough{\lstinline!Monitor.protect!} that are built on top of
monitors. One example of a library that uses monitors directly is
\passthrough{\lstinline!Tcp.Server.create!}, which tracks both
exceptions thrown by the logic that handles the network connection and
by the callback for responding to an individual request, in either case
responding to an exception by closing the connection. It is for building
this kind of custom error handling that monitors can be helpful.

\hypertarget{example-handling-exceptions-with-duckduckgo}{%
\subsubsection{Example: Handling Exceptions with
DuckDuckGo}\label{example-handling-exceptions-with-duckduckgo}}

Let's now go back and improve the exception handling of our DuckDuckGo
client. In particular, we'll change it so that any query that fails is
reported without preventing other queries from completing.
\index{exceptions/search
engine example}\index{DuckDuckGo search engine/exception handling in}

The search code as it is fails rarely, so let's make a change that
allows us to trigger failures more predictably. We'll do this by making
it possible to distribute the requests over multiple servers. Then,
we'll handle the errors that occur when one of those servers is
misspecified.

First we'll need to change \passthrough{\lstinline!query\_uri!} to take
an argument specifying the server to connect to:

\begin{lstlisting}[language=Caml]
(* Generate a DuckDuckGo search URI from a query string *)
let query_uri ~server query =
  let base_uri =
    Uri.of_string (String.concat ["http://";server;"/?format=json"])
  in
  Uri.add_query_param base_uri ("q", [query])
\end{lstlisting}

In addition, we'll make the necessary changes to get the list of servers
on the command-line, and to distribute the search queries round-robin
across the list of servers.

Now, let's see what happens when we rebuild the application and run it
two servers, one of which won't respond to the query.

\begin{lstlisting}[language=bash]
$ dune exec -- ./search.exe -servers localhost,api.duckduckgo.com "Concurrent Programming" "OCaml"
(monitor.ml.Error (Unix.Unix_error "Connection refused" connect 127.0.0.1:80)
 ("Raised by primitive operation at file \"duniverse/async_unix/src/unix_syscalls.ml\", line 1046, characters 17-74"
  "Called from file \"duniverse/async_kernel/src/deferred1.ml\", line 17, characters 40-45"
  "Called from file \"duniverse/async_kernel/src/job_queue.ml\", line 170, characters 6-47"
  "Caught by monitor Tcp.close_sock_on_error"))
[1]
\end{lstlisting}

As you can see, we got a ``Connection refused'' failure, which ends the
entire program, even though one of the two queries would have gone
through successfully on its own. We can handle the failures of
individual connections separately by using the
\passthrough{\lstinline!try\_with!} function within each call to
\passthrough{\lstinline!get\_definition!}, as follows:

\begin{lstlisting}[language=Caml]
(* Execute the DuckDuckGo search *)
let get_definition ~server word =
  try_with (fun () ->
    Cohttp_async.Client.get (query_uri ~server word)
    >>= fun (_, body) ->
    Cohttp_async.Body.to_string body
    >>| fun string ->
    (word, get_definition_from_json string))
  >>| function
  | Ok (word,result) -> (word, Ok result)
  | Error _          -> (word, Error "Unexpected failure")
\end{lstlisting}

Here, we first use \passthrough{\lstinline!try\_with!} to capture the
exception, and then use map (the \passthrough{\lstinline!>>|!} operator)
to convert the error into the form we want: a pair whose first element
is the word being searched for, and the second element is the (possibly
erroneous) result.

Now we just need to change the code for
\passthrough{\lstinline!print\_result!} so that it can handle the new
type:

\begin{lstlisting}[language=Caml]
(* Print out a word/definition pair *)
let print_result (word,definition) =
  printf "%s\n%s\n\n%s\n\n"
    word
    (String.init (String.length word) ~f:(fun _ -> '-'))
    (match definition with
     | Error s -> "DuckDuckGo query failed: " ^ s
     | Ok None -> "No definition found"
     | Ok (Some def) ->
       String.concat ~sep:"\n"
         (Wrapper.wrap (Wrapper.make 70) def))
\end{lstlisting}

Now, if we run that same query, we'll get individualized handling of the
connection failures:

\begin{lstlisting}[language=bash]
$ dune exec -- ./search.exe -servers localhost,api.duckduckgo.com "Concurrent Programming" OCaml
Concurrent Programming
----------------------

DuckDuckGo query failed: Unexpected failure

OCaml
-----

"OCaml, originally named Objective Caml, is the main implementation of
the programming language Caml, created by Xavier Leroy, Jérôme
Vouillon, Damien Doligez, Didier Rémy, Ascánder Suárez and others
in 1996. A member of the ML language family, OCaml extends the core
Caml language with object-oriented programming constructs."
\end{lstlisting}

Now, only the query that went to \passthrough{\lstinline!localhost!}
failed.

Note that in this code, we're relying on the fact that
\passthrough{\lstinline!Cohttp\_async.Client.get!} will clean up after
itself after an exception, in particular by closing its file
descriptors. If you need to implement such functionality directly, you
may want to use the \passthrough{\lstinline!Monitor.protect!} call,
which is analogous to the \passthrough{\lstinline!protect!} call
described in
\href{error-handling.html\#cleaning-up-in-the-presence-of-exceptions}{Cleaning
Up In The Presence Of Exceptions}.~

\hypertarget{timeouts-cancellation-and-choices}{%
\subsection{Timeouts, Cancellation, and
Choices}\label{timeouts-cancellation-and-choices}}

In a concurrent program, one often needs to combine results from
multiple, distinct concurrent subcomputations going on in the same
program. We already saw this in our DuckDuckGo example, where we used
\passthrough{\lstinline!Deferred.all!} and
\passthrough{\lstinline!Deferred.all\_unit!} to wait for a list of
deferreds to become determined. Another useful primitive is
\passthrough{\lstinline!Deferred.both!}, which lets you wait until two
deferreds of different types have returned, returning both values as a
tuple. Here, we use the function \passthrough{\lstinline!sec!}, which is
shorthand for creating a time-span equal to a given number of seconds:
\index{errors/timeouts and
cancellations}\index{Deferred.both}\index{cancellations}\index{timeouts}\index{Async
library/timeouts and cancellations}

\begin{lstlisting}[language=Caml]
# let string_and_float = Deferred.both
                           (after (sec 0.5)  >>| fun () -> "A")
  (after (sec 0.25) >>| fun () -> 32.33)
val string_and_float : (string * float) Deferred.t = <abstr>
# string_and_float
- : string * float = ("A", 32.33)
\end{lstlisting}

Sometimes, however, we want to wait only for the first of multiple
events to occur. This happens particularly when dealing with timeouts.
In that case, we can use the call
\passthrough{\lstinline!Deferred.any!}, which, given a list of
deferreds, returns a single deferred that will become determined once
any of the values on the list is determined.

\begin{lstlisting}[language=Caml]
# Deferred.any
  [ (after (sec 0.5) >>| fun () -> "half a second")
  ; (after (sec 1.0) >>| fun () -> "one second")
  ; (after (sec 4.0) >>| fun () -> "four seconds")
  ]
- : string = "half a second"
\end{lstlisting}

Let's use this to add timeouts to our DuckDuckGo searches. The following
code is a wrapper for \passthrough{\lstinline!get\_definition!} that
takes a timeout (in the form of a \passthrough{\lstinline!Time.Span.t!})
and returns either the definition, or, if that takes too long, an error:

\begin{lstlisting}[language=Caml]
let get_definition_with_timeout ~server ~timeout word =
  Deferred.any
    [ (after timeout >>| fun () -> (word,Error "Timed out"))
    ; (get_definition ~server word
       >>| fun (word,result) ->
       let result' = match result with
         | Ok _ as x -> x
         | Error _ -> Error "Unexpected failure"
       in
       (word,result')
      )
    ]
\end{lstlisting}

We use \passthrough{\lstinline!>>|!} above to transform the deferred
values we're waiting for so that \passthrough{\lstinline!Deferred.any!}
can choose between values of the same type.

A problem with this code is that the HTTP query kicked off by
\passthrough{\lstinline!get\_definition!} is not actually shut down when
the timeout fires. As such,
\passthrough{\lstinline!get\_definition\_with\_timeout!} can leak an
open connection. Happily, Cohttp does provide a way of shutting down a
client. You can pass a deferred under the label
\passthrough{\lstinline!interrupt!} to
\passthrough{\lstinline!Cohttp\_async.Client.get!}. Once
\passthrough{\lstinline!interrupt!} is determined, the client connection
will be shut down.

The following code shows how you can change
\passthrough{\lstinline!get\_definition!} and
\passthrough{\lstinline!get\_definition\_with\_timeout!} to cancel the
\passthrough{\lstinline!get!} call if the timeout expires:

\begin{lstlisting}[language=Caml]
(* Execute the DuckDuckGo search *)
let get_definition ~server ~interrupt word =
  try_with (fun () ->
    Cohttp_async.Client.get ~interrupt (query_uri ~server word)
    >>= fun (_, body) ->
    Cohttp_async.Body.to_string body
    >>| fun string ->
    (word, get_definition_from_json string))
  >>| function
  | Ok (word,result) -> (word, Ok result)
  | Error _          -> (word, Error "Unexpected failure")
\end{lstlisting}

Next, we'll modify
\passthrough{\lstinline!get\_definition\_with\_timeout!} to create a
deferred to pass in to \passthrough{\lstinline!get\_definition!}, which
will become determined when our timeout expires:

\begin{lstlisting}[language=Caml]
let get_definition_with_timeout ~server ~timeout word =
  get_definition ~server ~interrupt:(after timeout) word
  >>| fun (word,result) ->
  let result' = match result with
    | Ok _ as x -> x
    | Error _ -> Error "Unexpected failure"
  in
  (word,result')
\end{lstlisting}

This will cause the connection to shutdown cleanly when we time out; but
our code no longer explicitly knows whether or not the timeout has
kicked in. In particular, the error message on a timeout will now be
\passthrough{\lstinline!"Unexpected failure"!} rather than
\passthrough{\lstinline!"Timed out"!}, which it was in our previous
implementation.

We can get more precise handling of timeouts using Async's
\passthrough{\lstinline!choose!} function.
\passthrough{\lstinline!choose!} lets you pick among a collection of
different deferreds, reacting to exactly one of them. Each deferred is
paired, using the function \passthrough{\lstinline!choice!}, with a
function that is called if and only if that deferred is chosen. Here's
the type signature of \passthrough{\lstinline!choice!} and
\passthrough{\lstinline!choose!}:

\begin{lstlisting}[language=Caml]
# choice
- : 'a Deferred.t -> ('a -> 'b) -> 'b Deferred.choice = <fun>
# choose
- : 'a Deferred.choice list -> 'a Deferred.t = <fun>
\end{lstlisting}

Note that there's no guarantee that the winning deferred will be the one
that becomes determined first. But \passthrough{\lstinline!choose!} does
guarantee that only one \passthrough{\lstinline!choice!} will be chosen,
and only the chosen \passthrough{\lstinline!choice!} will execute the
attached function.

In the following example, we use \passthrough{\lstinline!choose!} to
ensure that the \passthrough{\lstinline!interrupt!} deferred becomes
determined if and only if the timeout deferred is chosen. Here's the
code:

\begin{lstlisting}[language=Caml]
let get_definition_with_timeout ~server ~timeout word =
  let interrupt = Ivar.create () in
  choose
    [ choice (after timeout) (fun () ->
       Ivar.fill interrupt ();
       (word,Error "Timed out"))
    ; choice (get_definition ~server ~interrupt:(Ivar.read interrupt) word)
        (fun (word,result) ->
           let result' = match result with
             | Ok _ as x -> x
             | Error _ -> Error "Unexpected failure"
           in
           (word,result')
        )
    ]
\end{lstlisting}

Now, if we run this with a suitably small timeout, we'll see that one
query succeeds and the other fails reporting a timeout:

\begin{lstlisting}[language=bash]
$ dune exec -- ./search.exe "concurrent programming" ocaml -timeout 0.1s
concurrent programming
----------------------

"Concurrent computing is a form of computing in which several
computations are executed during overlapping time
periods—concurrently—instead of sequentially. This is a property
of a system—this may be an individual program, a computer, or a
network—and there is a separate execution point or \"thread of
control\" for each computation. A concurrent system is one where a
computation can advance without waiting for all other computations to
complete."

ocaml
-----

DuckDuckGo query failed: Timed out
\end{lstlisting}

\hypertarget{working-with-system-threads}{%
\subsection{Working with System
Threads}\label{working-with-system-threads}}

Although we haven't worked with them yet, OCaml does have built-in
support for true system threads, i.e., kernel-level threads whose
interleaving is controlled by the operating system. We discussed in the
beginning of the chapter why Async is generally a better choice than
system threads, but even if you mostly use Async, OCaml's system threads
are sometimes necessary, and it's worth understanding them.
\index{parallelism}\index{kernel-level
threads}\index{threads/kernel-level threads}\protect\hypertarget{systhrd}{}{system
threads}\protect\hypertarget{ALsysthr}{}{Async library/system threads
and}

The most surprising aspect of OCaml's system threads is that they don't
afford you any access to physical parallelism. That's because OCaml's
runtime has a single runtime lock that at most one thread can be holding
at a time.

Given that threads don't provide physical parallelism, why are they
useful at all?

The most common reason for using system threads is that there are some
operating system calls that have no nonblocking alternative, which means
that you can't run them directly in a system like Async without blocking
your entire program. For this reason, Async maintains a thread pool for
running such calls. Most of the time, as a user of Async you don't need
to think about this, but it is happening under the covers.
\index{threads/benefits of}

Another reason to have multiple threads is to deal with non-OCaml
libraries that have their own event loop or for another reason need
their own threads. In that case, it's sometimes useful to run some OCaml
code on the foreign thread as part of the communication to your main
program. OCaml's foreign function interface is discussed in more detail
in
\href{foreign-function-interface.html\#foreign-function-interface}{Foreign
Function Interface}.

Another occasional use for system threads is to better interoperate with
compute-intensive OCaml code. In Async, if you have a long-running
computation that never calls \passthrough{\lstinline!bind!} or
\passthrough{\lstinline!map!}, then that computation will block out the
Async runtime until it completes.

One way of dealing with this is to explicitly break up the calculation
into smaller pieces that are separated by binds. But sometimes this
explicit yielding is impractical, since it may involve intrusive changes
to an existing codebase. Another solution is to run the code in question
in a separate thread. Async's \passthrough{\lstinline!In\_thread!}
module provides multiple facilities for doing just this,
\passthrough{\lstinline!In\_thread.run!} being the simplest. We can
simply write: \index{In\_thread module}

\begin{lstlisting}[language=Caml]
# let def = In_thread.run (fun () -> List.range 1 10)
val def : int list Deferred.t = <abstr>
# def
- : int list = [1; 2; 3; 4; 5; 6; 7; 8; 9]
\end{lstlisting}

to cause \passthrough{\lstinline!List.range 1 10!} to be run on one of
Async's worker threads. When the computation is complete, the result is
placed in the deferred, where it can be used in the ordinary way from
Async.

Interoperability between Async and system threads can be quite tricky.
Consider the following function for testing how responsive Async is. The
function takes a deferred-returning thunk, and it first runs that thunk,
and then uses \passthrough{\lstinline!Clock.every!} to wake up every 100
milliseconds and print out a timestamp, until the returned deferred
becomes determined, at which point it prints out one last timestamp:

\begin{lstlisting}[language=Caml]
# let log_delays thunk =
    let start = Time.now () in
    let print_time () =
      let diff = Time.diff (Time.now ()) start in
      printf "%s, " (Time.Span.to_string diff)
    in
    let d = thunk () in
    Clock.every (sec 0.1) ~stop:d print_time;
    d >>= fun () ->
    printf "\nFinished at: ";
    print_time ();
    printf "\n";
    Writer.flushed (force Writer.stdout);
val log_delays : (unit -> unit Deferred.t) -> unit Deferred.t = <fun>
\end{lstlisting}

If we feed this function a simple timeout deferred, it works as you
might expect, waking up roughly every 100 milliseconds:

\begin{lstlisting}[language=Caml]
# log_delays (fun () -> after (sec 0.5))
37.670135498046875us, 100.65722465515137ms, 201.19547843933105ms, 301.85389518737793ms, 402.58693695068359ms,
Finished at: 500.67615509033203ms,
- : unit = ()
\end{lstlisting}

Now see what happens if, instead of waiting on a clock event, we wait
for a busy loop to finish running:

\begin{lstlisting}[language=Caml]
# let busy_loop () =
    let x = ref None in
    for i = 1 to 100_000_000 do x := Some i done
val busy_loop : unit -> unit = <fun>
# log_delays (fun () -> return (busy_loop ()))
Finished at: 874.99594688415527ms,
- : unit = ()
\end{lstlisting}

As you can see, instead of waking up 10 times a second,
\passthrough{\lstinline!log\_delays!} is blocked out entirely while
\passthrough{\lstinline!busy\_loop!} churns away.

If, on the other hand, we use \passthrough{\lstinline!In\_thread.run!}
to offload this to a different system thread, the behavior will be
different:

\begin{lstlisting}[language=Caml]
# log_delays (fun () -> In_thread.run busy_loop)
31.709671020507812us, 107.50102996826172ms, 207.65542984008789ms, 307.95812606811523ms, 458.15873146057129ms, 608.44659805297852ms, 708.55593681335449ms, 808.81166458129883ms,
Finished at: 840.72136878967285ms,
- : unit = ()
\end{lstlisting}

Now \passthrough{\lstinline!log\_delays!} does get a chance to run, but
it's no longer at clean 100 millisecond intervals. The reason is that
now that we're using system threads, we are at the mercy of the
operating system to decide when each thread gets scheduled. The behavior
of threads is very much dependent on the operating system and how it is
configured.

Another tricky aspect of dealing with OCaml threads has to do with
allocation. When compiling to native code, OCaml's threads only get a
chance to give up the runtime lock when they interact with the
allocator, so if there's a piece of code that doesn't allocate at all,
then it will never allow another OCaml thread to run. Bytecode doesn't
have this behavior, so if we run a nonallocating loop in bytecode, our
timer process will get to run:

\begin{lstlisting}[language=Caml]
# let noalloc_busy_loop () =
    for i = 0 to 100_000_000 do () done
val noalloc_busy_loop : unit -> unit = <fun>
# log_delays (fun () -> In_thread.run noalloc_busy_loop)
32.186508178710938us, 116.56808853149414ms, 216.65477752685547ms, 316.83063507080078ms, 417.13213920593262ms,
Finished at: 418.69187355041504ms,
- : unit = ()
\end{lstlisting}

But if we compile this to a native-code executable, then the
nonallocating busy loop will block anything else from running:

\begin{lstlisting}[language=bash]
$ dune exec -- native_code_log_delays.exe
197.41058349609375us,
Finished at: 1.2127914428710938s,
\end{lstlisting}

The takeaway from these examples is that predicting thread interleavings
is a subtle business. Staying within the bounds of Async has its
limitations, but it leads to more predictable behavior.

\hypertarget{thread-safety-and-locking}{%
\subsubsection{Thread-Safety and
Locking}\label{thread-safety-and-locking}}

Once you start working with system threads, you'll need to be careful
about mutable data structures. Most mutable OCaml data structures do not
have well-defined semantics when accessed concurrently by multiple
threads. The issues you can run into range from runtime exceptions to
corrupted data structures to, in some rare cases, segfaults. That means
you should always use mutexes when sharing mutable data between
different systems threads. Even data structures that seem like they
should be safe but are mutable under the covers, like lazy values, can
have undefined behavior when accessed from multiple threads.
\index{mutexes}\index{segfaults}\index{threads/locking
and}\index{threads/thread-safety}

There are two commonly available mutex packages for OCaml: the
\passthrough{\lstinline!Mutex!} module that's part of the standard
library, which is just a wrapper over OS-level mutexes and
\passthrough{\lstinline!Nano\_mutex!}, a more efficient alternative that
takes advantage of some of the locking done by the OCaml runtime to
avoid needing to create an OS-level mutex much of the time. As a result,
creating a \passthrough{\lstinline!Nano\_mutex.t!} is 20 times faster
than creating a \passthrough{\lstinline!Mutex.t!}, and acquiring the
mutex is about 40 percent faster.

Overall, combining Async and threads is quite tricky, but it can be done
safely if the following hold:

\begin{itemize}
\item
  There is no shared mutable state between the various threads involved.
\item
  The computations executed by \passthrough{\lstinline!In\_thread.run!}
  do not make any calls to the Async library.
\end{itemize}

It is possible to safely use threads in ways that violate these
constraints. In particular, foreign threads can acquire the Async lock
using calls from the \passthrough{\lstinline!Thread\_safe!} module in
Async, and thereby run Async computations safely. This is a very
flexible way of connecting threads to the Async world, but it's a
complex use case that is beyond the scope of this chapter. ~~
