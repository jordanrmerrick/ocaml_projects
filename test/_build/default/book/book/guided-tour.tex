\hypertarget{a-guided-tour}{%
\section{A Guided Tour}\label{a-guided-tour}}

This chapter gives an overview of OCaml by walking through a series of
small examples that cover most of the major features of the language.
This should provide a sense of what OCaml can do, without getting too
deep into any one topic.

Throughout the book we're going to use \passthrough{\lstinline!Base!}, a
more full-featured and capable replacement for OCaml's standard library.
We'll also use \passthrough{\lstinline!utop!}, a shell that lets you
type in expressions and evaluate them interactively.
\passthrough{\lstinline!utop!} is an easier-to-use version of OCaml's
standard toplevel (which you can start by typing \emph{ocaml} at the
command line). These instructions will assume you're using
\passthrough{\lstinline!utop!}, but the ordinary toplevel should mostly
work fine.

\hypertarget{base-core-and-core_kernel}{%
\subsection{\texorpdfstring{\texttt{Base}, \texttt{Core} and
\texttt{Core\_kernel}}{Base, Core and Core\_kernel}}\label{base-core-and-core_kernel}}

\passthrough{\lstinline!Base!} is one of a family of three standard
library replacements, each with different use-cases, each building on
the last. Here's a quick summary.

\begin{itemize}
\item
  \emph{\passthrough{\lstinline!Base!}} is designed to provide the
  fundamentals needed from a standard library. It has a variety of basic
  efficient data structures like hash-tables, sets and sequences. It
  also defines the basic idioms for error handling and serialization,
  and contains well organized APIs for every basic data type from
  integers to lazy values. This comes along with a minimum of external
  dependencies, so \passthrough{\lstinline!Base!} just takes seconds to
  build and install. It's also portable, running on every platform that
  OCaml does, including Windows and JavaScript.
\item
  \emph{\passthrough{\lstinline!Core\_kernel!}} extends
  \passthrough{\lstinline!Base!} with many new data structures, like
  heaps, types to represent times and time-zones, support for efficient
  binary serializers, and other capabilities. It's still portable, but
  has many more dependencies, takes longer to build, and will add more
  to the size of your executables.
\item
  \emph{\passthrough{\lstinline!Core!}} is the most full-featured,
  extending \passthrough{\lstinline!Core\_kernel!} with support for a
  variety of UNIX APIs, but only works on UNIX-like OSs, including Linux
  and macOS.
\end{itemize}

We use \passthrough{\lstinline!Base!} in this section, but you should
check out \passthrough{\lstinline!Core!} and
\passthrough{\lstinline!Core\_kernel!}, depending on your requirements.
We'll discuss some of the additional functionality provided by
\passthrough{\lstinline!Core!} and
\passthrough{\lstinline!Core\_kernel!} later in the book.

Before getting started, make sure you have a working OCaml installation
so you can try out the examples as you read through the chapter.

\hypertarget{ocaml-as-a-calculator}{%
\subsection{OCaml as a Calculator}\label{ocaml-as-a-calculator}}

Our first step is to open \passthrough{\lstinline!Base!}:
\index{OCaml/numerical calculations
in}\index{numerical calculations}\index{Core standard library/opening}

\begin{lstlisting}[language=Caml]
# open Base
\end{lstlisting}

By opening \passthrough{\lstinline!Base!}, we make the definitions it
contains available without having to reference
\passthrough{\lstinline!Base!} explicitly. This is required for many of
the examples in the tour and in the remainder of the book.

Now let's try a few simple numerical calculations:

\begin{lstlisting}[language=Caml]
# 3 + 4
- : int = 7
# 8 / 3
- : int = 2
# 3.5 +. 6.
- : float = 9.5
# 30_000_000 / 300_000
- : int = 100
# 3 * 5 > 14
- : bool = true
\end{lstlisting}

By and large, this is pretty similar to what you'd find in any
programming language, but a few things jump right out at you:

\begin{itemize}
\item
  We needed to type \passthrough{\lstinline!;;!} in order to tell the
  toplevel that it should evaluate an expression. This is a peculiarity
  of the toplevel that is not required in standalone programs (though it
  is sometimes helpful to include \passthrough{\lstinline!;;!} to
  improve OCaml's error reporting, by making it more explicit where a
  given top-level declaration was intended to end).
\item
  After evaluating an expression, the toplevel first prints the type of
  the result, and then prints the result itself.
\item
  Function arguments are separated by spaces instead of by parentheses
  and commas, which is more like the UNIX shell than it is like
  traditional programming languages such as C or Java.
\item
  OCaml allows you to place underscores in the middle of numeric
  literals to improve readability. Note that underscores can be placed
  anywhere within a number, not just every three digits.
\item
  OCaml carefully distinguishes between \passthrough{\lstinline!float!},
  the type for floating-point numbers, and
  \passthrough{\lstinline!int!}, the type for integers. The types have
  different literals (\passthrough{\lstinline!6.!} instead of
  \passthrough{\lstinline!6!}) and different infix operators
  (\passthrough{\lstinline!+.!} instead of \passthrough{\lstinline!+!}),
  and OCaml doesn't automatically cast between these types. This can be
  a bit of a nuisance, but it has its benefits, since it prevents some
  kinds of bugs that arise in other languages due to unexpected
  differences between the behavior of \passthrough{\lstinline!int!} and
  \passthrough{\lstinline!float!}. For example, in many languages,
  \passthrough{\lstinline!1 / 3!} is zero, but
  \passthrough{\lstinline!1.0 /. 3.0!} is a third. OCaml requires you to
  be explicit about which operation you're using.
\end{itemize}

We can also create a variable to name the value of a given expression,
using the \passthrough{\lstinline!let!} keyword. This is known as a
\emph{let binding}:

\begin{lstlisting}[language=Caml]
# let x = 3 + 4
val x : int = 7
# let y = x + x
val y : int = 14
\end{lstlisting}

After a new variable is created, the toplevel tells us the name of the
variable (\passthrough{\lstinline!x!} or \passthrough{\lstinline!y!}),
in addition to its type (\passthrough{\lstinline!int!}) and value
(\passthrough{\lstinline!7!} or \passthrough{\lstinline!14!}).

Note that there are some constraints on what identifiers can be used for
variable names. Punctuation is excluded, except for
\passthrough{\lstinline!\_!} and \passthrough{\lstinline!'!}, and
variables must start with a lowercase letter or an underscore. Thus,
these are legal:

\begin{lstlisting}[language=Caml]
# let x7 = 3 + 4
val x7 : int = 7
# let x_plus_y = x + y
val x_plus_y : int = 21
# let x' = x + 1
val x' : int = 8
\end{lstlisting}

The following examples, however, are not legal:

\begin{lstlisting}[language=Caml]
# let Seven = 3 + 4
Line 1, characters 5-10:
Error: Unbound constructor Seven
# let 7x = 7
Line 1, characters 5-7:
Error: Unknown modifier 'x' for literal 7x
# let x-plus-y = x + y
Line 1, characters 7-11:
Error: Syntax error
\end{lstlisting}

This highlights that variables can't be capitalized, can't begin with
numbers, and can't contain dashes.

\hypertarget{functions-and-type-inference}{%
\subsection{Functions and Type
Inference}\label{functions-and-type-inference}}

The \passthrough{\lstinline!let!} syntax can also be used to define a
function:\index{let syntax/function
definition with}\index{functions/defining}

\begin{lstlisting}[language=Caml]
# let square x = x * x
val square : int -> int = <fun>
# square 2
- : int = 4
# square (square 2)
- : int = 16
\end{lstlisting}

Functions in OCaml are values like any other, which is why we use the
\passthrough{\lstinline!let!} keyword to bind a function to a variable
name, just as we use \passthrough{\lstinline!let!} to bind a simple
value like an integer to a variable name. When using
\passthrough{\lstinline!let!} to define a function, the first identifier
after the \passthrough{\lstinline!let!} is the function name, and each
subsequent identifier is a different argument to the function. Thus,
\passthrough{\lstinline!square!} is a function with a single argument.

Now that we're creating more interesting values like functions, the
types have gotten more interesting too.
\passthrough{\lstinline!int -> int!} is a function type, in this case
indicating a function that takes an \passthrough{\lstinline!int!} and
returns an \passthrough{\lstinline!int!}. We can also write functions
that take multiple arguments. (Reminder: Don't forget
\passthrough{\lstinline!open Base!}, or these examples won't work!)
\index{multi-argument
functions}\index{functions/with multiple arguments}

\begin{lstlisting}[language=Caml]
# let ratio x y =
    Float.of_int x /. Float.of_int y
val ratio : int -> int -> float = <fun>
# ratio 4 7
- : float = 0.571428571428571397
\end{lstlisting}

The preceding example also happens to be our first use of modules. Here,
\passthrough{\lstinline!Float.of\_int!} refers to the
\passthrough{\lstinline!of\_int!} function contained in the
\passthrough{\lstinline!Float!} module. This is different from what you
might expect from an object-oriented language, where dot-notation is
typically used for accessing a method of an object. Note that module
names always start with a capital letter.

Modules can also be opened to make their contents available without
explicitly qualifying by the module name. We did that once already, when
we opened \passthrough{\lstinline!Base!} earlier. We can use that to
make this code a little easier to read, both avoiding the repetition of
\passthrough{\lstinline!Float!} above, and avoiding use of the slightly
awkward \passthrough{\lstinline!/.!} operator. In the following example,
we open the \passthrough{\lstinline!Float.O!} module, which has a bunch
of useful operators and functions that are designed to be used in this
kind of context. Note that this causes the standard int-only arithmetic
operators to be shadowed locally.

\begin{lstlisting}[language=Caml]
# let ratio x y =
    let open Float.O in
    of_int x / of_int y
val ratio : int -> int -> float = <fun>
\end{lstlisting}

Note that we used a slightly different syntax for opening the module,
since we were only opening it in the local scope inside the definition
of \passthrough{\lstinline!ratio!}. There's also a more concise syntax
for local opens, as you can see here.

\begin{lstlisting}[language=Caml]
# let ratio x y =
    Float.O.(of_int x / of_int y)
val ratio : int -> int -> float = <fun>
\end{lstlisting}

The notation for the type-signature of a multiargument function may be a
little surprising at first, but we'll explain where it comes from when
we get to function currying in
\href{variables-and-functions.html\#multi-argument-functions}{Multi
Argument Functions}. For the moment, think of the arrows as separating
different arguments of the function, with the type after the final arrow
being the return value. Thus,
\passthrough{\lstinline!int -> int -> float!} describes a function that
takes two \passthrough{\lstinline!int!} arguments and returns a
\passthrough{\lstinline!float!}.

We can also write functions that take other functions as arguments.
Here's an example of a function that takes three arguments: a test
function and two integer arguments. The function returns the sum of the
integers that pass the test:

\begin{lstlisting}[language=Caml]
# let sum_if_true test first second =
    (if test first then first else 0)
    + (if test second then second else 0)
val sum_if_true : (int -> bool) -> int -> int -> int = <fun>
\end{lstlisting}

If we look at the inferred type signature in detail, we see that the
first argument is a function that takes an integer and returns a
boolean, and that the remaining two arguments are integers. Here's an
example of this function in action:

\begin{lstlisting}[language=Caml]
# let even x =
  x % 2 = 0
val even : int -> bool = <fun>
# sum_if_true even 3 4
- : int = 4
# sum_if_true even 2 4
- : int = 6
\end{lstlisting}

Note that in the definition of \passthrough{\lstinline!even!}, we used
\passthrough{\lstinline!=!} in two different ways: once as part of the
\passthrough{\lstinline!let!} binding that separates the thing being
defined from its definition; and once as an equality test, when
comparing \passthrough{\lstinline!x \% 2!} to
\passthrough{\lstinline!0!}. These are very different operations despite
the fact that they share some syntax.

\hypertarget{type-inference}{%
\subsubsection{Type Inference}\label{type-inference}}

As the types we encounter get more complicated, you might ask yourself
how OCaml is able to figure them out, given that we didn't write down
any explicit type information.\index{type inference/process of}

OCaml determines the type of an expression using a technique called
\emph{type inference}, by which the type of an expression is inferred
from the available type information about the components of that
expression.

As an example, let's walk through the process of inferring the type of
\passthrough{\lstinline!sum\_if\_true!}:

\begin{enumerate}
\def\labelenumi{\arabic{enumi}.}
\item
  OCaml requires that both branches of an \passthrough{\lstinline!if!}
  statement have the same type, so the expression

  \passthrough{\lstinline!if test first then first else 0!}

  requires that \passthrough{\lstinline!first!} must be the same type as
  \passthrough{\lstinline!0!}, and so \passthrough{\lstinline!first!}
  must be of type \passthrough{\lstinline!int!}. Similarly, from

  \passthrough{\lstinline!if test second then second else 0!}

  we can infer that \passthrough{\lstinline!second!} has type
  \passthrough{\lstinline!int!}.
\item
  \passthrough{\lstinline!test!} is passed
  \passthrough{\lstinline!first!} as an argument. Since
  \passthrough{\lstinline!first!} has type
  \passthrough{\lstinline!int!}, the input type of
  \passthrough{\lstinline!test!} must be \passthrough{\lstinline!int!}.
\item
  \passthrough{\lstinline!test first!} is used as the condition in an
  \passthrough{\lstinline!if!} statement, so the return type of
  \passthrough{\lstinline!test!} must be \passthrough{\lstinline!bool!}.
\item
  The fact that \passthrough{\lstinline!+!} returns
  \passthrough{\lstinline!int!} implies that the return value of
  \passthrough{\lstinline!sum\_if\_true!} must be int.
\end{enumerate}

Together, that nails down the types of all the variables, which
determines the overall type of \passthrough{\lstinline!sum\_if\_true!}.

Over time, you'll build a rough intuition for how the OCaml inference
engine works, which makes it easier to reason through your programs. You
can also make it easier to understand the types of a given expression by
adding explicit type annotations. These annotations don't change the
behavior of an OCaml program, but they can serve as useful
documentation, as well as catch unintended type changes. They can also
be helpful in figuring out why a given piece of code fails to compile.

Here's an annotated version of \passthrough{\lstinline!sum\_if\_true!}:

\begin{lstlisting}[language=Caml]
# let sum_if_true (test : int -> bool) (x : int) (y : int) : int =
    (if test x then x else 0)
    + (if test y then y else 0)
val sum_if_true : (int -> bool) -> int -> int -> int = <fun>
\end{lstlisting}

In the above, we've marked every argument to the function with its type,
with the final annotation indicating the type of the return value. Such
type annotations can be placed on any expression in an OCaml program.

\hypertarget{inferring-generic-types}{%
\subsubsection{Inferring Generic Types}\label{inferring-generic-types}}

Sometimes, there isn't enough information to fully determine the
concrete type of a given value. Consider this
function..\index{type inference/generic
types}

\begin{lstlisting}[language=Caml]
# let first_if_true test x y =
    if test x then x else y
val first_if_true : ('a -> bool) -> 'a -> 'a -> 'a = <fun>
\end{lstlisting}

\passthrough{\lstinline!first\_if\_true!} takes as its arguments a
function \passthrough{\lstinline!test!}, and two values,
\passthrough{\lstinline!x!} and \passthrough{\lstinline!y!}, where
\passthrough{\lstinline!x!} is to be returned if
\passthrough{\lstinline!test x!} evaluates to
\passthrough{\lstinline!true!}, and \passthrough{\lstinline!y!}
otherwise. So what's the type of the \passthrough{\lstinline!x!}
argument to \passthrough{\lstinline!first\_if\_true!}? There are no
obvious clues such as arithmetic operators or literals to narrow it
down. That makes it seem like \passthrough{\lstinline!first\_if\_true!}
would work on values of any type.

Indeed, if we look at the type returned by the toplevel, we see that
rather than choose a single concrete type, OCaml has introduced a
\emph{type variable}\passthrough{\lstinline!'a!} to express that the
type is generic. (You can tell it's a type variable by the leading
single quote mark.) In particular, the type of the
\passthrough{\lstinline!test!} argument is
\passthrough{\lstinline!('a -> bool)!}, which means that
\passthrough{\lstinline!test!} is a one-argument function whose return
value is \passthrough{\lstinline!bool!} and whose argument could be of
any type \passthrough{\lstinline!'a!}. But, whatever type
\passthrough{\lstinline!'a!} is, it has to be the same as the type of
the other two arguments, \passthrough{\lstinline!x!} and
\passthrough{\lstinline!y!}, and of the return value of
\passthrough{\lstinline!first\_if\_true!}. This kind of genericity is
called \emph{parametric polymorphism} because it works by parameterizing
the type in question with a type variable. It is very similar to
generics in C\# and Java. \index{parametric
polymorphism}\index{type variables}

Because the type of \passthrough{\lstinline!first\_if\_true!} is
generic, we can write this:

\begin{lstlisting}[language=Caml]
# let long_string s = String.length s > 6
val long_string : string -> bool = <fun>
# first_if_true long_string "short" "loooooong"
- : string = "loooooong"
\end{lstlisting}

As well as this:

\begin{lstlisting}[language=Caml]
# let big_number x = x > 3
val big_number : int -> bool = <fun>
# first_if_true big_number 4 3
- : int = 4
\end{lstlisting}

Both \passthrough{\lstinline!long\_string!} and
\passthrough{\lstinline!big\_number!} are functions, and each is passed
to \passthrough{\lstinline!first\_if\_true!} with two other arguments of
the appropriate type (strings in the first example, and integers in the
second). But we can't mix and match two different concrete types for
\passthrough{\lstinline!'a!} in the same use of
\passthrough{\lstinline!first\_if\_true!}:

\begin{lstlisting}[language=Caml]
# first_if_true big_number "short" "loooooong"
Line 1, characters 26-33:
Error: This expression has type string but an expression was expected of type
         int
\end{lstlisting}

In this example, \passthrough{\lstinline!big\_number!} requires that
\passthrough{\lstinline!'a!} be instantiated as
\passthrough{\lstinline!int!}, whereas \passthrough{\lstinline!"short"!}
and \passthrough{\lstinline!"loooooong"!} require that
\passthrough{\lstinline!'a!} be instantiated as
\passthrough{\lstinline!string!}, and they can't both be right at the
same time.

\hypertarget{type-errors-versus-exceptions}{%
\paragraph{Type Errors Versus
Exceptions}\label{type-errors-versus-exceptions}}

There's a big difference in OCaml between errors that are caught at
compile time and those that are caught at runtime. It's better to catch
errors as early as possible in the development process, and compilation
time is best of
all.\index{runtime exceptions vs. type errors}\index{errors/runtime vs. compile
time}\index{exceptions/vs. type errors}\index{type errors vs. exceptions}

Working in the toplevel somewhat obscures the difference between runtime
and compile-time errors, but that difference is still there. Generally,
type errors like this one:

\begin{lstlisting}[language=Caml]
# let add_potato x =
  x + "potato"
Line 2, characters 7-15:
Error: This expression has type string but an expression was expected of type
         int
\end{lstlisting}

are compile-time errors (because \passthrough{\lstinline!+!} requires
that both its arguments be of type \passthrough{\lstinline!int!}),
whereas errors that can't be caught by the type system, like division by
zero, lead to runtime exceptions:

\begin{lstlisting}[language=Caml]
# let is_a_multiple x y =
  x % y = 0
val is_a_multiple : int -> int -> bool = <fun>
# is_a_multiple 8 2
- : bool = true
# is_a_multiple 8 0
Exception:
(Invalid_argument "8 % 0 in core_int.ml: modulus should be positive")
\end{lstlisting}

The distinction here is that type errors will stop you whether or not
the offending code is ever actually executed. Merely defining
\passthrough{\lstinline!add\_potato!} is an error, whereas
\passthrough{\lstinline!is\_a\_multiple!} only fails when it's called,
and then, only when it's called with an input that triggers the
exception.

\hypertarget{tuples-lists-options-and-pattern-matching}{%
\subsection{Tuples, Lists, Options, and Pattern
Matching}\label{tuples-lists-options-and-pattern-matching}}

\hypertarget{tuples}{%
\subsubsection{Tuples}\label{tuples}}

So far we've encountered a handful of basic types like
\passthrough{\lstinline!int!}, \passthrough{\lstinline!float!}, and
\passthrough{\lstinline!string!}, as well as function types like
\passthrough{\lstinline!string -> int!}. But we haven't yet talked about
any data structures. We'll start by looking at a particularly simple
data structure, the tuple. A tuple is an ordered collection of values
that can each be of a different type. You can create a tuple by joining
values together with a comma.
\index{tuples}\index{data structures/tuples}

\begin{lstlisting}[language=Caml]
# let a_tuple = (3,"three")
val a_tuple : int * string = (3, "three")
# let another_tuple = (3,"four",5.)
val another_tuple : int * string * float = (3, "four", 5.)
\end{lstlisting}

(For the mathematically inclined, \passthrough{\lstinline!*!} is used in
the type \passthrough{\lstinline!t * s!} because that type corresponds
to the set of all pairs containing one value of type
\passthrough{\lstinline!t!} and one of type \passthrough{\lstinline!s!}.
In other words, it's the \emph{Cartesian product} of the two types,
which is why we use \passthrough{\lstinline!*!}, the symbol for
product.)

You can extract the components of a tuple using OCaml's pattern-matching
syntax, as shown below:

\begin{lstlisting}[language=Caml]
# let (x,y) = a_tuple
val x : int = 3
val y : string = "three"
\end{lstlisting}

Here, the \passthrough{\lstinline!(x,y)!} on the left-hand side of the
\passthrough{\lstinline!let!} binding is the pattern. This pattern lets
us mint the new variables \passthrough{\lstinline!x!} and
\passthrough{\lstinline!y!}, each bound to different components of the
value being matched. These can now be used in subsequent expressions:

\begin{lstlisting}[language=Caml]
# x + String.length y
- : int = 8
\end{lstlisting}

Note that the same syntax is used both for constructing and for pattern
matching on tuples.

Pattern matching can also show up in function arguments. Here's a
function for computing the distance between two points on the plane,
where each point is represented as a pair of
\passthrough{\lstinline!float!}s. The pattern-matching syntax lets us
get at the values we need with a minimum of fuss:

\begin{lstlisting}[language=Caml]
# let distance (x1,y1) (x2,y2) =
    Float.sqrt ((x1 -. x2) **. 2. +. (y1 -. y2) **. 2.)
val distance : float * float -> float * float -> float = <fun>
\end{lstlisting}

The \passthrough{\lstinline!**.!} operator used above is for raising a
floating-point number to a power.

This is just a first taste of pattern matching. Pattern matching is a
pervasive tool in OCaml, and as you'll see, it has surprising power.

\hypertarget{lists}{%
\subsubsection{Lists}\label{lists}}

Where tuples let you combine a fixed number of items, potentially of
different types, lists let you hold any number of items of the same
type. Consider the following
example:\protect\hypertarget{DSlists}{}{data structures/lists}

\begin{lstlisting}[language=Caml]
# let languages = ["OCaml";"Perl";"C"]
val languages : string list = ["OCaml"; "Perl"; "C"]
\end{lstlisting}

Note that you can't mix elements of different types in the same list,
unlike tuples:

\begin{lstlisting}[language=Caml]
# let numbers = [3;"four";5]
Line 1, characters 18-24:
Error: This expression has type string but an expression was expected of type
         int
\end{lstlisting}

\hypertarget{the-list-module}{%
\paragraph{The List module}\label{the-list-module}}

\passthrough{\lstinline!Base!} comes with a
\passthrough{\lstinline!List!} module that has a rich collection of
functions for working with lists. We can access values from within a
module by using dot notation. For example, this is how we compute the
length of a list:

\begin{lstlisting}[language=Caml]
# List.length languages
- : int = 3
\end{lstlisting}

Here's something a little more complicated. We can compute the list of
the lengths of each language as follows:

\begin{lstlisting}[language=Caml]
# List.map languages ~f:String.length
- : int list = [5; 4; 1]
\end{lstlisting}

\passthrough{\lstinline!List.map!} takes two arguments: a list and a
function for transforming the elements of that list. It returns a new
list with the transformed elements and does not modify the original
list.

Notably, the function passed to \passthrough{\lstinline!List.map!} is
passed under a \emph{labeled argument}\passthrough{\lstinline!\~f!}.
Labeled arguments are specified by name rather than by position, and
thus allow you to change the order in which arguments are presented to a
function without changing its behavior, as you can see
here:\index{arguments/labeled arguments}\index{labeled arguments}

\begin{lstlisting}[language=Caml]
# List.map ~f:String.length languages
- : int list = [5; 4; 1]
\end{lstlisting}

We'll learn more about labeled arguments and why they're important in
\href{variables-and-functions.html\#variables-and-functions}{Variables
And Functions}.

\hypertarget{constructing-lists-with}{%
\paragraph{Constructing lists with ::}\label{constructing-lists-with}}

In addition to constructing lists using brackets, we can use the list
constructor \passthrough{\lstinline!::!} for adding elements to the
front of a list:\index{operators/: :
operator}\index{lists/operator : :}

\begin{lstlisting}[language=Caml]
# "French" :: "Spanish" :: languages
- : string list = ["French"; "Spanish"; "OCaml"; "Perl"; "C"]
\end{lstlisting}

Here, we're creating a new and extended list, not changing the list we
started with, as you can see below:

\begin{lstlisting}[language=Caml]
# languages
- : string list = ["OCaml"; "Perl"; "C"]
\end{lstlisting}

\hypertarget{semicolons-versus-commas}{%
\subparagraph{Semicolons Versus Commas}\label{semicolons-versus-commas}}

Unlike many other languages, OCaml uses semicolons to separate list
elements in lists rather than commas. Commas, instead, are used for
separating elements in a tuple. If you try to use commas in a list,
you'll see that your code compiles but doesn't do quite what you might
expect:\index{commas vs.
semicolons}\index{semicolons vs. commas}

\begin{lstlisting}[language=Caml]
# ["OCaml", "Perl", "C"]
- : (string * string * string) list = [("OCaml", "Perl", "C")]
\end{lstlisting}

In particular, rather than a list of three strings, what we have is a
singleton list containing a three-tuple of strings.

This example uncovers the fact that commas create a tuple, even if there
are no surrounding parens. So, we can write:

\begin{lstlisting}[language=Caml]
# 1,2,3
- : int * int * int = (1, 2, 3)
\end{lstlisting}

to allocate a tuple of integers. This is generally considered poor style
and should be avoided.

The bracket notation for lists is really just syntactic sugar for
\passthrough{\lstinline!::!}. Thus, the following declarations are all
equivalent. Note that \passthrough{\lstinline![]!} is used to represent
the empty list and that \passthrough{\lstinline!::!} is
right-associative:

\begin{lstlisting}[language=Caml]
# [1; 2; 3]
- : int list = [1; 2; 3]
# 1 :: (2 :: (3 :: []))
- : int list = [1; 2; 3]
# 1 :: 2 :: 3 :: []
- : int list = [1; 2; 3]
\end{lstlisting}

The \passthrough{\lstinline!::!} constructor can only be used for adding
one element to the front of the list, with the list terminating at
\passthrough{\lstinline![]!}, the empty list. There's also a list
concatenation operator, \passthrough{\lstinline!@!}, which can
concatenate two lists:

\begin{lstlisting}[language=Caml]
# [1;2;3] @ [4;5;6]
- : int list = [1; 2; 3; 4; 5; 6]
\end{lstlisting}

It's important to remember that, unlike \passthrough{\lstinline!::!},
this is not a constant-time operation. Concatenating two lists takes
time proportional to the length of the first list.

\hypertarget{list-patterns-using-match}{%
\paragraph{List patterns using match}\label{list-patterns-using-match}}

The elements of a list can be accessed through pattern matching. List
patterns are based on the two list constructors,
\passthrough{\lstinline![]!} and \passthrough{\lstinline!::!}. Here's a
simple example:\index{pattern matching/in lists}\index{lists/pattern
matching}

\begin{lstlisting}[language=Caml]
# let my_favorite_language (my_favorite :: the_rest) =
    my_favorite
Lines 1-2, characters 26-16:
Warning 8: this pattern-matching is not exhaustive.
Here is an example of a case that is not matched:
[]
val my_favorite_language : 'a list -> 'a = <fun>
\end{lstlisting}

By pattern matching using \passthrough{\lstinline!::!}, we've isolated
and named the first element of the list
(\passthrough{\lstinline!my\_favorite!}) and the remainder of the list
(\passthrough{\lstinline!the\_rest!}). If you know Lisp or Scheme, what
we've done is the equivalent of using the functions
\passthrough{\lstinline!car!} and \passthrough{\lstinline!cdr!} to
isolate the first element of a list and the remainder of that list.

As you can see, however, the toplevel did not like this definition and
spit out a warning indicating that the pattern is not exhaustive. This
means that there are values of the type in question that won't be
captured by the pattern. The warning even gives an example of a value
that doesn't match the provided pattern, in particular,
\passthrough{\lstinline![]!}, the empty list. If we try to run
\passthrough{\lstinline!my\_favorite\_language!}, we'll see that it
works on nonempty lists and fails on empty ones:

\begin{lstlisting}[language=Caml]
# my_favorite_language ["English";"Spanish";"French"]
- : string = "English"
# my_favorite_language []
Exception: "Match_failure //toplevel//:1:26"
\end{lstlisting}

You can avoid these warnings, and more importantly make sure that your
code actually handles all of the possible cases, by using a
\passthrough{\lstinline!match!} statement instead.

A \passthrough{\lstinline!match!} statement is a kind of juiced-up
version of the \passthrough{\lstinline!switch!} statement found in C and
Java. It essentially lets you list a sequence of patterns, separated by
pipe characters. (The one before the first case is optional.) The
compiler then dispatches to the code following the first matching
pattern. As we've already seen, the pattern can mint new variables that
correspond to parts of the value being matched.

Here's a new version of \passthrough{\lstinline!my\_favorite\_language!}
that uses \passthrough{\lstinline!match!} and doesn't trigger a compiler
warning:

\begin{lstlisting}[language=Caml]
# let my_favorite_language languages =
    match languages with
    | first :: the_rest -> first
    | [] -> "OCaml" (* A good default! *)
val my_favorite_language : string list -> string = <fun>
# my_favorite_language ["English";"Spanish";"French"]
- : string = "English"
# my_favorite_language []
- : string = "OCaml"
\end{lstlisting}

The preceding code also includes our first comment. OCaml comments are
bounded by \passthrough{\lstinline!(*!} and \passthrough{\lstinline!*)!}
and can be nested arbitrarily and cover multiple lines. There's no
equivalent of C++-style single-line comments that are prefixed by
\passthrough{\lstinline!//!}.

The first pattern, \passthrough{\lstinline!first :: the\_rest!}, covers
the case where \passthrough{\lstinline!languages!} has at least one
element, since every list except for the empty list can be written down
with one or more \passthrough{\lstinline!::!}'s. The second pattern,
\passthrough{\lstinline![]!}, matches only the empty list. These cases
are exhaustive, since every list is either empty or has at least one
element, a fact that is verified by the compiler.

\hypertarget{recursive-list-functions}{%
\paragraph{Recursive list functions}\label{recursive-list-functions}}

Recursive functions, or functions that call themselves, are an important
part of working in OCaml or really any functional language. The typical
approach to designing a recursive function is to separate the logic into
a set of \emph{base cases} that can be solved directly and a set of
\emph{inductive cases}, where the function breaks the problem down into
smaller pieces and then calls itself to solve those smaller
problems.\index{recursive functions/list
functions}\index{lists/recursive list functions}

When writing recursive list functions, this separation between the base
cases and the inductive cases is often done using pattern matching.
Here's a simple example of a function that sums the elements of a list:

\begin{lstlisting}[language=Caml]
# let rec sum l =
    match l with
    | [] -> 0                   (* base case *)
    | hd :: tl -> hd + sum tl   (* inductive case *)
val sum : int list -> int = <fun>
# sum [1;2;3]
- : int = 6
\end{lstlisting}

Following the common OCaml idiom, we use \passthrough{\lstinline!hd!} to
refer to the head of the list and \passthrough{\lstinline!tl!} to refer
to the tail. Note that we had to use the \passthrough{\lstinline!rec!}
keyword to allow \passthrough{\lstinline!sum!} to refer to itself. As
you might imagine, the base case and inductive case are different arms
of the match.

Logically, you can think of the evaluation of a simple recursive
function like \passthrough{\lstinline!sum!} almost as if it were a
mathematical equation whose meaning you were unfolding step by step:

\begin{lstlisting}[language=Caml]
sum [1;2;3]
= 1 + sum [2;3]
= 1 + (2 + sum [3])
= 1 + (2 + (3 + sum []))
= 1 + (2 + (3 + 0))
= 1 + (2 + 3)
= 1 + 5
= 6
\end{lstlisting}

This suggests a reasonable mental model for what OCaml is actually doing
to evaluate a recursive function.

We can introduce more complicated list patterns as well. Here's a
function for removing sequential duplicates:

\begin{lstlisting}[language=Caml]
# let rec remove_sequential_duplicates list =
    match list with
    | [] -> []
    | first :: second :: tl ->
      let new_tl = remove_sequential_duplicates (second :: tl) in
      if first = second then new_tl else first :: new_tl
Lines 2-6, characters 5-57:
Warning 8: this pattern-matching is not exhaustive.
Here is an example of a case that is not matched:
_::[]
val remove_sequential_duplicates : int list -> int list = <fun>
\end{lstlisting}

Again, the first arm of the match is the base case, and the second is
the inductive case. Unfortunately, this code has a problem, as indicated
by the warning message. In particular, it doesn't handle one-element
lists. We can fix this warning by adding another case to the match:

\begin{lstlisting}[language=Caml]
# let rec remove_sequential_duplicates list =
    match list with
    | [] -> []
    | [hd] -> [hd]
    | hd1 :: hd2 :: tl ->
      let new_tl = remove_sequential_duplicates (hd2 :: tl) in
      if hd1 = hd2 then new_tl else hd1 :: new_tl
val remove_sequential_duplicates : int list -> int list = <fun>
# remove_sequential_duplicates [1;1;2;3;3;4;4;1;1;1]
- : int list = [1; 2; 3; 4; 1]
\end{lstlisting}

Note that this code used another variant of the list pattern,
\passthrough{\lstinline![hd]!}, to match a list with a single element.
We can do this to match a list with any fixed number of elements; for
example, \passthrough{\lstinline![x;y;z]!} will match any list with
exactly three elements and will bind those elements to the variables
\passthrough{\lstinline!x!}, \passthrough{\lstinline!y!}, and
\passthrough{\lstinline!z!}.

In the last few examples, our list processing code involved a lot of
recursive functions. In practice, this isn't usually necessary. Most of
the time, you'll find yourself happy to use the iteration functions
found in the \passthrough{\lstinline!List!} module. But it's good to
know how to use recursion for when you need to iterate in a new way. ~

\hypertarget{nesting-lets-with-let-and-in}{%
\paragraph{Nesting lets with let and
in}\label{nesting-lets-with-let-and-in}}

\passthrough{\lstinline!new\_tl!} in the above examples was our first
use of \passthrough{\lstinline!let!} to define a new variable within the
body of a function. A \passthrough{\lstinline!let!} paired with an
\passthrough{\lstinline!in!} can be used to introduce a new binding
within any local scope, including a function body. The
\passthrough{\lstinline!in!} marks the beginning of the scope within
which the new variable can be used. Thus, we could write:\index{let
syntax/nested let binding}

\begin{lstlisting}[language=Caml]
# let x = 7 in
  x + x
- : int = 14
\end{lstlisting}

Note that the scope of the \passthrough{\lstinline!let!} binding is
terminated by the double-semicolon, so the value of
\passthrough{\lstinline!x!} is no longer available:

\begin{lstlisting}[language=Caml]
# x
Line 1, characters 1-2:
Error: Unbound value x
\end{lstlisting}

We can also have multiple \passthrough{\lstinline!let!} statements in a
row, each one adding a new variable binding to what came before:

\begin{lstlisting}[language=Caml]
# let x = 7 in
  let y = x * x in
  x + y
- : int = 56
\end{lstlisting}

This kind of nested \passthrough{\lstinline!let!} binding is a common
way of building up a complex expression, with each
\passthrough{\lstinline!let!} naming some component, before combining
them in one final expression.

\hypertarget{options}{%
\subsubsection{Options}\label{options}}

Another common data structure in OCaml is the \emph{option}. An option
is used to express that a value might or might not be present. For
example:\index{options}\index{data structures/options}

\begin{lstlisting}[language=Caml]
# let divide x y =
  if y = 0 then None else Some (x / y)
val divide : int -> int -> int option = <fun>
\end{lstlisting}

The function \passthrough{\lstinline!divide!} either returns
\passthrough{\lstinline!None!} if the divisor is zero, or
\passthrough{\lstinline!Some!} of the result of the division otherwise.
\passthrough{\lstinline!Some!} and \passthrough{\lstinline!None!} are
constructors that let you build optional values, just as
\passthrough{\lstinline!::!} and \passthrough{\lstinline![]!} let you
build lists. You can think of an option as a specialized list that can
only have zero or one elements.

To examine the contents of an option, we use pattern matching, as we did
with tuples and lists. Let's see how this plays out in a small example.
We'll write a function that takes a filename, and returns a version of
that filename with the file extension (the part after the dot)
downcased. We'll base this on the function
\passthrough{\lstinline!String.rsplit2!} to split the string based on
the rightmost period found in the string. Note that
\passthrough{\lstinline!String.rsplit2!} has return type
\passthrough{\lstinline!(string * string) option!}, returning
\passthrough{\lstinline!None!} when no character was found to split on.

\begin{lstlisting}[language=Caml]
# let downcase_extension filename =
    match String.rsplit2 filename ~on:'.' with
    | None -> filename
    | Some (base,ext) ->
      base ^ "." ^ String.lowercase ext
val downcase_extension : string -> string = <fun>
# List.map ~f:downcase_extension
    [ "Hello_World.TXT"; "Hello_World.txt"; "Hello_World" ]
- : string list = ["Hello_World.txt"; "Hello_World.txt"; "Hello_World"]
\end{lstlisting}

Note that we used the \passthrough{\lstinline!^!} operator for
concatenating strings. The concatenation operator is provided as part of
the \passthrough{\lstinline!Pervasives!} module, which is automatically
opened in every OCaml program.

Options are important because they are the standard way in OCaml to
encode a value that might not be there; there's no such thing as a
\passthrough{\lstinline!NullPointerException!} in OCaml. This is
different from most other languages, including Java and C\#, where most
if not all data types are \emph{nullable}, meaning that, whatever their
type is, any given value also contains the possibility of being a null
value. In such languages, null is lurking
everywhere.\index{datatypes/nullable}

In OCaml, however, missing values are explicit. A value of type
\passthrough{\lstinline!string * string!} always contains two
well-defined values of type \passthrough{\lstinline!string!}. If you
want to allow, say, the first of those to be absent, then you need to
change the type to \passthrough{\lstinline!string option * string!}. As
we'll see in \href{error-handling.html\#error-handling}{Error Handling},
this explicitness allows the compiler to provide a great deal of help in
making sure you're correctly handling the possibility of missing data.

\hypertarget{records-and-variants}{%
\subsection{Records and Variants}\label{records-and-variants}}

So far, we've only looked at data structures that were predefined in the
language, like lists and tuples. But OCaml also allows us to define new
data types. Here's a toy example of a data type representing a point in
two-dimensional space:\index{datatypes/defining new}

\begin{lstlisting}[language=Caml]
# type point2d = { x : float; y : float }
type point2d = { x : float; y : float; }
\end{lstlisting}

\passthrough{\lstinline!point2d!} is a \emph{record} type, which you can
think of as a tuple where the individual fields are named, rather than
being defined positionally. Record types are easy enough to
construct:\index{records/record
types}\index{datatypes/record types}

\begin{lstlisting}[language=Caml]
# let p = { x = 3.; y = -4. }
val p : point2d = {x = 3.; y = -4.}
\end{lstlisting}

And we can get access to the contents of these types using pattern
matching:

\begin{lstlisting}[language=Caml]
# let magnitude { x = x_pos; y = y_pos } =
  Float.sqrt (x_pos **. 2. +. y_pos **. 2.)
val magnitude : point2d -> float = <fun>
\end{lstlisting}

The pattern match here binds the variable
\passthrough{\lstinline!x\_pos!} to the value contained in the
\passthrough{\lstinline!x!} field, and the variable
\passthrough{\lstinline!y\_pos!} to the value in the
\passthrough{\lstinline!y!} field.

We can write this more tersely using what's called \emph{field punning}.
In particular, when the name of the field and the name of the variable
it is bound to coincide, we don't have to write them both down. Using
this, our magnitude function can be rewritten as
follows:\index{fields/field punning}

\begin{lstlisting}[language=Caml]
# let magnitude { x; y } = Float.sqrt (x **. 2. +. y **. 2.)
val magnitude : point2d -> float = <fun>
\end{lstlisting}

Alternatively, we can use dot notation for accessing record fields:

\begin{lstlisting}[language=Caml]
# let distance v1 v2 =
  magnitude { x = v1.x -. v2.x; y = v1.y -. v2.y }
val distance : point2d -> point2d -> float = <fun>
\end{lstlisting}

And we can of course include our newly defined types as components in
larger types. Here, for example, are some types for modeling different
geometric objects that contain values of type
\passthrough{\lstinline!point2d!}:

\begin{lstlisting}[language=Caml]
# type circle_desc  = { center: point2d; radius: float }
type circle_desc = { center : point2d; radius : float; }
# type rect_desc    = { lower_left: point2d; width: float; height: float }
type rect_desc = { lower_left : point2d; width : float; height : float; }
# type segment_desc = { endpoint1: point2d; endpoint2: point2d }
type segment_desc = { endpoint1 : point2d; endpoint2 : point2d; }
\end{lstlisting}

Now, imagine that you want to combine multiple objects of these types
together as a description of a multi-object scene. You need some unified
way of representing these objects together in a single type. One way of
doing this is using a \emph{variant}
type:\index{datatypes/variant types}\index{variant
types/combining multiple object types with}

\begin{lstlisting}[language=Caml]
# type scene_element =
    | Circle  of circle_desc
    | Rect    of rect_desc
    | Segment of segment_desc
type scene_element =
    Circle of circle_desc
  | Rect of rect_desc
  | Segment of segment_desc
\end{lstlisting}

The \passthrough{\lstinline!|!} character separates the different cases
of the variant (the first \passthrough{\lstinline!|!} is optional), and
each case has a capitalized tag, like \passthrough{\lstinline!Circle!},
\passthrough{\lstinline!Rect!} or \passthrough{\lstinline!Segment!}, to
distinguish that case from the others.

Here's how we might write a function for testing whether a point is in
the interior of some element of a list of
\passthrough{\lstinline!scene\_element!}s:

\begin{lstlisting}[language=Caml]
# let is_inside_scene_element point scene_element =
    let open Float.O in
    match scene_element with
    | Circle { center; radius } ->
      distance center point < radius
    | Rect { lower_left; width; height } ->
      point.x    > lower_left.x && point.x < lower_left.x + width
      && point.y > lower_left.y && point.y < lower_left.y + height
    | Segment { endpoint1; endpoint2 } -> false

  let is_inside_scene point scene =
    List.exists scene
      ~f:(fun el -> is_inside_scene_element point el)
val is_inside_scene_element : point2d -> scene_element -> bool = <fun>
val is_inside_scene : point2d -> scene_element list -> bool = <fun>
# is_inside_scene {x=3.;y=7.}
  [ Circle {center = {x=4.;y= 4.}; radius = 0.5 } ]
- : bool = false
# is_inside_scene {x=3.;y=7.}
  [ Circle {center = {x=4.;y= 4.}; radius = 5.0 } ]
- : bool = true
\end{lstlisting}

You might at this point notice that the use of
\passthrough{\lstinline!match!} here is reminiscent of how we used
\passthrough{\lstinline!match!} with \passthrough{\lstinline!option!}
and \passthrough{\lstinline!list!}. This is no accident:
\passthrough{\lstinline!option!} and \passthrough{\lstinline!list!} are
just examples of variant types that are important enough to be defined
in the standard library (and in the case of lists, to have some special
syntax).

We also made our first use of an \emph{anonymous function} in the call
to \passthrough{\lstinline!List.exists!}. Anonymous functions are
declared using the \passthrough{\lstinline!fun!} keyword, and don't need
to be explicitly named. Such functions are common in OCaml, particularly
when using iteration functions like
\passthrough{\lstinline!List.exists!}.\index{anonymous
functions}\index{functions/anonymous functions}

The purpose of \passthrough{\lstinline!List.exists!} is to check if
there are any elements of the list in question for which the provided
function evaluates to \passthrough{\lstinline!true!}. In this case,
we're using \passthrough{\lstinline!List.exists!} to check if there is a
scene element within which our point resides.

\hypertarget{base-and-polymorphic-comparison}{%
\subsubsection{\texorpdfstring{\texttt{Base} and polymorphic
comparison}{Base and polymorphic comparison}}\label{base-and-polymorphic-comparison}}

One other thing to notice was the fact that we opened
\passthrough{\lstinline!Float.O!} in the definition of
\passthrough{\lstinline!is\_inside\_scene\_element!}. That allowed us to
use the simple, un-dotted infix operators, but more importantly it
brought the float comparison operators into scope. When using
\passthrough{\lstinline!Base!}, the default comparison operators work
only on integers, and you need to explicitly choose other comparison
operators when you want them. OCaml also offers a special set of
\emph{polymorphic comparison operators} that can work on almost any
type, but those are considered to be problematic, and so are hidden by
default by \passthrough{\lstinline!Base!}. We'll learn more about
polymorphic compare in
\href{lists-and-patterns.html\#terser-and-faster-patterns}{Terser and
Faster Patterns}

\hypertarget{imperative-programming}{%
\subsection{Imperative Programming}\label{imperative-programming}}

The code we've written so far has been almost entirely \emph{pure} or
\emph{functional}, which roughly speaking means that the code in
question doesn't modify variables or values as part of its execution.
Indeed, almost all of the data structures we've encountered are
\emph{immutable}, meaning there's no way in the language to modify them
at all. This is a quite different style from \emph{imperative}
programming, where computations are structured as sequences of
instructions that operate by making modifications to the state of the
program.\index{functional code}\index{pure code}\index{data
structures/immutable}\index{programming/immutable vs. imperative}

Functional code is the default in OCaml, with variable bindings and most
data structures being immutable. But OCaml also has excellent support
for imperative programming, including mutable data structures like
arrays and hash tables, and control-flow constructs like
\passthrough{\lstinline!for!} and \passthrough{\lstinline!while!} loops.

\hypertarget{arrays}{%
\subsubsection{Arrays}\label{arrays}}

Perhaps the simplest mutable data structure in OCaml is the array.
Arrays in OCaml are very similar to arrays in other languages like C:
indexing starts at 0, and accessing or modifying an array element is a
constant-time operation. Arrays are more compact in terms of memory
utilization than most other data structures in OCaml, including lists.
Here's an example:\index{data
structures/arrays}\index{arrays/imperative programming and}\index{imperative
programming/arrays}

\begin{lstlisting}[language=Caml]
# let numbers = [| 1; 2; 3; 4 |]
val numbers : int array = [|1; 2; 3; 4|]
# numbers.(2) <- 4
- : unit = ()
# numbers
- : int array = [|1; 2; 4; 4|]
\end{lstlisting}

The \passthrough{\lstinline!.(i)!} syntax is used to refer to an element
of an array, and the \passthrough{\lstinline!<-!} syntax is for
modification. Because the elements of the array are counted starting at
zero, element {.(2) is} the third element.

The \passthrough{\lstinline!unit!} type that we see in the preceding
code is interesting in that it has only one possible value, written
\passthrough{\lstinline!()!}. This means that a value of type
\passthrough{\lstinline!unit!} doesn't convey any information, and so is
generally used as a placeholder. Thus, we use
\passthrough{\lstinline!unit!} for the return value of an operation like
setting a mutable field that communicates by side effect rather than by
returning a value. It's also used as the argument to functions that
don't require an input value. This is similar to the role that
\passthrough{\lstinline!void!} plays in languages like C and Java.

\hypertarget{mutable-record-fields}{%
\subsubsection{Mutable Record Fields}\label{mutable-record-fields}}

The array is an important mutable data structure, but it's not the only
one. Records, which are immutable by default, can have some of their
fields explicitly declared as mutable. Here's an example of a mutable
data structure for storing a running statistical summary of a collection
of
numbers.\index{imperative programming/mutable record fields}\index{mutable record
fields}\index{data structures/mutable record fields}

\begin{lstlisting}[language=Caml]
# type running_sum =
    { mutable sum: float;
      mutable sum_sq: float; (* sum of squares *)
      mutable samples: int;
    }
type running_sum = {
  mutable sum : float;
  mutable sum_sq : float;
  mutable samples : int;
}
\end{lstlisting}

The fields in \passthrough{\lstinline!running\_sum!} are designed to be
easy to extend incrementally, and sufficient to compute means and
standard deviations, as shown in the following example. Note that there
are two \passthrough{\lstinline!let!} bindings in a row without a double
semicolon between them. That's because the double semicolon is required
only to tell \emph{utop} to process the input, not to separate two
declarations:

\begin{lstlisting}[language=Caml]
# let mean rsum = rsum.sum /. Float.of_int rsum.samples
  let stdev rsum =
    Float.sqrt (rsum.sum_sq /. Float.of_int rsum.samples
  -. (rsum.sum /. Float.of_int rsum.samples) **. 2.)
val mean : running_sum -> float = <fun>
val stdev : running_sum -> float = <fun>
\end{lstlisting}

We also need functions to create and update
\passthrough{\lstinline!running\_sum!}s:

\begin{lstlisting}[language=Caml]
# let create () = { sum = 0.; sum_sq = 0.; samples = 0 }
  let update rsum x =
    rsum.samples <- rsum.samples + 1;
    rsum.sum     <- rsum.sum     +. x;
    rsum.sum_sq  <- rsum.sum_sq  +. x *. x
val create : unit -> running_sum = <fun>
val update : running_sum -> float -> unit = <fun>
\end{lstlisting}

\passthrough{\lstinline!create!} returns a
\passthrough{\lstinline!running\_sum!} corresponding to the empty set,
and \passthrough{\lstinline!update rsum x!} changes
\passthrough{\lstinline!rsum!} to reflect the addition of
\passthrough{\lstinline!x!} to its set of samples by updating the number
of samples, the sum, and the sum of squares.

Note the use of single semicolons to sequence operations. When we were
working purely functionally, this wasn't necessary, but you start
needing it when you're writing imperative code.

Here's an example of \passthrough{\lstinline!create!} and
\passthrough{\lstinline!update!} in action. Note that this code uses
\passthrough{\lstinline!List.iter!}, which calls the function
\passthrough{\lstinline!\~f!} on each element of the provided list:

\begin{lstlisting}[language=Caml]
# let rsum = create ()
val rsum : running_sum = {sum = 0.; sum_sq = 0.; samples = 0}
# List.iter [1.;3.;2.;-7.;4.;5.] ~f:(fun x -> update rsum x)
- : unit = ()
# mean rsum
- : float = 1.33333333333333326
# stdev rsum
- : float = 3.94405318873307698
\end{lstlisting}

It's worth noting that the preceding algorithm is numerically naive and
has poor precision in the presence of cancellation. You can look at this
Wikipedia
\href{http://en.wikipedia.org/wiki/Algorithms_for_calculating_variance}{article
on algorithms for calculating variance} for more details, paying
particular attention to the weighted incremental and parallel
algorithms.

\hypertarget{refs}{%
\subsubsection{Refs}\label{refs}}

We can create a single mutable value by using a
\passthrough{\lstinline!ref!}. The \passthrough{\lstinline!ref!} type
comes predefined in the standard library, but there's nothing really
special about it. It's just a record type with a single mutable field
called
\passthrough{\lstinline!contents!}:\index{records/record types}\index{imperative programming/ref
type}

\begin{lstlisting}[language=Caml]
# let x = { contents = 0 }
val x : int ref = {contents = 0}
# x.contents <- x.contents + 1
- : unit = ()
# x
- : int ref = {contents = 1}
\end{lstlisting}

There are a handful of useful functions and operators defined for
\passthrough{\lstinline!ref!}s to make them more convenient to work
with:

\begin{lstlisting}[language=Caml]
# let x = ref 0  (* create a ref, i.e., { contents = 0 } *)
val x : int ref = {Base.Ref.contents = 0}
# !x             (* get the contents of a ref, i.e., x.contents *)
- : int = 0
# x := !x + 1    (* assignment, i.e., x.contents <- ... *)
- : unit = ()
# !x
- : int = 1
\end{lstlisting}

There's nothing magical with these operators either. You can completely
reimplement the \passthrough{\lstinline!ref!} type and all of these
operators in just a few lines of code:

\begin{lstlisting}[language=Caml]
# type 'a ref = { mutable contents : 'a }
type 'a ref = { mutable contents : 'a; }
# let ref x = { contents = x }
val ref : 'a -> 'a ref = <fun>
# let (!) r = r.contents
val ( ! ) : 'a ref -> 'a = <fun>
# let (:=) r x = r.contents <- x
val ( := ) : 'a ref -> 'a -> unit = <fun>
\end{lstlisting}

The \passthrough{\lstinline!'a!} before the
\passthrough{\lstinline!ref!} indicates that the
\passthrough{\lstinline!ref!} type is polymorphic, in the same way that
lists are polymorphic, meaning it can contain values of any type. The
parentheses around \passthrough{\lstinline"!"} and
\passthrough{\lstinline!:=!} are needed because these are operators,
rather than ordinary functions.\index{parametric polymorphism}

Even though a \passthrough{\lstinline!ref!} is just another record type,
it's important because it is the standard way of simulating the
traditional mutable variables you'll find in most languages. For
example, we can sum over the elements of a list imperatively by calling
\passthrough{\lstinline!List.iter!} to call a simple function on every
element of a list, using a \passthrough{\lstinline!ref!} to accumulate
the results:

\begin{lstlisting}[language=Caml]
# let sum list =
    let sum = ref 0 in
    List.iter list ~f:(fun x -> sum := !sum + x);
    !sum
val sum : int list -> int = <fun>
\end{lstlisting}

This isn't the most idiomatic way to sum up a list, but it shows how you
can use a \passthrough{\lstinline!ref!} in place of a mutable variable.

\hypertarget{for-and-while-loops}{%
\subsubsection{For and While Loops}\label{for-and-while-loops}}

OCaml also supports traditional imperative control-flow constructs like
\passthrough{\lstinline!for and while!} loops. Here, for example, is
some code for permuting an array that uses a
\passthrough{\lstinline!for!} loop. We use the
\passthrough{\lstinline!Random!} module as our source of randomness.
\passthrough{\lstinline!Random!} starts with a default seed, but you can
call \passthrough{\lstinline!Random.self\_init!} to choose a new seed at
random:\index{Random module}\index{while
loops}\index{for loops}\index{imperative programming/for and while
loops}

\begin{lstlisting}[language=Caml]
# let permute array =
    let length = Array.length array in
    for i = 0 to length - 2 do
      (* pick a j to swap with *)
      let j = i + Random.int (length - i) in
      (* Swap i and j *)
      let tmp = array.(i) in
      array.(i) <- array.(j);
      array.(j) <- tmp
    done
val permute : 'a array -> unit = <fun>
\end{lstlisting}

From a syntactic perspective, you should note the keywords that
distinguish a \passthrough{\lstinline!for!} loop:
\passthrough{\lstinline!for!}, \passthrough{\lstinline!to!},
\passthrough{\lstinline!do!}, and \passthrough{\lstinline!done!}.

Here's an example run of this code:

\begin{lstlisting}[language=Caml]
# let ar = Array.init 20 ~f:(fun i -> i)
val ar : int array =
  [|0; 1; 2; 3; 4; 5; 6; 7; 8; 9; 10; 11; 12; 13; 14; 15; 16; 17; 18; 19|]
# permute ar
- : unit = ()
# ar
- : int array =
[|12; 16; 5; 13; 1; 6; 0; 7; 15; 19; 14; 4; 2; 11; 3; 8; 17; 9; 10; 18|]
\end{lstlisting}

OCaml also supports \passthrough{\lstinline!while!} loops, as shown in
the following function for finding the position of the first negative
entry in an array. Note that \passthrough{\lstinline!while!} (like
\passthrough{\lstinline!for!}) is also a keyword:

\begin{lstlisting}[language=Caml]
# let find_first_negative_entry array =
    let pos = ref 0 in
    while !pos < Array.length array && array.(!pos) >= 0 do
      pos := !pos + 1
    done;
    if !pos = Array.length array then None else Some !pos
val find_first_negative_entry : int array -> int option = <fun>
# find_first_negative_entry [|1;2;0;3|]
- : int option = None
# find_first_negative_entry [|1;-2;0;3|]
- : int option = Some 1
\end{lstlisting}

As a side note, the preceding code takes advantage of the fact that
\passthrough{\lstinline!\&\&!}, OCaml's And operator, short-circuits. In
particular, in an expression of the form
\emph{\passthrough{\lstinline!expr1!}}\passthrough{\lstinline!\&\&!}\emph{\passthrough{\lstinline!expr2!}},
\emph{\passthrough{\lstinline!expr2!}} will only be evaluated if
\emph{\passthrough{\lstinline!expr1!}} evaluated to true. Were it not
for that, then the preceding function would result in an out-of-bounds
error. Indeed, we can trigger that out-of-bounds error by rewriting the
function to avoid the short-circuiting:

\begin{lstlisting}[language=Caml]
# let find_first_negative_entry array =
    let pos = ref 0 in
    while
      let pos_is_good = !pos < Array.length array in
      let element_is_non_negative = array.(!pos) >= 0 in
      pos_is_good && element_is_non_negative
    do
      pos := !pos + 1
    done;
    if !pos = Array.length array then None else Some !pos
val find_first_negative_entry : int array -> int option = <fun>
# find_first_negative_entry [|1;2;0;3|]
Exception: (Invalid_argument "index out of bounds")
\end{lstlisting}

The or operator, \passthrough{\lstinline!||!}, short-circuits in a
similar way to \passthrough{\lstinline!\&\&!}.

\hypertarget{a-complete-program}{%
\subsection{A Complete Program}\label{a-complete-program}}

So far, we've played with the basic features of the language via
\passthrough{\lstinline!utop!}. Now we'll show how to create a simple
standalone program. In particular, we'll create a program that sums up a
list of numbers read in from the standard
input.\index{programming/simple standalone example}

Here's the code, which you can save in a file called sum.ml. Note that
we don't terminate expressions with \passthrough{\lstinline!;;!} here,
since it's not required outside the toplevel.

\begin{lstlisting}[language=Caml]
open Base
open Stdio

let rec read_and_accumulate accum =
  let line = In_channel.input_line In_channel.stdin in
  match line with
  | None -> accum
  | Some x -> read_and_accumulate (accum +. Float.of_string x)

let () =
  printf "Total: %F\n" (read_and_accumulate 0.)
\end{lstlisting}

This is our first use of OCaml's input and output routines, and we
needed to open another library, \passthrough{\lstinline!Stdio!}, to get
access to them. The function
\passthrough{\lstinline!read\_and\_accumulate!} is a recursive function
that uses \passthrough{\lstinline!In\_channel.input\_line!} to read in
lines one by one from the standard input, invoking itself at each
iteration with its updated accumulated sum. Note that
\passthrough{\lstinline!input\_line!} returns an optional value, with
\passthrough{\lstinline!None!} indicating the end of the input stream.

After \passthrough{\lstinline!read\_and\_accumulate!} returns, the total
needs to be printed. This is done using the
\passthrough{\lstinline!printf!} command, which provides support for
type-safe format strings, similar to what you'll find in a variety of
languages. The format string is parsed by the compiler and used to
determine the number and type of the remaining arguments that are
{required}. In this case, there is a single formatting directive,
\passthrough{\lstinline!\%F!}, so \passthrough{\lstinline!printf!}
expects one additional argument of type \passthrough{\lstinline!float!}.

\hypertarget{compiling-and-running}{%
\subsubsection{Compiling and Running}\label{compiling-and-running}}

We'll compile our program using \passthrough{\lstinline!dune!}, a build
system that's designed for use with OCaml projects. First, we need to
write a \emph{dune} file to specify the build.

\begin{lstlisting}
(executable
 (name      sum)
 (libraries base stdio))
\end{lstlisting}

All we need to specify is the fact that we're building an executable
rather than a library, the name of the executable, and the name of the
libraries we depend on.

We can now invoke dune to build the executable.

\begin{lstlisting}[language=bash]
$ dune build sum.exe
\end{lstlisting}

The \passthrough{\lstinline!.exe!} suffix indicates that we're building
a native-code executable, which we'll discuss more in
\href{files-modules-and-programs.html\#files-modules-and-programs}{Files
Modules And Programs}. Once the build completes, we can use the
resulting program like any command-line utility. We can feed input to
\passthrough{\lstinline!sum.exe!} by typing in a sequence of numbers,
one per line, hitting \textbf{\passthrough{\lstinline!Ctrl-D!}} when
we're done:

\begin{lstlisting}
$ ./_build/default/sum.exe
1
2
3
94.5
Total: 100.5
\end{lstlisting}

More work is needed to make a really usable command-line program,
including a proper command-line parsing interface and better error
handling, all of which is covered in
\href{command-line-parsing.html\#command-line-parsing}{Command Line
Parsing}.

\hypertarget{where-to-go-from-here}{%
\subsection{Where to Go from Here}\label{where-to-go-from-here}}

That's it for the guided tour! There are plenty of features left and
lots of details to explain, but we hope that you now have a sense of
what to expect from OCaml, and that you'll be more comfortable reading
the rest of the book as a result.
