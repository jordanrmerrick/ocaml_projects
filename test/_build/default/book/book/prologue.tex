\hypertarget{prologue}{%
\section{Prologue}\label{prologue}}

\hypertarget{why-ocaml}{%
\subsection{Why OCaml?}\label{why-ocaml}}

Programming languages matter. They affect the reliability, security, and
efficiency of the code you write, as well as how easy it is to read,
refactor, and extend. The languages you know can also change how you
think, influencing the way you design software even when you're not
using them.

We wrote this book because we believe in the importance of programming
languages, and that OCaml in particular is an important language to
learn. The three of us have been using OCaml in our academic and
professional lives for over 15 years, and in that time we've come to see
it as a secret weapon for building complex software systems. This book
aims to make this secret weapon available to a wider audience, by
providing a clear guide to what you need to know to use OCaml
effectively in the real world.

What makes OCaml special is that it occupies a sweet spot in the space
of programming language designs. It provides a combination of
efficiency, expressiveness and practicality that is matched by no other
language. That is in large part because OCaml is an elegant combination
of a set of language features that have been developed over the last 40
years. These include: \index{OCaml/key features of}

\begin{itemize}
\item
  \emph{Garbage collection} for automatic memory management, now a
  feature of almost every modern, high-level language.
\item
  \emph{First-class functions} that can be passed around like ordinary
  values, as seen in JavaScript, Common Lisp, and C\#.
\item
  \emph{Static type-checking} to increase performance and reduce the
  number of runtime errors, as found in Java and C\#.
\item
  \emph{Parametric polymorphism}, which enables the construction of
  abstractions that work across different data types, similar to
  generics in Java and C\# and templates in {C++.}
\item
  Good support for \emph{immutable programming}, \emph{i.e.},
  programming without making destructive updates to data structures.
  This is present in traditional functional {languages} like Scheme, and
  is also found in distributed, big-data frameworks like Hadoop.
\item
  \emph{Type inference}, so you don't need to annotate every single
  variable in your program with its type. Instead, types are inferred
  based on how a value is used. Available in a limited form in C\# with
  implicitly typed local variables, and in C++11 with its
  \passthrough{\lstinline!auto!} keyword.
\item
  \emph{Algebraic data types} and \emph{pattern matching} to define and
  manipulate complex data structures. Available in Scala and F\#.
\end{itemize}

Some of you will know and love all of these features, and for others
they'll be largely new, but most of you will have seen some of them in
other languages that you've used. As we'll demonstrate over the course
of this book, there is something transformative about having all these
features together and able to interact in a single language. Despite
their importance, these ideas have made only limited inroads into
mainstream languages, and when they do arrive there, like first-class
functions in C\# or parametric polymorphism in Java, it's typically in a
limited and awkward form. The only languages that completely embody
these ideas are \emph{statically typed, functional programming
languages} like OCaml, F\#, Haskell, Scala, and Standard
ML.\index{OCaml/benefits of}

Among this worthy set of languages, OCaml stands apart because it
manages to provide a great deal of power while remaining highly
pragmatic. The compiler has a straightforward compilation strategy that
produces performant code without requiring heavy optimization and
without the complexities of dynamic just-in-time (JIT) compilation.
This, along with OCaml's strict evaluation model, makes runtime behavior
easy to predict. The garbage collector is \emph{incremental}, letting
you avoid large garbage collection (GC)-related pauses, and
\emph{precise}, meaning it will collect all unreferenced data (unlike
many reference-counting collectors), and the runtime is simple and
highly portable.

All of this makes OCaml a great choice for programmers who want to step
up to a better programming language, and at the same time get practical
work done.

\hypertarget{a-brief-history}{%
\subsubsection{A Brief History}\label{a-brief-history}}

OCaml was written in 1996 by Xavier Leroy, Jérôme Vouillon, Damien
Doligez, and Didier Rémy at INRIA in France. It was inspired by a long
line of research into ML starting in the 1960s, and continues to have
deep links to the academic community.\index{OCaml/history of}

ML was originally the \emph{meta language} of the LCF (Logic for
Computable Functions) proof assistant released by Robin Milner in 1972
(at Stanford, and later at Cambridge). ML was turned into a compiler in
order to make it easier to use LCF on different machines, and it was
gradually turned into a full-fledged system of its own by the 1980s.

The first implementation of Caml appeared in 1987. It was created by
Ascánder Suárez and later continued by Pierre Weis and Michel Mauny. In
1990, Xavier Leroy and Damien Doligez built a new implementation called
Caml Light that was based on a bytecode interpreter with a fast,
sequential garbage collector. Over the next few years useful libraries
appeared, such as Michel Mauny's syntax manipulation tools, and this
helped promote the use of Caml in education and research teams.

Xavier Leroy continued extending Caml Light with new features, which
resulted in the 1995 release of Caml Special Light. This improved the
executable efficiency significantly by adding a fast native code
compiler that made Caml's performance competitive with mainstream
languages such as C++. A module system inspired by Standard ML also
provided powerful facilities for abstraction and made larger-scale
programs easier to construct.

The modern OCaml emerged in 1996, when a powerful and elegant object
system was implemented by Didier Rémy and Jérôme Vouillon. This object
system was notable for supporting many common object-oriented idioms in
a statically type-safe way, whereas the same idioms required runtime
checks in languages such as C++ or Java. In 2000, Jacques Garrigue
extended OCaml with several new features such as polymorphic methods,
variants, and labeled and optional arguments.

The last decade has seen OCaml attract a significant user base, and
language improvements have been steadily added to support the growing
commercial and academic {codebases}. First-class modules, Generalized
Algebraic Data Types (GADTs), and dynamic linking have improved the
flexibility of the language. There is also fast native code support for
x86\_64, ARM, PowerPC, and Sparc, making OCaml a good choice for systems
where resource usage, predictability, and performance all matter.

\hypertarget{the-core-standard-library}{%
\subsubsection{\texorpdfstring{The \texttt{Base} Standard
Library}{The Base Standard Library}}\label{the-core-standard-library}}

However good it is, a language on its own isn't enough. You also need a
set of libraries to build your applications on. A common source of
frustration for those learning OCaml is that the standard library that
ships with the compiler is limited, covering only a subset of the
functionality you would expect from a general-purpose standard library.
That's because the standard library isn't really a general-purpose tool;
its fundamental role is in bootstrapping the compiler, and has been
purposefully kept small and portable.

Happily, in the world of open source software, nothing stops alternative
libraries from being written to supplement the compiler-supplied
standard library. \passthrough{\lstinline!Base!} is an example of such a
library, and it's the standard library we'll use through most of this
book. \index{Base standard library}

Jane Street, a company that has been using OCaml for more than 15 years,
developed the code in \passthrough{\lstinline!Base!} for its own
internal use, but from the start designed it with an eye toward being a
general-purpose standard library. Like the OCaml language itself,
\passthrough{\lstinline!Base!} is engineered with correctness,
reliability, and performance in mind. It's also designed to be easy to
install and highly portable. As such, it works on every platform OCaml
does, including UNIX, Mac, Windows, and JavaScript.

\passthrough{\lstinline!Base!} is distributed with a set of syntax
extensions that provide useful new functionality to OCaml, and there are
additional libraries that are designed to work well with it, including
\passthrough{\lstinline!Core!}, an extension to
\passthrough{\lstinline!Base!} that includes support for UNIX-specific
APIs and a wealth of new data structures and tools; and
\passthrough{\lstinline!Async!}, a library for concurrent programming of
the kind that often comes up when building user interfaces or networked
applications. All of these libraries are distributed under a liberal
Apache 2 license to permit free use in hobby, academic, and commercial
settings.

\hypertarget{the-ocaml-platform}{%
\subsubsection{The OCaml Platform}\label{the-ocaml-platform}}

\passthrough{\lstinline!Base!} is a comprehensive and effective standard
library, but there's much more OCaml software out there. A large
community of programmers has been using OCaml since its first release in
1996, and has generated many useful libraries and tools. We'll introduce
some of these libraries in the course of the examples presented in the
book.\index{OCaml/third-party libraries for}

The installation and management of these third-party libraries is made
much easier via a package management tool known as
\href{http://opam.ocaml.org/}{opam}. We'll explain more about opam as
the book unfolds, but it forms the basis of the Platform, which is a set
of tools and libraries that, along with the OCaml compiler, lets you
build real-world applications quickly and effectively.

We'll also use opam for installing the \passthrough{\lstinline!utop!}
command-line interface. This is a modern interactive tool that supports
command history, macro expansion, module completion, and other niceties
that make it much more pleasant to work with the language. We'll be
using \passthrough{\lstinline!utop!} throughout the book to let you step
through the examples interactively.

\hypertarget{about-this-book}{%
\subsection{About This Book}\label{about-this-book}}

\emph{Real World OCaml} is aimed at programmers who have some experience
with conventional programming languages, but not specifically with
statically typed functional programming. Depending on your background,
many of the concepts we cover will be new, including traditional
functional-programming techniques like higher-order functions and
immutable data types, as well as aspects of OCaml's powerful type and
module systems.

If you already know OCaml, this book may surprise you. Core redefines
most of the standard namespace to make better use of the OCaml module
system and expose a number of powerful, reusable data structures by
default. Older OCaml code will still interoperate with Core, but you may
need to adapt it for maximal benefit. All the new code that we write
uses Core, and we believe the Core model is worth learning; it's been
successfully used on large, multimillion-line codebases and removes a
big barrier to building sophisticated applications in OCaml.

Code that uses only the traditional compiler standard library will
always exist, but there are other online resources for learning how that
works. \emph{Real World OCaml} focuses on the techniques the authors
have used in their personal experience to construct scalable, robust
software systems.

\hypertarget{what-to-expect}{%
\subsubsection{What to Expect}\label{what-to-expect}}

\emph{Real World OCaml} is split into three parts:

\begin{itemize}
\item
  Part I covers the language itself, opening with a guided tour designed
  to provide a quick sketch of the language. Don't expect to understand
  everything in the tour; it's meant to give you a taste of many
  different aspects of the language, but the ideas covered there will be
  explained in more depth in the chapters that follow.

  After covering the core language, Part I then moves onto more advanced
  features like modules, functors, and objects, which may take some time
  to digest. Understanding these concepts is important, though. These
  ideas will put you in good stead even beyond OCaml when switching to
  other modern languages, many of which have drawn inspiration from ML.
\item
  Part II builds on the basics by working through useful tools and
  techniques for addressing common practical applications, from
  command-line parsing to asynchronous network programming. Along the
  way, you'll see how some of the {concepts} from Part I are glued
  together into real libraries and tools that combine different features
  of the language to good effect.
\item
  Part III discusses OCaml's runtime system and compiler toolchain. It
  is remarkably simple when compared to some other language
  implementations (such as Java's or .NET's CLR). Reading this part will
  enable you to build very-high-performance systems, or to interface
  with C libraries. This is also where we talk about profiling and
  debugging techniques using tools such as GNU
  \passthrough{\lstinline!gdb!}.
\end{itemize}

\hypertarget{installation-instructions}{%
\subsubsection{Installation
Instructions}\label{installation-instructions}}

\emph{Real World OCaml} uses some tools that we've developed while
writing this book. Some of these resulted in improvements to the OCaml
compiler, which means that you will need to ensure that you have an
up-to-date development environment (using the 4.09.0 version of the
compiler). The installation process is largely automated through the
opam package manager. Instructions on how to it set up and what packages
to install can be found at \href{install.html}{the installation
page}.\index{installation
instructions}\index{OCaml/installation instructions}

\passthrough{\lstinline!Core!} requires a UNIX based operating system,
and so only works on systems like Mac OS X, Linux, FreeBSD, and OpenBSD.
Core includes a portable subset called
\passthrough{\lstinline!Core\_kernel!} which works anywhere OCaml is,
and in particular works on Windows and Javascript. The examples in Part
I of the book will only use \passthrough{\lstinline!Core\_kernel!} and
other highly portable libraries.\index{OCaml/operating system
support}

This book is not intended as a reference manual. We aim to teach you
about the language and about libraries tools and techniques that will
help you be a more effective OCaml programmer. But it's no replacement
for API documentation or the OCaml manual and man pages. You can find
documentation for all of the libraries and tools referenced in the book
\href{https://ocaml.janestreet.com/ocaml-core/}{online}.

\hypertarget{code-examples}{%
\subsubsection{Code Examples}\label{code-examples}}

All of the code examples in this book are available freely online under
a public-domain-like license. You are most welcome to copy and use any
of the snippets as you see fit in your own code, without any attribution
or other restrictions on their use.\index{OCaml/code examples
for}

The full text of the book, along with all of the example code is
available online at
\href{https://github.com/realworldocaml/book}{https://github.com/realworldocaml/book}.

\hypertarget{contributors}{%
\subsection{Contributors}\label{contributors}}

We would especially like to thank the following individuals for
improving \emph{Real World OCaml}:

\begin{itemize}
\item
  Leo White contributed greatly to the content and examples in
  \href{objects.html\#objects}{Objects} and
  \href{classes.html\#classes}{Classes}.
\item
  Jeremy Yallop authored and documented the Ctypes library described in
  \href{foreign-function-interface.html\#foreign-function-interface}{Foreign
  Function Interface}.
\item
  Stephen Weeks is responsible for much of the modular architecture
  behind Core, and his extensive notes formed the basis of
  \href{runtime-memory-layout.html\#memory-representation-of-values}{Memory
  Representation Of Values} and
  \href{garbage-collector.html\#understanding-the-garbage-collector}{Understanding
  The Garbage Collector}.
\item
  Jérémie Dimino, the author of \passthrough{\lstinline!utop!}, the
  interactive command-line interface that is used throughout this book.
  We're particularly grateful for the changes that he pushed through to
  make \passthrough{\lstinline!utop!} work better in the context of the
  book.
\item
  Ashish Agarwal and Christoph Troestler worked on improving the book's
  toolchain. This allowed us to update the book to track changes to
  OCaml and various libraries and tools. Ashish also developed a new and
  improved version of the book's website.
\item
  The many people who collectively submitted over 2400 comments to
  online drafts of this book, through whose efforts countless errors
  were found and fixed.
\end{itemize}
