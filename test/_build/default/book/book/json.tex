\hypertarget{handling-json-data}{%
\section{Handling JSON Data}\label{handling-json-data}}

Data serialization, i.e., converting data to and from a sequence of
bytes that's suitable for writing to disk or sending across the network,
is an important and common programming task. You often have to match
someone else's data format (such as XML), sometimes you need a highly
efficient format, and other times you want something that is easy for
humans to edit. To this end, OCaml libraries provide several techniques
for data serialization depending on what your problem is.
\index{JSON data/basics of}\protect\hypertarget{SERjson}{}{serialization
formats/JSON}\protect\hypertarget{DATjson}{}{data serialization/with
JSON}

We'll start by using the popular and simple JSON data format and then
look at other serialization formats later in the book. This chapter
introduces you to a couple of new techniques that glue together the
basic ideas from Part I of the book by using:

\begin{itemize}
\item
  \emph{Polymorphic variants} to write more extensible libraries and
  protocols (but still retain the ability to extend them if needed)
\item
  \emph{Functional combinators} to compose common operations over data
  structures in a type-safe way
\item
  External tools to generate boilerplate OCaml modules and signatures
  from external specification files
\end{itemize}

\hypertarget{json-basics}{%
\subsection{JSON Basics}\label{json-basics}}

JSON is a lightweight data-interchange format often used in web services
and browsers. It's described in
\href{http://www.ietf.org/rfc/rfc4627.txt}{RFC4627} and is easier to
parse and generate than alternatives such as XML. You'll run into JSON
very often when working with modern web APIs, so we'll cover several
different ways to manipulate it in this chapter. \index{RFC4627}

JSON consists of two basic structures: an unordered collection of
key/value pairs, and an ordered list of values. Values can be strings,
Booleans, floats, integers, or null. Let's see what a JSON record for an
example book description looks like:
\index{values/in JSON data}\index{key/value pairs}

\begin{lstlisting}
{
  "title": "Real World OCaml",
  "tags" : [ "functional programming", "ocaml", "algorithms" ],
  "pages": 450,
  "authors": [
    { "name": "Jason Hickey", "affiliation": "Google" },
    { "name": "Anil Madhavapeddy", "affiliation": "Cambridge"},
    { "name": "Yaron Minsky", "affiliation": "Jane Street"}
  ],
  "is_online": true
}
\end{lstlisting}

The outermost JSON value is usually a record (delimited by the curly
braces) and contains an unordered set of key/value pairs. The keys must
be strings, but values can be any JSON type. In the preceding example,
\passthrough{\lstinline!tags!} is a string list, while the
\passthrough{\lstinline!authors!} field contains a list of records.
Unlike OCaml lists, JSON lists can contain multiple different JSON types
within a single list.

This free-form nature of JSON types is both a blessing and a curse. It's
very easy to generate JSON values, but code that parses them also has to
handle subtle variations in how the values are represented. For example,
what if the preceding \passthrough{\lstinline!pages!} value is actually
represented as a string value of " \passthrough{\lstinline!450!}"
instead of an integer? \index{JSON data/benefits and drawbacks of}

Our first task is to parse the JSON into a more structured OCaml type so
that we can use static typing more effectively. When manipulating JSON
in Python or Ruby, you might write unit tests to check that you have
handled unusual inputs. The OCaml model prefers compile-time static
checking as well as unit tests. For example, using pattern matching can
warn you if you've not checked that a value can be
\passthrough{\lstinline!Null!} as well as contain an actual value.
\index{Yojson
library/installation of}\index{static checking}\index{compile-time static
checking}\index{unit tests}

\hypertarget{installing-the-yojson-library}{%
\subsubsection{Installing the Yojson
Library}\label{installing-the-yojson-library}}

There are several JSON libraries available for OCaml. For this chapter,
we've picked the \href{https://github.com/mjambon/yojson}{Yojson}
library by Martin Jambon. It's easiest to install via OPAM by running
\passthrough{\lstinline!opam install yojson!}. See
\href{install.html}{the installation instructions} if you haven't
already got opam. Once installed, you can open it in the
\passthrough{\lstinline!utop!} toplevel by:

\begin{lstlisting}[language=Caml]
# open Core_kernel
# #require "yojson"
# open Yojson
\end{lstlisting}

\hypertarget{parsing-json-with-yojson}{%
\subsection{Parsing JSON with Yojson}\label{parsing-json-with-yojson}}

The JSON specification has very few data types, and the
\passthrough{\lstinline!Yojson.Basic.t!} type that follows is sufficient
to express any valid JSON structure: \index{JSON
data/parsing with Yojson}\index{Yojson library/parsing JSON with}

\begin{lstlisting}[language=Caml]
type json = [
  | `Assoc of (string * json) list
  | `Bool of bool
  | `Float of float
  | `Int of int
  | `List of json list
  | `Null
  | `String of string
]
\end{lstlisting}

Some interesting properties should leap out at you after reading this
definition:

\begin{itemize}
\item
  The \passthrough{\lstinline!json!} type is \emph{recursive}, which is
  to say that some of the tags refer back to the overall
  \passthrough{\lstinline!json!} type. In particular,
  \passthrough{\lstinline!Assoc!} and \passthrough{\lstinline!List!}
  types can contain references to further JSON values of different
  types. This is unlike the OCaml lists, whose contents must be of a
  uniform type. \index{recursion/in json types}
\item
  The definition specifically includes a \passthrough{\lstinline!Null!}
  variant for empty fields. OCaml doesn't allow null values by default,
  so this must be encoded explicitly.
\item
  The type definition uses polymorphic variants and not normal variants.
  This will become significant later, when we extend it with custom
  extensions to the JSON format.
  \index{polymorphic variant types/in JSON data}
\end{itemize}

Let's parse the earlier JSON example into this type now. The first stop
is the \passthrough{\lstinline!Yojson.Basic!} documentation, where we
find these helpful functions:

\begin{lstlisting}[language=Caml]
val from_string : ?buf:Bi_outbuf.t -> ?fname:string -> ?lnum:int ->
   string -> json
(* Read a JSON value from a string.
   [buf]   : use this buffer at will during parsing instead of
             creating a new one.
   [fname] : data file name to be used in error messages. It does not
             have to be a real file.
   [lnum]  : number of the first line of input. Default is 1. *)

val from_file : ?buf:Bi_outbuf.t -> ?fname:string -> ?lnum:int ->
   string -> json
(* Read a JSON value from a file. See [from_string] for the meaning of the optional
   arguments. *)

val from_channel : ?buf:Bi_outbuf.t -> ?fname:string -> ?lnum:int ->
  in_channel -> json
  (** Read a JSON value from a channel.
      See [from_string] for the meaning of the optional arguments. *)
\end{lstlisting}

When first reading these interfaces, you can generally ignore the
optional arguments (which have the question marks in the type
signature), since they should have sensible defaults. In the preceding
signature, the optional arguments offer finer control over the memory
buffer allocation and error messages from parsing incorrect JSON.

The type signature for these functions with the optional elements
removed makes their purpose much clearer. The three ways of parsing JSON
are either directly from a string, from a file on a filesystem, or via a
buffered input channel:

\begin{lstlisting}[language=Caml]
val from_string  : string     -> json
val from_file    : string     -> json
val from_channel : in_channel -> json
\end{lstlisting}

The next example shows both the \passthrough{\lstinline!string!} and
\passthrough{\lstinline!file!} functions in action, assuming the JSON
record is stored in a file called \emph{book.json}:

\begin{lstlisting}[language=Caml]
open Core

let () =
  (* Read JSON file into an OCaml string *)
  let buf = In_channel.read_all "book.json" in
  (* Use the string JSON constructor *)
  let json1 = Yojson.Basic.from_string buf in
  (* Use the file JSON constructor *)
  let json2 = Yojson.Basic.from_file "book.json" in
  (* Test that the two values are the same *)
  print_endline (if json1 = json2 then "OK" else "FAIL")
\end{lstlisting}

You can build this by running \passthrough{\lstinline!dune!}:

\begin{lstlisting}
(executable
  (name      read_json)
  (libraries core yojson))
\end{lstlisting}

\begin{lstlisting}[language=bash]
\end{lstlisting}

The \passthrough{\lstinline!from\_file!} function accepts an input
filename and takes care of opening and closing it for you. It's far more
common to use \passthrough{\lstinline!from\_string!} to construct JSON
values though, since these strings come in via a network connection
(we'll see more of this in
\href{concurrent-programming.html\#concurrent-programming-with-async}{Concurrent
Programming With Async}) or a database. Finally, the example checks that
the two input mechanisms actually resulted in the same OCaml data
structure.

\hypertarget{selecting-values-from-json-structures}{%
\subsection{Selecting Values from JSON
Structures}\label{selecting-values-from-json-structures}}

Now that we've figured out how to parse the example JSON into an OCaml
value, let's manipulate it from OCaml code and extract specific fields:
\protect\hypertarget{VALjson}{}{values/selecting from JSON
structures}\protect\hypertarget{JSONselval}{}{JSON data/selecting values
from}

\begin{lstlisting}[language=Caml]
open Core

let () =
  (* Read the JSON file *)
  let json = Yojson.Basic.from_file "book.json" in

  (* Locally open the JSON manipulation functions *)
  let open Yojson.Basic.Util in
  let title = json |> member "title" |> to_string in
  let tags = json |> member "tags" |> to_list |> filter_string in
  let pages = json |> member "pages" |> to_int in
  let is_online = json |> member "is_online" |> to_bool_option in
  let is_translated = json |> member "is_translated" |> to_bool_option in
  let authors = json |> member "authors" |> to_list in
  let names = List.map authors ~f:(fun json -> member "name" json |> to_string) in

  (* Print the results of the parsing *)
  printf "Title: %s (%d)\n" title pages;
  printf "Authors: %s\n" (String.concat ~sep:", " names);
  printf "Tags: %s\n" (String.concat ~sep:", " tags);
  let string_of_bool_option =
    function
    | None -> "<unknown>"
    | Some true -> "yes"
    | Some false -> "no" in
  printf "Online: %s\n" (string_of_bool_option is_online);
  printf "Translated: %s\n" (string_of_bool_option is_translated)
\end{lstlisting}

Now build and run this in the same way as the previous example:

\begin{lstlisting}
(executable
  (name      parse_book)
  (libraries core yojson))
\end{lstlisting}

\begin{lstlisting}[language=bash]
$ dune build parse_book.exe
$ ./_build/default/parse_book.exe
Title: Real World OCaml (450)
Authors: Jason Hickey, Anil Madhavapeddy, Yaron Minsky
Tags: functional programming, ocaml, algorithms
Online: yes
Translated: <unknown>
\end{lstlisting}

This code introduces the \passthrough{\lstinline!Yojson.Basic.Util!}
module, which contains \emph{combinator} functions that let you easily
map a JSON object into a more strongly typed OCaml value.
\index{combinators/functional
combinators}\index{functional combinators}

\hypertarget{functional-combinators}{%
\subparagraph{Functional Combinators}\label{functional-combinators}}

Combinators are a design pattern that crops up quite often in functional
programming. John Hughes defines them as ``a function which builds
program fragments from program fragments.'' In a functional language,
this generally means higher-order functions that combine other functions
to apply useful transformations over values.

You've already run across several of these in the
\passthrough{\lstinline!List!} module:

\begin{lstlisting}[language=Caml]
val map  : 'a list -> f:('a -> 'b)   -> 'b list
val fold : 'a list -> init:'accum -> f:('accum -> 'a -> 'accum) -> 'accum
\end{lstlisting}

\passthrough{\lstinline!map!} and \passthrough{\lstinline!fold!} are
extremely common combinators that transform an input list by applying a
function to each value of the list. The \passthrough{\lstinline!map!}
combinator is simplest, with the resulting list being output directly.
\passthrough{\lstinline!fold!} applies each value in the input list to a
function that accumulates a single result, and returns that instead:

\begin{lstlisting}[language=Caml]
val iter : 'a list -> f:('a -> unit) -> unit
\end{lstlisting}

\passthrough{\lstinline!iter!} is a more specialized combinator that is
only useful when writing imperative code. The input function is applied
to every value, but no result is supplied. The function must instead
apply some side effect such as changing a mutable record field or
printing to the standard output.

\passthrough{\lstinline!Yojson!} provides several combinators in the
\passthrough{\lstinline!Yojson.Basic.Util!} module, some of which are
listed in \href{json.html\#table15_1}{Table15\_1}.
\index{combinators/in Yojson library}\index{Yojson library/combinators in}

\hypertarget{table15_1}{}
\begin{longtable}[]{@{}lll@{}}
\caption{Yojson combinators}\tabularnewline
\toprule
\begin{minipage}[b]{0.36\columnwidth}\raggedright
Function\strut
\end{minipage} & \begin{minipage}[b]{0.24\columnwidth}\raggedright
Type\strut
\end{minipage} & \begin{minipage}[b]{0.32\columnwidth}\raggedright
Purpose\strut
\end{minipage}\tabularnewline
\midrule
\endfirsthead
\toprule
\begin{minipage}[b]{0.36\columnwidth}\raggedright
Function\strut
\end{minipage} & \begin{minipage}[b]{0.24\columnwidth}\raggedright
Type\strut
\end{minipage} & \begin{minipage}[b]{0.32\columnwidth}\raggedright
Purpose\strut
\end{minipage}\tabularnewline
\midrule
\endhead
\begin{minipage}[t]{0.36\columnwidth}\raggedright
member\strut
\end{minipage} & \begin{minipage}[t]{0.24\columnwidth}\raggedright
\passthrough{\lstinline!string -> json -> json!}\strut
\end{minipage} & \begin{minipage}[t]{0.32\columnwidth}\raggedright
Select a named field from a JSON record.\strut
\end{minipage}\tabularnewline
\begin{minipage}[t]{0.36\columnwidth}\raggedright
to\_string\strut
\end{minipage} & \begin{minipage}[t]{0.24\columnwidth}\raggedright
\passthrough{\lstinline!json -> string!}\strut
\end{minipage} & \begin{minipage}[t]{0.32\columnwidth}\raggedright
Convert a JSON value into an OCaml \passthrough{\lstinline!string!}.
Raises an exception if this is impossible.\strut
\end{minipage}\tabularnewline
\begin{minipage}[t]{0.36\columnwidth}\raggedright
to\_int\strut
\end{minipage} & \begin{minipage}[t]{0.24\columnwidth}\raggedright
\passthrough{\lstinline!json -> int!}\strut
\end{minipage} & \begin{minipage}[t]{0.32\columnwidth}\raggedright
Convert a JSON value into an OCaml \passthrough{\lstinline!int!}. Raises
an exception if this is impossible.\strut
\end{minipage}\tabularnewline
\begin{minipage}[t]{0.36\columnwidth}\raggedright
filter\_string\strut
\end{minipage} & \begin{minipage}[t]{0.24\columnwidth}\raggedright
\passthrough{\lstinline!json list -> string list!}\strut
\end{minipage} & \begin{minipage}[t]{0.32\columnwidth}\raggedright
Filter valid strings from a list of JSON fields, and return them as an
OCaml list of strings.\strut
\end{minipage}\tabularnewline
\bottomrule
\end{longtable}

We'll go through each of these uses one by one now. The following
examples also use the \passthrough{\lstinline!|>!} pipe-forward operator
that we explained in
\href{variables-and-functions.html\#variables-and-functions}{Variables
And Functions}. This lets us chain together multiple JSON selection
functions and feed the output from one into the next one, without having
to create separate \passthrough{\lstinline!let!} bindings for each one.
\index{filter\_string function}\index{to\_init
function}\index{functions/to\_init function}\index{to\_string
function}\index{functions/to\_string function}\index{functions/member
functions}\index{member function}

Let's start with selecting a single \passthrough{\lstinline!title!}
field from the record:

\begin{lstlisting}[language=Caml]
# open Yojson.Basic.Util
# let title = json |> member "title" |> to_string
val title : string = "Real World OCaml"
\end{lstlisting}

The \passthrough{\lstinline!member!} function accepts a JSON object and
named key and returns the JSON field associated with that key, or
\passthrough{\lstinline!Null!}. Since we know that the
\passthrough{\lstinline!title!} value is always a string in our example
schema, we want to convert it to an OCaml string. The
\passthrough{\lstinline!to\_string!} function performs this conversion
and raises an exception if there is an unexpected JSON type. The
\passthrough{\lstinline!|>!} operator provides a convenient way to chain
these operations {together}:

\begin{lstlisting}[language=Caml]
# let tags = json |> member "tags" |> to_list |> filter_string
val tags : string list = ["functional programming"; "ocaml"; "algorithms"]
# let pages = json |> member "pages" |> to_int
val pages : int = 450
\end{lstlisting}

The \passthrough{\lstinline!tags!} field is similar to
\passthrough{\lstinline!title!}, but the field is a list of strings
instead of a single one. Converting this to an OCaml
\passthrough{\lstinline!string list!} is a two-stage process. First, we
convert the JSON \passthrough{\lstinline!List!} to an OCaml list of JSON
values and then filter out the \passthrough{\lstinline!String!} values
as an OCaml \passthrough{\lstinline!string list!}. Remember that OCaml
lists must contain values of the same type, so any JSON values that
cannot be converted to a \passthrough{\lstinline!string!} will be
skipped from the output of \passthrough{\lstinline!filter\_string!}:

\begin{lstlisting}[language=Caml]
# let is_online = json |> member "is_online" |> to_bool_option
val is_online : bool option = Some true
# let is_translated = json |> member "is_translated" |> to_bool_option
val is_translated : bool option = None
\end{lstlisting}

The \passthrough{\lstinline!is\_online!} and
\passthrough{\lstinline!is\_translated!} fields are optional in our JSON
schema, so no error should be raised if they are not present. The OCaml
type is a \passthrough{\lstinline!bool option!} to reflect this and can
be extracted via \passthrough{\lstinline!to\_bool\_option!}. In our
example JSON, only \passthrough{\lstinline!is\_online!} is present and
\passthrough{\lstinline!is\_translated!} will be
\passthrough{\lstinline!None!}:

\begin{lstlisting}[language=Caml]
# let authors = json |> member "authors" |> to_list
val authors : Yojson.Basic.t list =
  [`Assoc
     [("name", `String "Jason Hickey"); ("affiliation", `String "Google")];
   `Assoc
     [("name", `String "Anil Madhavapeddy");
      ("affiliation", `String "Cambridge")];
   `Assoc
     [("name", `String "Yaron Minsky");
      ("affiliation", `String "Jane Street")]]
\end{lstlisting}

The final use of JSON combinators is to extract all the
\passthrough{\lstinline!name!} fields from the list of authors. We first
construct the \passthrough{\lstinline!author list!}, and then
\passthrough{\lstinline!map!} it into a
\passthrough{\lstinline!string list!}. Notice that the example
explicitly binds \passthrough{\lstinline!authors!} to a variable name.
It can also be written more succinctly using the pipe-forward operator:

\begin{lstlisting}[language=Caml]
# let names =
    json |> member "authors" |> to_list
  |> List.map ~f:(fun json -> member "name" json |> to_string)
val names : string list =
  ["Jason Hickey"; "Anil Madhavapeddy"; "Yaron Minsky"]
\end{lstlisting}

This style of programming, which omits variable names and chains
functions together, is known as \emph{point-free programming}. It's a
succinct style but shouldn't be overused due to the increased difficulty
of debugging intermediate values. If an explicit {name} is assigned to
each stage of the transformations, debuggers in particular have an
easier time making the program flow simpler to represent to the
programmer.

This technique of using statically typed parsing functions is very
powerful in combination with the OCaml type system. Many errors that
don't make sense at runtime (for example, mixing up lists and objects)
will be caught statically via a type error. ~~

\hypertarget{constructing-json-values}{%
\subsection{Constructing JSON Values}\label{constructing-json-values}}

Building and printing JSON values is pretty straightforward given the
\passthrough{\lstinline!Yojson.Basic.t!} type. You can just construct
values of type \passthrough{\lstinline!t!} and call the
\passthrough{\lstinline!to\_string!} function on them. Let's remind
ourselves of the \passthrough{\lstinline!Yojson.Basic.t!} type again:
\index{values/in JSON data}\index{JSON
data/constructing values}

\begin{lstlisting}[language=Caml]
type json = [
  | `Assoc of (string * json) list
  | `Bool of bool
  | `Float of float
  | `Int of int
  | `List of json list
  | `Null
  | `String of string
]
\end{lstlisting}

We can directly build a JSON value against this type and use the
pretty-printing functions in the \passthrough{\lstinline!Yojson.Basic!}
module to display JSON output:

\begin{lstlisting}[language=Caml]
# let person = `Assoc [ ("name", `String "Anil") ]
val person : [> `Assoc of (string * [> `String of string ]) list ] =
  `Assoc [("name", `String "Anil")]
\end{lstlisting}

In the preceding example, we've constructed a simple JSON object that
represents a single person. We haven't actually defined the type of
\passthrough{\lstinline!person!} explicitly, as we're relying on the
magic of polymorphic variants to make this all work.

The OCaml type system infers a type for \passthrough{\lstinline!person!}
based on how you construct its value. In this case, only the
\passthrough{\lstinline!Assoc!} and \passthrough{\lstinline!String!}
variants are used to define the record, and so the inferred type only
contains these fields without knowledge of the other possible allowed
variants in JSON records that you haven't used yet
(e.g.~\passthrough{\lstinline!Int!} or \passthrough{\lstinline!Null!}):

\begin{lstlisting}[language=Caml]
# Yojson.Basic.pretty_to_string
- : ?std:bool -> Yojson.Basic.t -> string = <fun>
\end{lstlisting}

The \passthrough{\lstinline!pretty\_to\_string!} function has a more
explicit signature that requires an argument of type
\passthrough{\lstinline!Yojson.Basic.t!}. When
\passthrough{\lstinline!person!} is applied to
\passthrough{\lstinline!pretty\_to\_string!}, the inferred type of
\passthrough{\lstinline!person!} is statically checked against the
structure of the \passthrough{\lstinline!json!} type to ensure that
they're compatible:

\begin{lstlisting}[language=Caml]
# Yojson.Basic.pretty_to_string person
- : string = "{ \"name\": \"Anil\" }"
# Yojson.Basic.pretty_to_channel stdout person
{ "name": "Anil" }
- : unit = ()
\end{lstlisting}

In this case, there are no problems. Our
\passthrough{\lstinline!person!} value has an inferred type that is a
valid subtype of \passthrough{\lstinline!json!}, and so the conversion
to a string just works without us ever having to explicitly specify a
type for \passthrough{\lstinline!person!}. Type inference lets you write
more succinct code without sacrificing runtime reliability, as all the
uses of polymorphic variants are still checked at compile time.
\index{errors/type errors}\index{type checking}\index{polymorphic
variant types/type checking and}\index{type inference/benefits of}

\hypertarget{polymorphic-variants-and-easier-type-checking}{%
\subparagraph{Polymorphic Variants and Easier Type
Checking}\label{polymorphic-variants-and-easier-type-checking}}

One difficulty you will encounter is that type errors involving
polymorphic variants can be quite verbose. For example, suppose you
build an \passthrough{\lstinline!Assoc!} and mistakenly include a single
value instead of a list of keys:

\begin{lstlisting}[language=Caml]
# let person = `Assoc ("name", `String "Anil")
val person : [> `Assoc of string * [> `String of string ] ] =
  `Assoc ("name", `String "Anil")
# Yojson.Basic.pretty_to_string person
Line 1, characters 31-37:
Error: This expression has type
         [> `Assoc of string * [> `String of string ] ]
       but an expression was expected of type Yojson.Basic.t
       Types for tag `Assoc are incompatible
\end{lstlisting}

The type error is more verbose than it needs to be, which can be
inconvenient to wade through for larger values. You can help the
compiler to narrow down this error to a shorter form by adding explicit
type annotations as a hint about your intentions:

\begin{lstlisting}[language=Caml]
# let (person : Yojson.Basic.t) =
  `Assoc ("name", `String "Anil")
Line 2, characters 10-34:
Error: This expression has type 'a * 'b
       but an expression was expected of type (string * Yojson.Basic.t) list
\end{lstlisting}

We've annotated \passthrough{\lstinline!person!} as being of type
\passthrough{\lstinline!Yojson.Basic.t!}, and as a result, the compiler
spots that the argument to the \passthrough{\lstinline!Assoc!} variant
has the incorrect type. This illustrates the strengths and weaknesses of
polymorphic variants: they're lightweight and flexible, but the error
messages can be quite confusing. However, a bit of careful manual type
annotation makes tracking down such issues much easier.

We'll discuss more techniques like this that help you interpret type
errors more easily in
\href{compiler-frontend.html\#the-compiler-frontend-parsing-and-type-checking}{The
Compiler Frontend Parsing And Type Checking}.

\hypertarget{using-non-standard-json-extensions}{%
\subsection{Using Nonstandard JSON
Extensions}\label{using-non-standard-json-extensions}}

The standard JSON types are \emph{really} basic, and OCaml types are far
more expressive. Yojson supports an extended JSON format for those times
when you're not interoperating with external systems and just want a
convenient human-readable, local format. The
\passthrough{\lstinline!Yojson.Safe.json!} type is a superset of the
\passthrough{\lstinline!Basic!} polymorphic variant and looks like this:
\index{Yojson library/extended
JSON format support}\index{JSON data/nonstandard extensions for}

\begin{lstlisting}[language=Caml]
type json = [
  | `Assoc of (string * json) list
  | `Bool of bool
  | `Float of float
  | `Floatlit of string
  | `Int of int
  | `Intlit of string
  | `List of json list
  | `Null
  | `String of string
  | `Stringlit of string
  | `Tuple of json list
  | `Variant of string * json option
]
\end{lstlisting}

The \passthrough{\lstinline!Safe.json!} type includes all of the
variants from \passthrough{\lstinline!Basic.json!} and extends it with a
few more useful ones. A standard JSON type such as a
\passthrough{\lstinline!String!} will type-check against both the
\passthrough{\lstinline!Basic!} module and also the nonstandard
\passthrough{\lstinline!Safe!} module. If you use the extended values
with the \passthrough{\lstinline!Basic!} module, however, the compiler
will reject your code until you make it compliant with the portable
subset of JSON.

Yojson supports the following JSON extensions:
\index{variant types/Yojson support
for}\index{tuples}\index{lit suffix}

\begin{description}
\tightlist
\item[The \texttt{lit} suffix]
Denotes that the value is stored as a JSON string. For example, a
\passthrough{\lstinline!Floatlit!} will be stored as
\passthrough{\lstinline!"1.234"!} instead of
\passthrough{\lstinline!1.234!}.
\item[The \texttt{Tuple} type]
Stored as \passthrough{\lstinline!("abc", 123)!} instead of a list.
\item[The \texttt{Variant} type]
Encodes OCaml variants more explicitly, as
\passthrough{\lstinline!<"Foo">!} or
\passthrough{\lstinline!<"Bar":123>!} for a variant with parameters.
\end{description}

The only purpose of these extensions is to have greater control over how
OCaml values are represented in JSON (for instance, storing a
floating-point number as a JSON string). The output still obeys the same
standard format that can be easily exchanged with other languages.

You can convert a \passthrough{\lstinline!Safe.json!} to a
\passthrough{\lstinline!Basic.json!} type by using the
\passthrough{\lstinline!to\_basic!} function as follows:

\begin{lstlisting}[language=Caml]
val to_basic : json -> Yojson.Basic.t
(** Tuples are converted to JSON arrays, Variants are converted to
    JSON strings or arrays of a string (constructor) and a json value
    (argument). Long integers are converted to JSON strings.
    Examples:

    `Tuple [ `Int 1; `Float 2.3 ]   ->    `List [ `Int 1; `Float 2.3 ]
    `Variant ("A", None)            ->    `String "A"
    `Variant ("B", Some x)          ->    `List [ `String "B", x ]
    `Intlit "12345678901234567890"  ->    `String "12345678901234567890"
 *)
\end{lstlisting}

\hypertarget{automatically-mapping-json-to-ocaml-types}{%
\subsection{Automatically Mapping JSON to OCaml
Types}\label{automatically-mapping-json-to-ocaml-types}}

The combinators described previously make it easy to write functions
that extract fields from JSON records, but the process is still pretty
manual. When you implement larger specifications, it's much easier to
generate the mappings from JSON schemas to OCaml values more
mechanically than writing conversion functions individually.
\protect\hypertarget{MAPjson}{}{mapping/of JSON to OCaml
types}\protect\hypertarget{JSONautomap}{}{JSON data/automatic mapping
of}

We'll cover an alternative JSON processing method that is better for
larger-scale JSON handling now, using the
\href{http://mjambon.com/atd-biniou-intro.html}{OCaml} tool. This will
introduce our first \emph{Domain Specific Language} that compiles JSON
specifications into OCaml modules, which are then used throughout your
application. \index{ATDgen
Library/installation of}\index{Domain Specific Language}

\hypertarget{installing-the-atdgen-library-and-tool}{%
\subsubsection{Installing the ATDgen Library and
Tool}\label{installing-the-atdgen-library-and-tool}}

ATDgen installs some OCaml libraries that interface with Yojson, and
also a command-line tool that generates code. It can all be installed
via OPAM:

\begin{lstlisting}
$ opam install atdgen
$ atdgen -version
2.0.0
\end{lstlisting}

The command-line tool will be installed within your
\textasciitilde/.opam directory and should already be on your
\passthrough{\lstinline!PATH!} from running
\passthrough{\lstinline!opam config env!}. See \href{install.html}{the
installation instructions} if this isn't working.

\hypertarget{atd-basics}{%
\subsubsection{ATD Basics}\label{atd-basics}}

The idea behind ATD is to specify the format of the JSON in a separate
file and then run a compiler (\passthrough{\lstinline!atdgen!}) that
outputs OCaml code to construct and parse JSON values. This means that
you don't need to write any OCaml parsing code at all, as it will all be
autogenerated for you. \index{ATDgen Library/basics
of}

Let's go straight into looking at an example of how this works, by using
a small portion of the GitHub API. GitHub is a popular code hosting and
sharing website that provides a JSON-based web
\href{http://developer.github.com}{API}. The following ATD code fragment
describes the GitHub authorization API (which is based on a
pseudostandard web protocol known as OAuth): \index{GitHub
API}\index{OAuth web protocol}

\begin{lstlisting}
type scope = [
    User <json name="user">
  | Public_repo <json name="public_repo">
  | Repo <json name="repo">
  | Repo_status <json name="repo_status">
  | Delete_repo <json name="delete_repo">
  | Gist <json name="gist">
]

type app = {
  name: string;
  url: string;
}  <ocaml field_prefix="app_">

type authorization_request = {
  scopes: scope list;
  note: string;
} <ocaml field_prefix="auth_req_">

type authorization_response = {
  scopes: scope list;
  token: string;
  app: app;
  url: string;
  id: int;
  ?note: string option;
  ?note_url: string option;
}
\end{lstlisting}

The ATD specification syntax is deliberately quite similar to OCaml type
definitions. Every JSON record is assigned a type name (e.g.,
\passthrough{\lstinline!app!} in the preceding example). You can also
define variants that are similar to OCaml's variant types (e.g.,
\passthrough{\lstinline!scope!} in the example).

\hypertarget{atd-annotations}{%
\subsubsection{ATD Annotations}\label{atd-annotations}}

ATD does deviate from OCaml syntax due to its support for annotations
within the specification. The annotations can customize the code that is
generated for a particular target (of which the OCaml backend is of most
interest to us). \index{ATDgen Library/annotations in}

For example, the preceding GitHub \passthrough{\lstinline!scope!} field
is defined as a variant type, with each option starting with an
uppercase letter as is conventional for OCaml variants. However, the
JSON values that come back from GitHub are actually lowercase and so
aren't exactly the same as the option name.

The annotation \passthrough{\lstinline!<json name="user">!} signals that
the JSON value of the field is \passthrough{\lstinline!user!}, but that
the variable name of the parsed variant in OCaml should be
\passthrough{\lstinline!User!}. These annotations are often useful to
map JSON values to reserved keywords in OCaml (e.g.,
\passthrough{\lstinline!type!}).

\hypertarget{compiling-atd-specifications-to-ocaml}{%
\subsubsection{Compiling ATD Specifications to
OCaml}\label{compiling-atd-specifications-to-ocaml}}

The ATD specification we defined can be compiled to OCaml code using the
\passthrough{\lstinline!atdgen!} command-line tool. Let's run the
compiler twice to generate some OCaml type definitions and a JSON
serializing module that converts between input data and those type
definitions. \index{ATDgen Library/compiling
specifications to OCaml}

The \passthrough{\lstinline!atdgen!} command will generate some new
files in your current directory. \passthrough{\lstinline!github\_t.ml!}
and \passthrough{\lstinline!github\_t.mli!} will contain an OCaml module
with types defined that correspond to the ATD file:

\begin{lstlisting}[language=bash]
$ atdgen -t github.atd
$ atdgen -j github.atd
$ ocamlfind ocamlc -package atd -i github_t.mli
type scope =
    [ `Delete_repo | `Gist | `Public_repo | `Repo | `Repo_status | `User ]
type app = { app_name : string; app_url : string; }
type authorization_response = {
  scopes : scope list;
  token : string;
  app : app;
  url : string;
  id : int;
  note : string option;
  note_url : string option;
}
type authorization_request = {
  auth_req_scopes : scope list;
  auth_req_note : string;
}
\end{lstlisting}

There is an obvious correspondence to the ATD definition. Note that
field names in OCaml records in the same module cannot shadow one
another, and so we instruct ATDgen to prefix every field with a name
that distinguishes it from other records in the same module. For
example, \passthrough{\lstinline!<ocaml field\_prefix="auth\_req\_">!}
in the ATD spec prefixes every field name in the generated
\passthrough{\lstinline!authorization\_request!} record with
\passthrough{\lstinline!auth\_req!}.

The \passthrough{\lstinline!Github\_t!} module only contains the type
definitions, while \passthrough{\lstinline!Github\_j!} provides
serialization functions to and from JSON. You can read the
\passthrough{\lstinline!github\_j.mli!} to see the full interface, but
the important functions for most uses are the conversion functions to
and from a string. For our preceding example, this looks like:

\begin{lstlisting}[language=Caml]
val string_of_authorization_request :
  ?len:int -> authorization_request -> string
  (** Serialize a value of type {!authorization_request}
      into a JSON string.
      @param len specifies the initial length
                 of the buffer used internally.
                 Default: 1024. *)

val string_of_authorization_response :
  ?len:int -> authorization_response -> string
  (** Serialize a value of type {!authorization_response}
      into a JSON string.
      @param len specifies the initial length
                 of the buffer used internally.
                 Default: 1024. *)
\end{lstlisting}

This is pretty convenient! We've now written a single ATD file, and all
the OCaml boilerplate to convert between JSON and a strongly typed
record has been generated for us. You can control various aspects of the
serializer by passing flags to \passthrough{\lstinline!atdgen!}. The
important ones for JSON are: {-j-defaults}{-j-custom-fields
FUNCTION}{-j-std flag}\index{flags}

\begin{description}
\tightlist
\item[\texttt{-j-std}]
Converts tuples and variants into standard JSON and refuse to print NaN
and infinities. You should specify this if you intend to interoperate
with services that aren't using ATD.
\item[\texttt{-j-custom-fields\ FUNCTION}]
Calls a custom function for every unknown field encountered, instead of
raising a parsing exception.
\item[\texttt{-j-defaults}]
Always explicitly outputs a JSON value if possible. This requires the
default value for that field to be defined in the ATD specification.
\end{description}

The full \href{https://atd.readthedocs.io/en/latest/}{ATD specification}
is quite sophisticated and documented online. The ATD compiler can also
target formats other than JSON and outputs code for other languages
(such as Java) if you need more interoperability.

There are also several similar projects that automate the code
generation process. \href{http://piqi.org}{Piqi} supports conversions
between XML, JSON, and the Google protobuf format; and
\href{http://thrift.apache.org}{Thrift} supports many other programming
languages and includes OCaml bindings.

\hypertarget{example-querying-github-organization-information}{%
\subsubsection{Example: Querying GitHub Organization
Information}\label{example-querying-github-organization-information}}

Let's finish up with an example of some live JSON parsing from GitHub
and build a tool to query organization information via their API. Start
by looking at the online \href{http://developer.github.com/v3/orgs/}{API
documentation} for GitHub to see what the JSON schema for retrieving the
organization information looks like.
\index{GitHub API}\index{ATDgen Library/example of}

Now create an ATD file that covers the fields we need. Any extra fields
present in the response will be ignored by the ATD parser, so we don't
need a completely exhaustive specification of every field that GitHub
might send back:

\begin{lstlisting}
type org = {
  login: string;
  id: int;
  url: string;
  ?name: string option;
  ?blog: string option;
  ?email: string option;
  public_repos: int
}
\end{lstlisting}

Let's build the OCaml type declaration first by calling
\passthrough{\lstinline!atdgen -t!} on the specification file:

\begin{lstlisting}[language=bash]
$ dune build github_org_t.mli
$ cat _build/default/github_org_t.mli
(* Auto-generated from "github_org.atd" *)
              [@@@ocaml.warning "-27-32-35-39"]

type org = {
  login: string;
  id: int;
  url: string;
  name: string option;
  blog: string option;
  email: string option;
  public_repos: int
}
\end{lstlisting}

The OCaml type has an obvious mapping to the ATD spec, but we still need
the logic to convert JSON buffers to and from this type. Calling
\passthrough{\lstinline!atdgen -j!} will generate this serialization
code for us in a new file called
\passthrough{\lstinline!github\_org\_j.ml!}:

\begin{lstlisting}[language=bash]
$ dune build github_org_j.mli
$ cat _build/default/github_org_j.mli
(* Auto-generated from "github_org.atd" *)
[@@@ocaml.warning "-27-32-35-39"]

type org = Github_org_t.org = {
  login: string;
  id: int;
  url: string;
  name: string option;
  blog: string option;
  email: string option;
  public_repos: int
}

val write_org :
  Bi_outbuf.t -> org -> unit
  (** Output a JSON value of type {!org}. *)

val string_of_org :
  ?len:int -> org -> string
  (** Serialize a value of type {!org}
      into a JSON string.
      @param len specifies the initial length
                 of the buffer used internally.
                 Default: 1024. *)

val read_org :
  Yojson.Safe.lexer_state -> Lexing.lexbuf -> org
  (** Input JSON data of type {!org}. *)

val org_of_string :
  string -> org
  (** Deserialize JSON data of type {!org}. *)
\end{lstlisting}

The \passthrough{\lstinline!Github\_org\_j!} serializer interface
contains everything we need to map to and from the OCaml types and JSON.
The easiest way to use this interface is by using the
\passthrough{\lstinline!string\_of\_org!} and
\passthrough{\lstinline!org\_of\_string!} functions, but there are also
more advanced low-level buffer functions available if you need higher
performance (but we won't go into that in this tutorial).

All we need to complete our example is an OCaml program that fetches the
JSON and uses these modules to output a one-line summary. Our following
example does just that.

The following code calls the cURL command-line utility by using the
\passthrough{\lstinline!Shell!} interface to run an external command and
capture its output. You'll need to ensure that you have cURL installed
on your system before running the example. You might also need to
\passthrough{\lstinline!opam install shell!} if you haven't installed it
previously:

\begin{lstlisting}[language=Caml]
open Core

let print_org file () =
  let url = sprintf "https://api.github.com/orgs/%s" file in
  Shell.run_full "curl" [url]
  |> Github_org_j.org_of_string
  |> fun org ->
  let open Github_org_t in
  let name = Option.value ~default:"???" org.name in
  printf "%s (%d) with %d public repos\n"
    name org.id org.public_repos

let () =
  Command.basic_spec ~summary:"Print Github organization information"
    Command.Spec.(empty +> anon ("organization" %: string))
    print_org
  |> Command.run
\end{lstlisting}

The following is a short shell script that generates all of the OCaml
code and also builds the final executable:

\begin{lstlisting}
(rule
  (targets github_org_j.ml github_org_j.mli)
  (deps    github_org.atd)
  (mode    fallback)
  (action  (run atdgen -j %{deps})))

(rule
  (targets github_org_t.ml github_org_t.mli)
  (deps    github_org.atd)
  (mode    fallback)
  (action  (run atdgen -t %{deps})))

(executable
  (name      github_org_info)
  (libraries core yojson atdgen shell)
  (flags     :standard -w -32)
  (modules   github_org_info github_org_t github_org_j))
\end{lstlisting}

\begin{lstlisting}[language=bash]
$ dune build github_org_info.exe
\end{lstlisting}

You can now run the command-line tool with a single argument to specify
the name of the organization, and it will dynamically fetch the JSON
from the web, parse it, and render the summary to your console:

\begin{lstlisting}[language=bash]
$ dune exec -- ./github_org_info.exe mirage
MirageOS (131943) with 125 public repos
$ dune exec -- ./github_org_info.exe janestreet
??? (3384712) with 145 public repos
\end{lstlisting}

The JSON returned from the \passthrough{\lstinline!janestreet!} query is
missing an organization name, but this is explicitly reflected in the
OCaml type, since the ATD spec marked \passthrough{\lstinline!name!} as
an optional field. Our OCaml code explicitly handles this case and
doesn't have to worry about null-pointer exceptions. Similarly, the JSON
integer for the \passthrough{\lstinline!id!} is mapped into a native
OCaml integer via the ATD conversion.

While this tool is obviously quite simple, the ability to specify
optional and default fields is very powerful. Take a look at the full
ATD specification for the GitHub API in the
\href{http://github.com/avsm/ocaml-github}{\passthrough{\lstinline!ocaml-github!}}
repository online, which has lots of quirks typical in real-world web
APIs. ~~~~

Our example shells out to \passthrough{\lstinline!curl!} on the command
line to obtain the JSON, which is rather inefficient. We'll explain how
to integrate the HTTP fetch directly into your OCaml application in
\href{concurrent-programming.html\#concurrent-programming-with-async}{Concurrent
Programming With Async}.
