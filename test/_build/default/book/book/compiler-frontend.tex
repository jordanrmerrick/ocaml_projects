\hypertarget{the-compiler-frontend-parsing-and-type-checking}{%
\section{\texorpdfstring{The Compiler Frontend: Parsing and {Type
Checking}}{The Compiler Frontend: Parsing and Type Checking}}\label{the-compiler-frontend-parsing-and-type-checking}}

Compiling source code into executable programs involves a fairly complex
set of libraries, linkers, and assemblers. It's important to understand
how these fit together to help with your day-to-day workflow of
developing, debugging, and deploying
applications.\index{compilation process/toolchain for}

OCaml has a strong emphasis on static type safety and rejects source
code that doesn't meet its requirements as early as possible. The
compiler does this by running the source code through a series of checks
and transformations. Each stage performs its job (e.g., type checking,
optimization, or code generation) and discards some information from the
previous stage. The final native code output is low-level assembly code
that doesn't know anything about the OCaml modules or objects that the
compiler started
with.\index{static checking}\index{compile-time static checking}

You don't have to do all of this manually, of course. The compiler
frontends (\passthrough{\lstinline!ocamlc!} and
\passthrough{\lstinline!ocamlopt!}) are invoked via the command line and
chain the stages together for you. Sometimes though, you'll need to dive
into the toolchain to hunt down a bug or investigate a performance
problem. This chapter explains the compiler pipeline in more depth so
you understand how to harness the command-line tools effectively.
\index{OCaml
toolchain/ocamlc}\index{OCaml toolchain/ocamlopt}

In this chapter, we'll cover the following topics:

\begin{itemize}
\item
  The compilation pipeline and what each stage represents
\item
  The type-checking process, including module resolution
\end{itemize}

The details of the compilation process into executable code can be found
next, in
\href{compiler-backend.html\#the-compiler-backend-byte-code-and-native-code}{The
Compiler Backend Byte Code And Native Code}.

\hypertarget{an-overview-of-the-toolchain}{%
\subsection{An Overview of the
Toolchain}\label{an-overview-of-the-toolchain}}

The OCaml tools accept textual source code as input, using the filename
extensions \passthrough{\lstinline!.ml!} and
\passthrough{\lstinline!.mli!} for modules and signatures, respectively.
We explained the basics of the build process in
\href{files-modules-and-programs.html\#files-modules-and-programs}{Files
Modules And Programs}, so we'll assume you've built a few OCaml programs
already by this point.\index{OCaml toolchain/overview of}

Each source file represents a \emph{compilation unit} that is built
separately. The compiler generates intermediate files with different
filename extensions to use as it advances through the compilation
stages. The linker takes a collection of compiled units and produces a
standalone executable or library archive that can be reused by other
applications.\index{compilation units}

The overall compilation pipeline looks like this: \index{compilation
process/diagram of}

Notice that the pipeline branches toward the end. OCaml has multiple
compiler backends that reuse the early stages of compilation but produce
very different final outputs. The \emph{bytecode} can be run by a
portable interpreter and can even be transformed into JavaScript (via
\href{http://ocsigen.org/js_of_ocaml}{js\_of\_ocaml}) or C source code
(via \href{https://github.com/ocaml-bytes/ocamlcc}{OCamlCC}). The
\emph{native code} compiler generates specialized executable binaries
suitable for high-performance
applications.\index{compilation process/compiler source
code}\index{code compilers/bytecode vs. native code}

\hypertarget{obtaining-the-compiler-source-code}{%
\subparagraph{Obtaining the Compiler Source
Code}\label{obtaining-the-compiler-source-code}}

Although it's not necessary to understand the examples, you may find it
useful to have a copy of the OCaml source tree checked out while you
read through this chapter. The source code is available from multiple
places:

\begin{itemize}
\item
  Stable releases as zip and tar archives from the
  \href{http://ocaml.org/docs/install.html}{OCaml download site}
\item
  A Git repository with all the history and development branches
  included, browsable online at
  \href{https://github.com/ocaml/ocaml}{GitHub}
\end{itemize}

The source tree is split up into subdirectories. The core compiler
consists of:

\begin{description}
\tightlist
\item[\texttt{config/}]
Configuration directives to tailor OCaml for your operating system and
architecture.
\item[\texttt{bytecomp/}]
Bytecode compiler that converts OCaml into an interpreted executable
format.
\item[\texttt{asmcomp/}]
Native-code compiler that converts OCaml into high performance native
code executables.
\item[\texttt{parsing/}]
The OCaml lexer, parser, and libraries for manipulating them.
\item[\texttt{typing/}]
The static type checking implementation and type definitions.
\item[\texttt{driver/}]
Command-line interfaces for the compiler tools.
\end{description}

A number of tools and scripts are also built alongside the core
compiler:

\begin{description}
\tightlist
\item[\texttt{debugger/}]
The interactive bytecode debugger.
\item[\texttt{toplevel/}]
Interactive top-level console.
\item[\texttt{emacs/}]
A \emph{caml-mode} for the Emacs editor.
\item[\texttt{stdlib/}]
The compiler standard library, including the
\passthrough{\lstinline!Pervasives!} module.
\item[\texttt{ocamlbuild/}]
Build system that automates common OCaml compilation modes.
\item[\texttt{otherlibs/}]
Optional libraries such as the Unix and graphics modules.
\item[\texttt{tools/}]
Command-line utilities such as \passthrough{\lstinline!ocamldep!} that
are installed with the compiler.
\item[\texttt{testsuite/}]
Regression tests for the core compiler.
\end{description}

We'll go through each of the compilation stages now and explain how they
will be useful to you during day-to-day OCaml development.

\hypertarget{parsing-source-code}{%
\subsection{Parsing Source Code}\label{parsing-source-code}}

When a source file is passed to the OCaml compiler, its first task is to
parse the text into a more structured abstract syntax tree (AST). The
parsing logic is implemented in OCaml itself using the techniques
described earlier in
\href{parsing-with-ocamllex-and-menhir.html\#parsing-with-ocamllex-and-menhir}{Parsing
With Ocamllex And Menhir}. The lexer and parser rules can be found in
the \passthrough{\lstinline!parsing!} directory in the source
distribution.\index{AST (abstract syntax-tree)}\protect\hypertarget{SCpras}{}{source
code/parsing of}\protect\hypertarget{PARSsource}{}{parsing/of source
code}\protect\hypertarget{CPpars}{}{compilation process/parsing source
code}

\hypertarget{syntax-errors}{%
\subsubsection{Syntax Errors}\label{syntax-errors}}

The OCaml parser's goal is to output a well-formed AST data structure to
the next phase of compilation, and so it any source code that doesn't
match basic syntactic requirements. The compiler emits a \emph{syntax
error} in this situation, with a pointer to the filename and line and
character number that's as close to the error as
possible.\index{errors/syntax errors}\index{syntax
errors}

Here's an example syntax error that we obtain by performing a module
assignment as a statement instead of as a \passthrough{\lstinline!let!}
binding:

\begin{lstlisting}[language=Caml]
let () =
  module MyString = String;
  ()
\end{lstlisting}

The code results in a syntax error when compiled:

\begin{lstlisting}[language=bash]
$ ocamlc -c broken_module.ml
File "broken_module.ml", line 2, characters 2-8:
2 |   module MyString = String;
      ^^^^^^
Error: Syntax error
[2]
\end{lstlisting}

The correct version of this source code creates the
\passthrough{\lstinline!MyString!} module correctly via a local open,
and compiles successfully:

\begin{lstlisting}[language=Caml]
let () =
  let module MyString = String in
  ()
\end{lstlisting}

The syntax error points to the line and character number of the first
token that couldn't be parsed. In the broken example, the
\passthrough{\lstinline!module!} keyword isn't a valid token at that
point in parsing, so the error location information is correct.

\hypertarget{automatically-indenting-source-code}{%
\subsubsection{Automatically Indenting Source
Code}\label{automatically-indenting-source-code}}

Sadly, syntax errors do get more inaccurate sometimes, depending on the
nature of your mistake. Try to spot the deliberate error in the
following function definitions:
\index{source code/automatically indenting}

\begin{lstlisting}[language=Caml]
let concat_and_print x y =
  let v = x ^ y in
  print_endline v;
  v;

let add_and_print x y =
  let v = x + y in
  print_endline (string_of_int v);
  v

let () =
  let _x = add_and_print 1 2 in
  let _y = concat_and_print "a" "b" in
  ()
\end{lstlisting}

When you compile this file, you'll get a syntax error again:

\begin{lstlisting}[language=bash]
$ ocamlc -c follow_on_function.ml
File "follow_on_function.ml", line 11, characters 0-3:
11 | let () =
     ^^^
Error: Syntax error
[2]
\end{lstlisting}

The line number in the error points to the end of the
\passthrough{\lstinline!add\_and\_print!} function, but the actual error
is at the end of the \emph{first} function definition. There's an extra
semicolon at the end of the first definition that causes the second
definition to become part of the first \passthrough{\lstinline!let!}
binding. This eventually results in a parsing error at the very end of
the second function.

This class of bug (due to a single errant character) can be hard to spot
in a large body of code. Luckily, there's a great tool available via
OPAM called \passthrough{\lstinline!ocp-indent!} that applies structured
indenting rules to your source code on a line-by-line basis. This not
only beautifies your code layout, but it also makes this syntax error
much easier to locate.\index{debugging/single errant
characters}

Let's run our erroneous file through
\passthrough{\lstinline!ocp-indent!} and see how it processes it:

\begin{lstlisting}[language=bash]
$ ocp-indent follow_on_function.ml
let concat_and_print x y =
  let v = x ^ y in
  print_endline v;
  v;

  let add_and_print x y =
    let v = x + y in
    print_endline (string_of_int v);
    v

let () =
  let _x = add_and_print 1 2 in
  let _y = concat_and_print "a" "b" in
  ()
\end{lstlisting}

The \passthrough{\lstinline!add\_and\_print!} definition has been
indented as if it were part of the first
\passthrough{\lstinline!concat\_and\_print!} definition, and the errant
semicolon is now much easier to spot. We just need to remove that
semicolon and rerun \passthrough{\lstinline!ocp-indent!} to verify that
the syntax is correct:

\begin{lstlisting}[language=bash]
$ ocp-indent follow_on_function_fixed.ml
(*TODO: Check contents*)
let concat_and_print x y =
  let v = x ^ y in
  print_endline v;
  v

let add_and_print x y =
  let v = x + y in
  print_endline (string_of_int v);
  v

let () =
  let _x = add_and_print 1 2 in
  let _y = concat_and_print "a" "b" in
  ()
\end{lstlisting}

The \passthrough{\lstinline!ocp-indent!}
\href{https://github.com/OCamlPro/ocp-indent}{homepage} documents how to
integrate it with your favorite editor. All the Core libraries are
formatted using it to ensure consistency, and it's a good idea to do
this before publishing your own source code online.

\hypertarget{generating-documentation-from-interfaces}{%
\subsubsection{Generating Documentation from
Interfaces}\label{generating-documentation-from-interfaces}}

Whitespace and source code comments are removed during parsing and
aren't significant in determining the semantics of the program. However,
other tools in the OCaml distribution can interpret comments for their
own ends. \index{OCaml
toolchain/ocamldoc}\index{interfaces/generating documentation
from}\index{documentation, generating from interfaces}

The \passthrough{\lstinline!ocamldoc!} tool uses specially formatted
comments in the source code to generate documentation bundles. These
comments are combined with the function definitions and signatures, and
output as structured documentation in a variety of formats. It can
generate HTML pages, LaTeX and PDF documents, UNIX manual pages, and
even module dependency graphs that can be viewed using
\href{http://www.graphviz.org}{Graphviz}.

Here's a sample of some source code that's been annotated with
\passthrough{\lstinline!ocamldoc!} comments:

\begin{lstlisting}[language=Caml]
(** example.ml: The first special comment of the file is the comment
    associated with the whole module. *)

(** Comment for exception My_exception. *)
exception My_exception of (int -> int) * int

(** Comment for type [weather]  *)
type weather =
  | Rain of int (** The comment for construtor Rain *)
  | Sun         (** The comment for constructor Sun *)

(** Find the current weather for a country
    @author Anil Madhavapeddy
    @param location The country to get the weather for.
*)
let what_is_the_weather_in location =
  match location with
  | `Cambridge  -> Rain 100
  | `New_york   -> Rain 20
  | `California -> Sun
\end{lstlisting}

The \passthrough{\lstinline!ocamldoc!} comments are distinguished by
beginning with the double asterisk. There are formatting conventions for
the contents of the comment to mark metadata. For instance, the
\passthrough{\lstinline!@tag!} fields mark specific properties such as
the author of that section of code.

Try compiling the HTML documentation and UNIX man pages by running
\passthrough{\lstinline!ocamldoc!} over the source file:

\begin{lstlisting}
$ mkdir -p html man/man3
$ ocamldoc -html -d html doc.ml
$ ocamldoc -man -d man/man3 doc.ml
$ man -M man Doc
\end{lstlisting}

You should now have HTML files inside the html/ directory and also be
able to view the UNIX manual pages held in man/man3. There are quite a
few comment formats and options to control the output for the various
backends. Refer to the
\href{http://caml.inria.fr/pub/docs/manual-ocaml/manual029.html}{OCaml
manual} for the complete
list.\index{Xen}\index{JSON data/Xen custom generator
for}\index{Bibtex}\index{OCaml toolchain/ocamldoc-generators}\index{Argot
HTML generator}\index{HTML
generators}~~~

\hypertarget{using-custom-ocamldoc-generators}{%
\paragraph{Using Custom ocamldoc
Generators}\label{using-custom-ocamldoc-generators}}

The default HTML output stylesheets from
\passthrough{\lstinline!ocamldoc!} are pretty spartan and distinctly Web
1.0. The tool supports plugging in custom documentation generators, and
there are several available that provide prettier or more detailed
output:

\begin{itemize}
\item
  \href{http://argot.x9c.fr/}{Argot} is an enhanced HTML generator that
  supports code folding and searching by name or type definition.
\item
  \href{https://gitorious.org/ocamldoc-generators/ocamldoc-generators}{ocamldoc
  generators} add support for Bibtex references within comments and
  generating literate documentation that embeds the code alongside the
  comments.
\item
  JSON output is available via a custom
  \href{https://github.com/xen-org/ocamldoc-json}{generator} in Xen.
\end{itemize}

\hypertarget{static-type-checking}{%
\subsection{Static Type Checking}\label{static-type-checking}}

After obtaining a valid abstract syntax tree, the compiler has to verify
that the code obeys the rules of the OCaml type system. Code that is
syntactically correct but misuses values is rejected with an explanation
of the problem.

Although type checking is done in a single pass in OCaml, it actually
consists of three distinct steps that happen
simultaneously:\index{explicit
subtyping}\index{automatic type inference}\index{subtyping/in static type
checking}\index{modules/in static type checking}\index{type inference/in
static type checking}\protect\hypertarget{CPstatictype}{}{compilation
process/static type checking}

\begin{description}
\tightlist
\item[automatic type inference]
An algorithm that calculates types for a module without requiring manual
type annotations
\item[module system]
Combines software components with explicit knowledge of their type
signatures
\item[explicit subtyping]
Checks for objects and polymorphic variants
\end{description}

Automatic type inference lets you write succinct code for a particular
task and have the compiler ensure that your use of variables is locally
consistent.

Type inference doesn't scale to very large codebases that depend on
separate compilation of files. A small change in one module may ripple
through thousands of other files and libraries and require all of them
to be recompiled. The module system solves this by providing the
facility to combine and manipulate explicit type signatures for modules
within a large project, and also to reuse them via functors and
first-class
modules.\index{modules/benefits of}\index{type inference/drawbacks of}

Subtyping in OCaml objects is always an explicit operation (via the
\passthrough{\lstinline!:>!} operator). This means that it doesn't
complicate the core type inference engine and can be tested as a
separate concern.

\hypertarget{displaying-inferred-types-from-the-compiler}{%
\subsubsection{Displaying Inferred Types from the
Compiler}\label{displaying-inferred-types-from-the-compiler}}

We've already seen how you can explore type inference directly from the
toplevel. It's also possible to generate type signatures for an entire
file by asking the compiler to do the work for you. Create a file with a
single type definition and value:

\begin{lstlisting}[language=Caml]
type t = Foo | Bar
let v = Foo
\end{lstlisting}

Now run the compiler with the \passthrough{\lstinline!-i!} flag to infer
the type signature for that file. This runs the type checker but doesn't
compile the code any further after displaying the interface to the
standard output:

\begin{lstlisting}[language=bash]
\end{lstlisting}

The output is the default signature for the module that represents the
input file. It's often useful to redirect this output to an
\passthrough{\lstinline!mli!} file to give you a starting signature to
edit the external interface without having to type it all in by hand.

The compiler stores a compiled version of the interface as a
\passthrough{\lstinline!cmi!} file. This interface is either obtained
from compiling an \passthrough{\lstinline!mli!} signature file for a
module, or by the inferred type if there is only an
\passthrough{\lstinline!ml!} implementation present.

The compiler makes sure that your \passthrough{\lstinline!ml!} and
\passthrough{\lstinline!mli!} files have compatible signatures. The type
checker throws an immediate error if this isn't the case:

\begin{lstlisting}[language=Caml]
type t = Foo
\end{lstlisting}

\begin{lstlisting}[language=Caml]
type t = Bar
\end{lstlisting}

\begin{lstlisting}[language=bash]
$ ocamlc -c conflicting_interface.mli conflicting_interface.ml
File "conflicting_interface.ml", line 1:
Error: The implementation conflicting_interface.ml
       does not match the interface conflicting_interface.cmi:
       Type declarations do not match:
         type t = Foo
       is not included in
         type t = Bar
       File "conflicting_interface.mli", line 1, characters 0-12:
         Expected declaration
       File "conflicting_interface.ml", line 1, characters 0-12:
         Actual declaration
       Fields number 1 have different names, Foo and Bar.
[2]
\end{lstlisting}

\hypertarget{which-comes-first-the-ml-or-the-mli}{%
\paragraph{Which Comes First: The ml or the
mli?}\label{which-comes-first-the-ml-or-the-mli}}

There are two schools of thought on which order OCaml code should be
written in. It's very easy to begin writing code by starting with an
\passthrough{\lstinline!ml!} file and using the type inference to guide
you as you build up your functions. The \passthrough{\lstinline!mli!}
file can then be generated as described, and the exported functions
documented.\index{code compilers/order of code}\index{mli files}\index{files/mli
files}\index{ml files}\index{files/ml files}

If you're writing code that spans multiple files, it's sometimes easier
to start by writing all the \passthrough{\lstinline!mli!} signatures and
checking that they type-check against one another. Once the signatures
are in place, you can write the implementations with the confidence that
they'll all glue together correctly, with no cyclic dependencies among
the modules.

As with any such stylistic debate, you should experiment with which
system works best for you. Everyone agrees on one thing though: no
matter in what order you write them, production code should always
explicitly define an \passthrough{\lstinline!mli!} file for every
\passthrough{\lstinline!ml!} file in the project. It's also perfectly
fine to have an \passthrough{\lstinline!mli!} file without a
corresponding \passthrough{\lstinline!ml!} file if you're only declaring
signatures (such as module types).

Signature files provide a place to write succinct documentation and to
abstract internal details that shouldn't be exported. Maintaining
separate signature files also speeds up incremental compilation in
larger code bases, since recompiling a \passthrough{\lstinline!mli!}
signature is much faster than a full compilation of the implementation
to native code.

\hypertarget{type-inference-1}{%
\subsubsection{Type Inference}\label{type-inference-1}}

Type inference is the process of determining the appropriate types for
expressions based on their use. It's a feature that's partially present
in many other languages such as Haskell and Scala, but OCaml embeds it
as a fundamental feature throughout the core language.
\index{Hindley-Milner
algorithm}\index{type inference/algorithm basis of}

OCaml type inference is based on the Hindley-Milner algorithm, which is
notable for its ability to infer the most general type for an expression
without requiring any explicit type annotations. The algorithm can
deduce multiple types for an expression and has the notion of a
\emph{principal type} that is the most general choice from the possible
inferences. Manual type annotations can specialize the type explicitly,
but the automatic inference selects the most general type unless told
otherwise.

OCaml does have some language extensions that strain the limits of
principal type inference, but by and large, most programs you write will
never \emph{require} annotations (although they sometimes help the
compiler produce better error messages).

\hypertarget{adding-type-annotations-to-find-errors}{%
\paragraph{Adding type annotations to find
errors}\label{adding-type-annotations-to-find-errors}}

It's often said that the hardest part of writing OCaml code is getting
past the type checker---but once the code does compile, it works
correctly the first time! This is an exaggeration of course, but it can
certainly feel true when moving from a dynamically typed language. The
OCaml static type system protects you from certain classes of bugs such
as memory errors and abstraction violations by rejecting your program at
compilation time rather than by generating an error at runtime. Learning
how to navigate the type checker's compile-time feedback is key to
building robust libraries and applications that take full advantage of
these static checks.\index{type
inference/error detection with}\index{annotations, for type
checking}\index{errors/detecting with type annotations}\index{type
annotations}\index{compile-time static checking}

There are a couple of tricks to make it easier to quickly locate type
errors in your code. The first is to introduce manual type annotations
to narrow down the source of your error more accurately. These
annotations shouldn't actually change your types and can be removed once
your code is correct. However, they act as anchors to locate errors
while you're still writing your code.

Manual type annotations are particularly useful if you use lots of
polymorphic variants or objects. Type inference with row polymorphism
can generate some very large signatures, and errors tend to propagate
more widely than if you are using more explicitly typed variants or
classes.\index{polymorphic
variant types/type checking and}\index{row polymorphism}

For instance, consider this broken example that expresses some simple
algebraic operations over integers:

\begin{lstlisting}[language=Caml]
let rec algebra =
  function
  | `Add (x,y) -> (algebra x) + (algebra y)
  | `Sub (x,y) -> (algebra x) - (algebra y)
  | `Mul (x,y) -> (algebra x) * (algebra y)
  | `Num x     -> x

let _ =
  algebra (
    `Add (
      (`Num 0),
      (`Sub (
          (`Num 1),
          (`Mul (
              (`Nu 3),(`Num 2)
            ))
        ))
    ))
\end{lstlisting}

There's a single character typo in the code so that it uses
\passthrough{\lstinline!Nu!} instead of \passthrough{\lstinline!Num!}.
The resulting type error is impressive:

\begin{lstlisting}[language=bash]
$ ocamlc -c broken_poly.ml
File "broken_poly.ml", lines 9-18, characters 10-6:
 9 | ..........(
10 |     `Add (
11 |       (`Num 0),
12 |       (`Sub (
13 |           (`Num 1),
14 |           (`Mul (
15 |               (`Nu 3),(`Num 2)
16 |             ))
17 |         ))
18 |     ))
Error: This expression has type
         [> `Add of
              ([< `Add of 'a * 'a
                | `Mul of 'a * 'a
                | `Num of int
                | `Sub of 'a * 'a
                > `Num ]
               as 'a) *
              [> `Sub of 'a * [> `Mul of [> `Nu of int ] * [> `Num of int ] ]
              ] ]
       but an expression was expected of type
         [< `Add of 'a * 'a | `Mul of 'a * 'a | `Num of int | `Sub of 'a * 'a
          > `Num ]
         as 'a
       The second variant type does not allow tag(s) `Nu
[2]
\end{lstlisting}

The type error is perfectly accurate, but rather verbose and with a line
number that doesn't point to the exact location of the incorrect variant
name. The best the compiler can do is to point you in the general
direction of the \passthrough{\lstinline!algebra!} function application.

This is because the type checker doesn't have enough information to
match the inferred type of the \passthrough{\lstinline!algebra!}
definition to its application a few lines down. It calculates types for
both expressions separately, and when they don't match up, outputs the
difference as best it can.

Let's see what happens with an explicit type annotation to help the
compiler out:

\begin{lstlisting}[language=Caml]
type t = [
  | `Add of t * t
  | `Sub of t * t
  | `Mul of t * t
  | `Num of int
]

let rec algebra (x:t) =
  match x with
  | `Add (x,y) -> (algebra x) + (algebra y)
  | `Sub (x,y) -> (algebra x) - (algebra y)
  | `Mul (x,y) -> (algebra x) * (algebra y)
  | `Num x     -> x

let _ =
  algebra (
    `Add (
      (`Num 0),
      (`Sub (
          (`Num 1),
          (`Mul (
              (`Nu 3),(`Num 2)
            ))
        ))
    ))
\end{lstlisting}

This code contains exactly the same error as before, but we've added a
closed type definition of the polymorphic variants, and a type
annotation to the \passthrough{\lstinline!algebra!} definition. The
compiler error we get is much more useful now:

\begin{lstlisting}[language=bash]
$ ocamlc -i broken_poly_with_annot.ml
File "broken_poly_with_annot.ml", line 22, characters 14-21:
22 |               (`Nu 3),(`Num 2)
                   ^^^^^^^
Error: This expression has type [> `Nu of int ]
       but an expression was expected of type t
       The second variant type does not allow tag(s) `Nu
[2]
\end{lstlisting}

This error points directly to the correct line number that contains the
typo. Once you fix the problem, you can remove the manual annotations if
you prefer more succinct code. You can also leave the annotations there,
of course, to help with future refactoring and debugging.

\hypertarget{enforcing-principal-typing}{%
\paragraph{Enforcing principal
typing}\label{enforcing-principal-typing}}

The compiler also has a stricter \emph{principal type checking} mode
that is activated via the {-principal} flag. This warns about risky uses
of type information to ensure that the type inference has one principal
result. A type is considered risky if the success or failure of type
inference depends on the order in which subexpressions are
typed.\index{type inference/principality checks}\index{risky type}\index{principal
type checking}

The principality check only affects a few language features:

\begin{itemize}
\item
  Polymorphic methods for objects
\item
  Permuting the order of labeled arguments in a function from their type
  definition
\item
  Discarding optional labeled arguments
\item
  Generalized algebraic data types (GADTs) present from OCaml 4.0 onward
\item
  Automatic disambiguation of record field and constructor names (since
  OCaml 4.1)
\end{itemize}

Here's an example of principality warnings when used with record
disambiguation.

\begin{lstlisting}[language=Caml]
type s = { foo: int; bar: unit }
type t = { foo: int }

let f x =
  x.bar;
  x.foo
\end{lstlisting}

Inferring the signature with \passthrough{\lstinline!-principal!} will
show you a new warning:

\begin{lstlisting}[language=bash]
$ ocamlc -i -principal non_principal.ml
File "non_principal.ml", line 6, characters 4-7:
6 |   x.foo
        ^^^
Warning 18: this type-based field disambiguation is not principal.
type s = { foo : int; bar : unit; }
type t = { foo : int; }
val f : s -> int
\end{lstlisting}

This example isn't principal, since the inferred type for
\passthrough{\lstinline!x.foo!} is guided by the inferred type of
\passthrough{\lstinline!x.bar!}, whereas principal typing requires that
each subexpression's type can be calculated independently. If the
\passthrough{\lstinline!x.bar!} use is removed from the definition of
\passthrough{\lstinline!f!}, its argument would be of type
\passthrough{\lstinline!t!} and not \passthrough{\lstinline!type s!}.

You can fix this either by permuting the order of the type declarations,
or by adding an explicit type annotation:

\begin{lstlisting}[language=Caml]
type s = { foo: int; bar: unit }
type t = { foo: int }

let f (x:s) =
  x.bar;
  x.foo
\end{lstlisting}

There is now no ambiguity about the inferred types, since we've
explicitly given the argument a type, and the order of inference of the
subexpressions no longer matters.

\begin{lstlisting}[language=bash]
$ ocamlc -i -principal principal.ml
type s = { foo : int; bar : unit; }
type t = { foo : int; }
val f : s -> int
\end{lstlisting}

The \passthrough{\lstinline!dune!} equivalent is to add the flag
\passthrough{\lstinline!-principal!} to your build description.

\begin{lstlisting}[language=Caml]
(executable
  (name principal)
  (flags :standard -principal)
  (modules principal))

(executable
  (name non_principal)
  (flags :standard -principal)
  (modules non_principal))
\end{lstlisting}

The \passthrough{\lstinline!:standard!} directive will include all the
default flags, and then \passthrough{\lstinline!-principal!} will be
appended after those in the compiler build flags.

\begin{lstlisting}[language=bash]
$ dune build principal.exe
$ dune build non_principal.exe
File "non_principal.ml", line 6, characters 4-7:
6 |   x.foo
        ^^^
Error (warning 18): this type-based field disambiguation is not principal.
[1]
\end{lstlisting}

Ideally, all code should systematically use
\passthrough{\lstinline!-principal!}. It reduces variance in type
inference and enforces the notion of a single known type. However, there
are drawbacks to this mode: type inference is slower, and the
\passthrough{\lstinline!cmi!} files become larger. This is generally
only a problem if you extensively use objects, which usually have larger
type signatures to cover all their methods.

If compiling in principal mode works, it is guaranteed that the program
will pass type checking in non-principal mode, too. Bear in mind that
the \passthrough{\lstinline!cmi!} files generated in principal mode
differ from the default mode. Try to ensure that you compile your whole
project with it activated. Getting the files mixed up won't let you
violate type safety, but it can result in the type checker failing
unexpectedly very occasionally. In this case, just recompile with a
clean source tree.

\hypertarget{modules-and-separate-compilation}{%
\subsubsection{Modules and Separate
Compilation}\label{modules-and-separate-compilation}}

The OCaml module system enables smaller components to be reused
effectively in large projects while still retaining all the benefits of
static type safety. We covered the basics of using modules earlier in
\href{files-modules-and-programs.html\#files-modules-and-programs}{Files
Modules And Programs}. The module language that operates over these
signatures also extends to functors and first-class modules, described
in \href{functors.html\#functors}{Functors} and
\href{first-class-modules.html\#first-class-modules}{First Class
Modules}, respectively. \index{modules/separate compilation in}

This section discusses how the compiler implements them in more detail.
Modules are essential for larger projects that consist of many source
files (also known as \emph{compilation units}). It's impractical to
recompile every single source file when changing just one or two files,
and the module system minimizes such recompilation while still
encouraging code reuse. \index{compilation
units}

\hypertarget{the-mapping-between-files-and-modules}{%
\paragraph{The mapping between files and
modules}\label{the-mapping-between-files-and-modules}}

Individual compilation units provide a convenient way to break up a big
module hierarchy into a collection of files. The relationship between
files and modules can be explained directly in terms of the module
system. \index{files/relationship with modules}

Create a file called \passthrough{\lstinline!alice.ml!} with the
following contents:

\begin{lstlisting}[language=Caml]
let friends = [ Bob.name ]
\end{lstlisting}

and a corresponding signature file:

\begin{lstlisting}[language=Caml]
val friends : Bob.t list
\end{lstlisting}

These two files are exactly analogous to including the following code
directly in another module that references
\passthrough{\lstinline!Alice!}:

\begin{lstlisting}[language=Caml]
module Alice : sig
  val friends : Bob.t list
end = struct
  let friends = [ Bob.name ]
end
\end{lstlisting}

\hypertarget{defining-a-module-search-path}{%
\paragraph{Defining a module search
path}\label{defining-a-module-search-path}}

In the preceding example, \passthrough{\lstinline!Alice!} also has a
reference to another module \passthrough{\lstinline!Bob!}. For the
overall type of \passthrough{\lstinline!Alice!} to be valid, the
compiler also needs to check that the \passthrough{\lstinline!Bob!}
module contains at least a \passthrough{\lstinline!Bob.name!} value and
defines a \passthrough{\lstinline!Bob.t!} type.
\index{modules/defining search paths}

The type checker resolves such module references into concrete
structures and signatures in order to unify types across module
boundaries. It does this by searching a list of directories for a
compiled interface file matching that module's name. For example, it
will look for \passthrough{\lstinline!alice.cmi!} and
\passthrough{\lstinline!bob.cmi!} on the search path and use the first
ones it encounters as the interfaces for \passthrough{\lstinline!Alice!}
and \passthrough{\lstinline!Bob!}.

The module search path is set by adding \passthrough{\lstinline!-I!}
flags to the compiler command line with the directory containing the
\passthrough{\lstinline!cmi!} files as the argument. Manually specifying
these flags gets complex when you have lots of libraries, and is the
reason why the OCamlfind frontend to the compiler exists. OCamlfind
automates the process of turning third-party package names and build
descriptions into command-line flags that are passed to the compiler
command line.

By default, only the current directory and the OCaml standard library
will be searched for \passthrough{\lstinline!cmi!} files. The
\passthrough{\lstinline!Pervasives!} module from the standard library
will also be opened by default in every compilation unit. The standard
library location is obtained by running
\passthrough{\lstinline!ocamlc -where!} and can be overridden by setting
the \passthrough{\lstinline!CAMLLIB!} environment variable. Needless to
say, don't override the default path unless you have a good reason to
(such as setting up a cross-compilation environment).
\index{cmi files}\index{files/cmi
files}\index{OCaml toolchain/ocamlogjinfo}

\hypertarget{inspecting-compilation-units-with-ocamlobjinfo}{%
\subparagraph{Inspecting Compilation Units with
ocamlobjinfo}\label{inspecting-compilation-units-with-ocamlobjinfo}}

For separate compilation to be sound, we need to ensure that all the
\passthrough{\lstinline!cmi!} files used to type-check a module are the
same across compilation runs. If they vary, this raises the possibility
of two modules checking different type signatures for a common module
with the same name. This in turn lets the program completely violate the
static type system and can lead to memory corruption and crashes.

OCaml guards against this by recording a MD5 checksum in every
\passthrough{\lstinline!cmi!}. Let's examine our earlier
\passthrough{\lstinline!typedef.ml!} more closely:

\begin{lstlisting}[language=bash]
$ ocamlc -c typedef.ml
$ ocamlobjinfo typedef.cmi
File typedef.cmi
Unit name: Typedef
Interfaces imported:
    cdd43318ee9dd1b187513a4341737717    Typedef
    9b04ecdc97e5102c1d342892ef7ad9a2    Pervasives
    79ae8c0eb753af6b441fe05456c7970b    CamlinternalFormatBasics
\end{lstlisting}

\passthrough{\lstinline!ocamlobjinfo!} examines the compiled interface
and displays what other compilation units it depends on. In this case,
we don't use any external modules other than
\passthrough{\lstinline!Pervasives!}. Every module depends on
\passthrough{\lstinline!Pervasives!} by default, unless you use the
\passthrough{\lstinline!-nopervasives!} flag (this is an advanced use
case, and you shouldn't normally need it).

The long alphanumeric identifier beside each module name is a hash
calculated from all the types and values exported from that compilation
unit. It's used during type-checking and linking to ensure that all of
the compilation units have been compiled consistently against one
another. A difference in the hashes means that a compilation unit with
the same module name may have conflicting type signatures in different
modules. The compiler will reject such programs with an error similar to
this:

\begin{lstlisting}
$ ocamlc -c foo.ml
File "foo.ml", line 1, characters 0-1:
Error: The files /home/build/bar.cmi
       and /usr/lib/ocaml/map.cmi make inconsistent assumptions
       over interface Map
\end{lstlisting}

This hash check is very conservative, but ensures that separate
compilation remains type-safe all the way up to the final link phase.
Your build system should ensure that you never see the preceding error
messages, but if you do run into it, just clean out your intermediate
files and recompile from scratch.

\hypertarget{packing-modules-together}{%
\subsubsection{Packing Modules
Together}\label{packing-modules-together}}

The module-to-file mapping described so far rigidly enforces a 1:1
mapping between a top-level module and a file. It's often convenient to
split larger modules into separate files to make editing easier, but
still compile them all into a single OCaml module.
\index{modules/packing together}

The \passthrough{\lstinline!-pack!} compiler option accepts a list of
compiled object files ( \passthrough{\lstinline!.cmo!} in bytecode and
\passthrough{\lstinline!.cmx!} for native code) and their associated
\passthrough{\lstinline!.cmi!} compiled interfaces, and combines them
into a single module that contains them as submodules of the output.
Packing thus generates an entirely new \passthrough{\lstinline!.cmo!}
(or \passthrough{\lstinline!.cmx!} file) and
\passthrough{\lstinline!.cmi!} that includes the input modules.

Packing for native code introduces an additional requirement: the
modules that are intended to be packed must be compiled with the
\passthrough{\lstinline!-for-pack!} argument that specifies the eventual
name of the pack. The easiest way to handle packing is to let
\passthrough{\lstinline!ocamlbuild!} figure out the command-line
arguments for you, so let's try that out next with a simple example.

First, create a couple of toy modules called
\passthrough{\lstinline!A.ml!} and \passthrough{\lstinline!B.ml!} that
contain a single value. You will also need a
\passthrough{\lstinline!\_tags!} file that adds the
\passthrough{\lstinline!-for-pack!} option for the
\passthrough{\lstinline!cmx!} files (but careful to exclude the pack
target itself). Finally, the \passthrough{\lstinline!X.mlpack!} file
contains the list of modules that are intended to be packed under module
\passthrough{\lstinline!X!}. There are special rules in
\passthrough{\lstinline!ocamlbuild!} that tell it how to map
\passthrough{\lstinline!\%.mlpack!} files to the packed
\passthrough{\lstinline!\%.cmx!} or \passthrough{\lstinline!\%.cmo!}
equivalent:

\begin{lstlisting}[language=bash]
$ cat A.ml
let v = "hello"
$ cat B.ml
let w = 42
$ cat _tags
<*.cmx> and not "X.cmx": for-pack(X)
$ cat X.mlpack
A
B
\end{lstlisting}

You can now run \emph{corebuild} to build the
\passthrough{\lstinline!X.cmx!} file directly, but let's create a new
module to link against \passthrough{\lstinline!X!} to complete the
example:

\begin{lstlisting}[language=Caml]
let v = X.A.v
let w = X.B.w
\end{lstlisting}

You can now compile this test module and see that its inferred interface
is the result of using the packed contents of
\passthrough{\lstinline!X!}. We further verify this by examining the
imported interfaces in \passthrough{\lstinline!Test!} and confirming
that neither \passthrough{\lstinline!A!} nor \passthrough{\lstinline!B!}
are mentioned in there and that only the packed
\passthrough{\lstinline!X!} module is used:

\begin{lstlisting}[language=bash]
$ corebuild test.inferred.mli test.cmi
ocamlfind ocamldep -package core -ppx 'ppx-jane -as-ppx' -modules test.ml > test.ml.depends
ocamlfind ocamldep -package core -ppx 'ppx-jane -as-ppx' -modules A.ml > A.ml.depends
ocamlfind ocamldep -package core -ppx 'ppx-jane -as-ppx' -modules B.ml > B.ml.depends
ocamlfind ocamlc -c -w A-4-33-40-41-42-43-34-44 -strict-sequence -g -bin-annot -short-paths -thread -package core -ppx 'ppx-jane -as-ppx' -o A.cmo A.ml
ocamlfind ocamlc -c -w A-4-33-40-41-42-43-34-44 -strict-sequence -g -bin-annot -short-paths -thread -package core -ppx 'ppx-jane -as-ppx' -o B.cmo B.ml
ocamlfind ocamlc -pack -g -bin-annot A.cmo B.cmo -o X.cmo
ocamlfind ocamlc -i -thread -short-paths -package core -ppx 'ppx-jane -as-ppx' test.ml > test.inferred.mli
ocamlfind ocamlc -c -w A-4-33-40-41-42-43-34-44 -strict-sequence -g -bin-annot -short-paths -thread -package core -ppx 'ppx-jane -as-ppx' -o test.cmo test.ml
$ cat _build/test.inferred.mli
val v : string
val w : int
$ ocamlobjinfo _build/test.cmi
File _build/test.cmi
Unit name: Test
Interfaces imported:
    7b1e33d4304b9f8a8e844081c001ef22    Test
    27a343af5f1904230d1edc24926fde0e    X
    9b04ecdc97e5102c1d342892ef7ad9a2    Pervasives
    79ae8c0eb753af6b441fe05456c7970b    CamlinternalFormatBasics
\end{lstlisting}

\hypertarget{packing-and-search-paths}{%
\paragraph{Packing and Search Paths}\label{packing-and-search-paths}}

One very common build error that happens with packing is confusion
resulting from building the packed \passthrough{\lstinline!cmi!} in the
same directory as the submodules. When you add this directory to your
module search path, the submodules are also visible. If you forget to
include the top-level prefix (e.g., \passthrough{\lstinline!X.A!}) and
instead use a submodule directly (\passthrough{\lstinline!A!}), then
this will compile and link fine.

However, the types of \passthrough{\lstinline!A!} and
\passthrough{\lstinline!X.A!} are \emph{not} automatically equivalent so
the type checker will complain if you attempt to mix and match the
packed and unpacked versions of the library.

This mostly only happens with unit tests, since they are built at the
same time as the library. You can avoid it by being aware of the need to
open the packed module from the test, or only using the library after it
has been installed (and hence not exposing the intermediate compiled
modules).

\hypertarget{shorter-module-paths-in-type-errors}{%
\subsubsection{Shorter Module Paths in Type
Errors}\label{shorter-module-paths-in-type-errors}}

Core uses the OCaml module system quite extensively to provide a
complete replacement standard library. It collects these modules into a
single \passthrough{\lstinline!Std!} module, which provides a single
module that needs to be opened to import the replacement modules and
functions. \index{errors/reducing verbosity
in}

There's one downside to this approach: type errors suddenly get much
more verbose. We can see this if you run the vanilla OCaml toplevel (not
\passthrough{\lstinline!utop!}).

\begin{lstlisting}
$ ocaml
# List.map print_endline "" ;;
Error: This expression has type string but an expression was expected of type
         string list
\end{lstlisting}

This type error without \passthrough{\lstinline!Core!} has a
straightforward type error. When we switch to Core, though, it gets more
verbose:

\begin{lstlisting}
$ ocaml
# open Core ;;
# List.map ~f:print_endline "" ;;
Error: This expression has type string but an expression was expected of type
         'a Core.List.t = 'a list
\end{lstlisting}

The default \passthrough{\lstinline!List!} module in OCaml is overridden
by \passthrough{\lstinline!Core.List!}. The compiler does its best to
show the type equivalence, but at the cost of a more verbose error
message.

The compiler can remedy this via a so-called short paths heuristic. This
causes the compiler to search all the type aliases for the shortest
module path and use that as the preferred output type. The option is
activated by passing \passthrough{\lstinline!-short-paths!} to the
compiler, and works on the toplevel, too.\index{short
paths heuristic}

\begin{lstlisting}
$ ocaml -short-paths
# open Core;;
# List.map ~f:print_endline "foo";;
Error: This expression has type string but an expression was expected of type
         'a list
\end{lstlisting}

The \passthrough{\lstinline!utop!} enhanced toplevel activates short
paths by default, which is why we have not had to do this before in our
interactive examples. However, the compiler doesn't default to the short
path heuristic, since there are some situations where the type aliasing
information is useful to know, and it would be lost in the error if the
shortest module path is always picked.

You'll need to choose for yourself if you prefer short paths or the
default behavior in your own projects, and pass the
\passthrough{\lstinline!-short-paths!} flag to the compiler if you need
it.~

\hypertarget{the-typed-syntax-tree}{%
\subsection{The Typed Syntax Tree}\label{the-typed-syntax-tree}}

When the type checking process has successfully completed, it is
combined with the AST to form a \emph{typed abstract syntax tree}. This
contains precise location information for every token in the input file,
and decorates each token with concrete type
information.\index{cmti files}\index{cmt
files}\index{files/cmtii files}\index{files/cmt files}\index{AST (abstract
syntax-tree)}\protect\hypertarget{typesyntree}{}{typed syntax
tree}\protect\hypertarget{CPtypsyn}{}{compilation process/typed syntax
tree}

The compiler can output this as compiled \passthrough{\lstinline!cmt!}
and \passthrough{\lstinline!cmti!} files that contain the typed AST for
the implementation and signatures of a compilation unit. This is
activated by passing the \passthrough{\lstinline!-bin-annot!} flag to
the compiler.

The \passthrough{\lstinline!cmt!} files are particularly useful for IDE
tools to match up OCaml source code at a specific location to the
inferred or external types.

\hypertarget{using-ocp-index-for-auto-completion}{%
\subsubsection{Using ocp-index for
Autocompletion}\label{using-ocp-index-for-auto-completion}}

One such command-line tool to display autocompletion information in your
editor is \passthrough{\lstinline!ocp-index!}. Install it via OPAM as
follows:\index{autocompletion}\index{ocp-index}

\begin{lstlisting}[language=bash]
$ opam install ocp-index
$ ocp-index
\end{lstlisting}

Let's refer back to our Ncurses binding example from the beginning of
\href{foreign-function-interface.html\#foreign-function-interface}{Foreign
Function Interface}. This module defined bindings for the Ncurses
library. First, compile the interfaces with
\passthrough{\lstinline!-bin-annot!} so that we can obtain the
\passthrough{\lstinline!cmt!} and \passthrough{\lstinline!cmti!} files,
and then run \passthrough{\lstinline!ocp-index!} in completion mode:

\begin{lstlisting}[language=bash]
$ (cd ffi/ncurses && corebuild -pkg ctypes.foreign -tag bin_annot ncurses.cmi)
ocamlfind ocamldep -package ctypes.foreign -package core -ppx 'ppx-jane -as-ppx' -modules ncurses.mli > ncurses.mli.depends
ocamlfind ocamlc -c -w A-4-33-40-41-42-43-34-44 -strict-sequence -g -bin-annot -short-paths -thread -package ctypes.foreign -package core -ppx 'ppx-jane -as-ppx' -o ncurses.cmi ncurses.mli
$ ocp-index complete -I ffi Ncur
$ ocp-index complete -I ffi Ncurses.a
$ ocp-index complete -I ffi Ncurses.
\end{lstlisting}

You need to pass \passthrough{\lstinline!ocp-index!} a set of
directories to search for \passthrough{\lstinline!cmt!} files in, and a
fragment of text to autocomplete. As you can imagine, autocompletion is
invaluable on larger codebases. See the
\href{https://github.com/ocamlpro/ocp-index}{\emph{ocp-index}} home page
for more information on how to integrate it with your favorite editor.

\hypertarget{examining-the-typed-syntax-tree-directly}{%
\subsubsection{Examining the Typed Syntax Tree
Directly}\label{examining-the-typed-syntax-tree-directly}}

The compiler has a couple of advanced flags that can dump the raw output
of the internal AST representation. You can't depend on these flags to
give the same output across compiler revisions, but they are a useful
learning tool.\index{flags}

We'll use our toy \passthrough{\lstinline!typedef.ml!} again:

\begin{lstlisting}[language=Caml]
type t = Foo | Bar
let v = Foo
\end{lstlisting}

Let's first look at the untyped syntax tree that's generated from the
parsing phase:

\begin{lstlisting}[language=bash]
$ ocamlc -dparsetree typedef.ml 2>&1
[
  structure_item (typedef.ml[1,0+0]..[1,0+18])
    Pstr_type Rec
    [
      type_declaration "t" (typedef.ml[1,0+5]..[1,0+6]) (typedef.ml[1,0+0]..[1,0+18])
        ptype_params =
          []
        ptype_cstrs =
          []
        ptype_kind =
          Ptype_variant
            [
              (typedef.ml[1,0+9]..[1,0+12])
                "Foo" (typedef.ml[1,0+9]..[1,0+12])
                []
                None
              (typedef.ml[1,0+13]..[1,0+18])
                "Bar" (typedef.ml[1,0+15]..[1,0+18])
                []
                None
            ]
        ptype_private = Public
        ptype_manifest =
          None
    ]
  structure_item (typedef.ml[2,19+0]..[2,19+11])
    Pstr_value Nonrec
    [
      <def>
        pattern (typedef.ml[2,19+4]..[2,19+5])
          Ppat_var "v" (typedef.ml[2,19+4]..[2,19+5])
        expression (typedef.ml[2,19+8]..[2,19+11])
          Pexp_construct "Foo" (typedef.ml[2,19+8]..[2,19+11])
          None
    ]
]
\end{lstlisting}

This is rather a lot of output for a simple two-line program, but it
shows just how much structure the OCaml parser generates even from a
small source file.

Each portion of the AST is decorated with the precise location
information (including the filename and character location of the
token). This code hasn't been type checked yet, so the raw tokens are
all included.

The typed AST that is normally output as a compiled
\passthrough{\lstinline!cmt!} file can be displayed in a more
developer-readable form via the \passthrough{\lstinline!-dtypedtree!}
option:

\begin{lstlisting}[language=bash]
$ ocamlc -dtypedtree typedef.ml 2>&1
[
  structure_item (typedef.ml[1,0+0]..typedef.ml[1,0+18])
    Tstr_type Rec
    [
      type_declaration t/80 (typedef.ml[1,0+0]..typedef.ml[1,0+18])
        ptype_params =
          []
        ptype_cstrs =
          []
        ptype_kind =
          Ttype_variant
            [
              (typedef.ml[1,0+9]..typedef.ml[1,0+12])
                Foo/81
                []
                None
              (typedef.ml[1,0+13]..typedef.ml[1,0+18])
                Bar/82
                []
                None
            ]
        ptype_private = Public
        ptype_manifest =
          None
    ]
  structure_item (typedef.ml[2,19+0]..typedef.ml[2,19+11])
    Tstr_value Nonrec
    [
      <def>
        pattern (typedef.ml[2,19+4]..typedef.ml[2,19+5])
          Tpat_var "v/83"
        expression (typedef.ml[2,19+8]..typedef.ml[2,19+11])
          Texp_construct "Foo"
          []
    ]
]
\end{lstlisting}

The typed AST is more explicit than the untyped syntax tree. For
instance, the type declaration has been given a unique name
(\passthrough{\lstinline!t/1008!}), as has the
\passthrough{\lstinline!v!} value (\passthrough{\lstinline!v/1011!}). ~~

You'll rarely need to look at this raw output from the compiler unless
you're building IDE tools such as \passthrough{\lstinline!ocp-index!},
or are hacking on extensions to the core compiler itself. However, it's
useful to know that this intermediate form exists before we delve
further into the code generation process next, in
\href{compiler-backend.html\#the-compiler-backend-byte-code-and-native-code}{The
Compiler Backend Byte Code And Native Code}.

There are several new integrated tools emerging that combine these typed
AST files with common editors such as Emacs or Vim. The best of these is
\href{https://github.com/def-lkb/merlin}{Merlin}, which adds value and
module autocompletion, displays inferred types and can build and display
errors directly from within your editor. There are instructions
available on its homepage for configuring Merlin with your favorite
editor.
