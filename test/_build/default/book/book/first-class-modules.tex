\hypertarget{first-class-modules}{%
\section{First-Class Modules}\label{first-class-modules}}

You can think of OCaml as being broken up into two parts: a core
language that is concerned with values and types, and a module language
that is concerned with modules and module signatures. These sublanguages
are stratified, in that modules can contain types and values, but
ordinary values can't contain modules or module types. That means you
can't do things like define a variable whose value is a module, or a
function that takes a module as an argument.
\protect\hypertarget{MODfirst}{}{modules/first-class modules}

OCaml provides a way around this stratification in the form of
\emph{first-class modules}. First-class modules are ordinary values that
can be created from and converted back to regular modules.
\protect\hypertarget{FCMwork}{}{first-class modules/working with}

First-class modules are a sophisticated technique, and you'll need to
get comfortable with some advanced aspects of the language to use them
effectively. But it's worth learning, because letting modules into the
core language is quite powerful, increasing the range of what you can
express and making it easier to build flexible and modular {systems}.

\hypertarget{working-with-first-class-modules}{%
\subsection{Working with First-Class
Modules}\label{working-with-first-class-modules}}

We'll start out by covering the basic mechanics of first-class modules
by working through some toy examples. We'll get to more realistic
examples in the next section.

In that light, consider the following signature of a module with a
single integer variable:

\begin{lstlisting}[language=Caml]
# open Base
# module type X_int = sig val x : int end
module type X_int = sig val x : int end
\end{lstlisting}

We can also create a module that matches this signature:

\begin{lstlisting}[language=Caml]
# module Three : X_int = struct let x = 3 end
module Three : X_int
# Three.x
- : int = 3
\end{lstlisting}

A first-class module is created by packaging up a module with a
signature that it satisfies. This is done using the
\passthrough{\lstinline!module!} keyword. \index{module
keyword}

\begin{lstlisting}
(module <Module> : <Module_type>)
\end{lstlisting}

We can convert \passthrough{\lstinline!Three!} into a first-class module
as follows:

\begin{lstlisting}[language=Caml]
# let three = (module Three : X_int)
val three : (module X_int) = <module>
\end{lstlisting}

The module type doesn't need to be part of the construction of a
first-class module if it can be inferred. Thus, we can write:

\begin{lstlisting}[language=Caml]
# module Four = struct let x = 4 end
module Four : sig val x : int end
# let numbers = [ three; (module Four) ]
val numbers : (module X_int) list = [<module>; <module>]
\end{lstlisting}

We can also create a first-class module from an anonymous module:

\begin{lstlisting}[language=Caml]
# let numbers = [three; (module struct let x = 4 end)]
val numbers : (module X_int) list = [<module>; <module>]
\end{lstlisting}

In order to access the contents of a first-class module, you need to
unpack it into an ordinary module. This can be done using the
\passthrough{\lstinline!val!} keyword, using this syntax:

\begin{lstlisting}
(val <first_class_module> : <Module_type>)
\end{lstlisting}

Here's an example:

\begin{lstlisting}[language=Caml]
# module New_three = (val three : X_int)
module New_three : X_int
# New_three.x
- : int = 3
\end{lstlisting}

We can also write ordinary functions which consume and create
first-class modules. The following shows the definition of two
functions: \passthrough{\lstinline!to\_int!}, which converts a
\passthrough{\lstinline!(module X\_int)!} into an
\passthrough{\lstinline!int!}; and \passthrough{\lstinline!plus!}, which
returns the sum of two \passthrough{\lstinline!(module X\_int)!}:

\begin{lstlisting}[language=Caml]
# let to_int m =
    let module M = (val m : X_int) in
    M.x
val to_int : (module X_int) -> int = <fun>
# let plus m1 m2 =
    (module struct
      let x = to_int m1 + to_int m2
    end : X_int)
val plus : (module X_int) -> (module X_int) -> (module X_int) = <fun>
\end{lstlisting}

With these functions in hand, we can now work with values of type
\passthrough{\lstinline!(module X\_int)!} in a more natural style,
taking advantage of the concision and simplicity of the core {language}:

\begin{lstlisting}[language=Caml]
# let six = plus three three
val six : (module X_int) = <module>
# to_int (List.fold ~init:six ~f:plus [three;three])
- : int = 12
\end{lstlisting}

There are some useful syntactic shortcuts when dealing with first-class
modules. One notable one is that you can do the conversion to an
ordinary module within a pattern match. Thus, we can rewrite the
\passthrough{\lstinline!to\_int!} function as follows:

\begin{lstlisting}[language=Caml]
# let to_int (module M : X_int) = M.x
val to_int : (module X_int) -> int = <fun>
\end{lstlisting}

First-class modules can contain types and functions in addition to
simple values like \passthrough{\lstinline!int!}. Here's an interface
that contains a type and a corresponding \passthrough{\lstinline!bump!}
operation that takes a value of the type and produces a new one:

\begin{lstlisting}[language=Caml]
# module type Bumpable = sig
    type t
    val bump : t -> t
  end
module type Bumpable = sig type t val bump : t -> t end
\end{lstlisting}

We can create multiple instances of this module with different
underlying types:

\begin{lstlisting}[language=Caml]
# module Int_bumper = struct
    type t = int
    let bump n = n + 1
  end
module Int_bumper : sig type t = int val bump : t -> t end
# module Float_bumper = struct
    type t = float
    let bump n = n +. 1.
  end
module Float_bumper : sig type t = float val bump : t -> t end
\end{lstlisting}

And we can convert these to first-class modules:

\begin{lstlisting}[language=Caml]
# let int_bumper = (module Int_bumper : Bumpable)
val int_bumper : (module Bumpable) = <module>
\end{lstlisting}

But you can't do much with \passthrough{\lstinline!int\_bumper!}, since
\passthrough{\lstinline!int\_bumper!} is fully abstract, so that we can
no longer recover the fact that the type in question is
\passthrough{\lstinline!int!}.

\begin{lstlisting}[language=Caml]
# let (module Bumpable) = int_bumper in Bumpable.bump 3
Line 1, characters 53-54:
Error: This expression has type int but an expression was expected of type
         Bumpable.t
\end{lstlisting}

To make \passthrough{\lstinline!int\_bumper!} usable, we need to expose
the type, which we can do as follows:

\begin{lstlisting}[language=Caml]
# let int_bumper = (module Int_bumper : Bumpable with type t = int)
val int_bumper : (module Bumpable with type t = int) = <module>
# let float_bumper = (module Float_bumper : Bumpable with type t = float)
val float_bumper : (module Bumpable with type t = float) = <module>
\end{lstlisting}

The sharing constraints we've added above make the resulting first-class
modules {polymorphic} in the type \passthrough{\lstinline!t!}. As a
result, we can now use these first-class modules on values of the
matching type:

\begin{lstlisting}[language=Caml]
# let (module Bumpable) = int_bumper in Bumpable.bump 3
- : int = 4
# let (module Bumpable) = float_bumper in Bumpable.bump 3.5
- : float = 4.5
\end{lstlisting}

We can also write functions that use such first-class modules
polymorphically. The following function takes two arguments: a
\passthrough{\lstinline!Bumpable!} module and a list of elements of the
same type as the type \passthrough{\lstinline!t!} of the module:
\index{polymorphism/in first-class modules}\index{first-class modules/polymorphism
in}

\begin{lstlisting}[language=Caml]
# let bump_list
        (type a)
        (module B : Bumpable with type t = a)
        (l: a list)
    =
    List.map ~f:B.bump l
val bump_list : (module Bumpable with type t = 'a) -> 'a list -> 'a list =
  <fun>
\end{lstlisting}

Here, we used a feature of OCaml that hasn't come up before: a
\emph{locally abstract type}. For any function, you can declare a
pseudoparameter of the form \passthrough{\lstinline!(type a)!} which
introduces a fresh type named \passthrough{\lstinline!a!}. This type
acts like an abstract type within the context of the function. In the
example above, the locally abstract type was used as part of a sharing
constraint that ties the type \passthrough{\lstinline!B.t!} with the
type of the elements of the list passed in.
\index{datatypes/locally abstract types}\index{abstract types}\index{locally
abstract types}\index{sharing constraint}

The resulting function is polymorphic in both the type of the list
element and the type \passthrough{\lstinline!Bumpable.t!}. We can see
this function in action:

\begin{lstlisting}[language=Caml]
# bump_list int_bumper [1;2;3]
- : int list = [2; 3; 4]
# bump_list float_bumper [1.5;2.5;3.5]
- : float list = [2.5; 3.5; 4.5]
\end{lstlisting}

Polymorphic first-class modules are important because they allow you to
connect the types associated with a first-class module to the types of
other values you're working with.

\hypertarget{more-on-locally-abstract-types}{%
\subsubsection{More on Locally Abstract
Types}\label{more-on-locally-abstract-types}}

One of the key properties of locally abstract types is that they're
dealt with as abstract types in the function they're defined within, but
are polymorphic from the outside. Consider the following example:
\index{polymorphism/in locally abstract types}

\begin{lstlisting}[language=Caml]
# let wrap_in_list (type a) (x:a) = [x]
val wrap_in_list : 'a -> 'a list = <fun>
\end{lstlisting}

This compiles successfully because the type \passthrough{\lstinline!a!}
is used in a way that is compatible with it being abstract, but the type
of the function that is inferred is polymorphic.

If, on the other hand, we try to use the type
\passthrough{\lstinline!a!} as equivalent to some concrete type, say,
\passthrough{\lstinline!int!}, then the compiler will complain:

\begin{lstlisting}[language=Caml]
# let double_int (type a) (x:a) = x + x
Line 1, characters 33-34:
Error: This expression has type a but an expression was expected of type int
\end{lstlisting}

One common use of locally abstract types is to create a new type that
can be used in constructing a module. Here's an example of doing this to
create a new first-class module:

\begin{lstlisting}[language=Caml]
# module type Comparable = sig
    type t
    val compare : t -> t -> int
  end
module type Comparable = sig type t val compare : t -> t -> int end
# let create_comparable (type a) compare =
    (module struct
      type t = a
      let compare = compare
    end : Comparable with type t = a)
val create_comparable :
  ('a -> 'a -> int) -> (module Comparable with type t = 'a) = <fun>
# create_comparable Int.compare
- : (module Comparable with type t = int) = <module>
# create_comparable Float.compare
- : (module Comparable with type t = float) = <module>
\end{lstlisting}

Here, what we effectively do is capture a polymorphic type and export it
as a concrete type within a module.

This technique is useful beyond first-class modules. For example, we can
use the same approach to construct a local module to be fed to a
functor. ~

\hypertarget{example-a-query-handling-framework}{%
\subsection{Example: A Query-Handling
Framework}\label{example-a-query-handling-framework}}

Now let's look at first-class modules in the context of a more complete
and realistic example. In particular, consider the following signature
for a module that implements a system for responding to user-generated
queries.
\index{query-handlers/and first-class modules}\protect\hypertarget{FCMquery}{}{first-class
modules/query-handling framework}

\begin{lstlisting}[language=Caml]
# module type Query_handler = sig

    (** Configuration for a query handler.  Note that this can be
        converted to and from an s-expression *)
    type config [@@deriving sexp]

    (** The name of the query-handling service *)
    val name : string

    (** The state of the query handler *)
    type t

    (** Creates a new query handler from a config *)
    val create : config -> t

    (** Evaluate a given query, where both input and output are
        s-expressions *)
    val eval : t -> Sexp.t -> Sexp.t Or_error.t
  end
module type Query_handler =
  sig
    type config
    val sexp_of_config : config -> Sexp.t
    val config_of_sexp : Sexp.t -> config
    val name : string
    type t
    val create : config -> t
    val eval : t -> Sexp.t -> Sexp.t Or_error.t
  end
\end{lstlisting}

Here, we used s-expressions as the format for queries and responses, as
well as the configuration for the query handler. S-expressions are a
simple, flexible, and human-readable serialization format commonly used
in Core. For now, it's enough to think of them as balanced parenthetical
expressions whose atomic values are strings, e.g.,
\passthrough{\lstinline!(this (is an) (s expression))!}.\index{s-expressions/in queries and
responses}

In addition, we use the \passthrough{\lstinline!ppx\_sexp\_conv!} syntax
extension which interprets the
\passthrough{\lstinline![@@deriving sexp]!} annotation. When
\passthrough{\lstinline!ppx\_sexp\_conv!} sees
\passthrough{\lstinline![@@deriving sexp]!} attached to a signature, it
replaces it with declarations of s-expression converters, for
example:\index{sexp declaration}

\begin{lstlisting}[language=Caml]
# module type M = sig type t [@@deriving sexp] end
module type M =
  sig type t val t_of_sexp : Sexp.t -> t val sexp_of_t : t -> Sexp.t end
\end{lstlisting}

In a module, \passthrough{\lstinline![@@deriving sexp]!} adds the
implementation of those functions. Thus, we can write:

\begin{lstlisting}[language=Caml]
# type u = { a: int; b: float } [@@deriving sexp]
type u = { a : int; b : float; }
val u_of_sexp : Sexp.t -> u = <fun>
val sexp_of_u : u -> Sexp.t = <fun>
# sexp_of_u {a=3;b=7.}
- : Sexp.t = ((a 3) (b 7))
# u_of_sexp (Core_kernel.Sexp.of_string "((a 43) (b 3.4))")
- : u = {a = 43; b = 3.4}
\end{lstlisting}

This is all described in more detail in
\href{data-serialization.html\#data-serialization-with-s-expressions}{Data
Serialization With S Expressions}.

\hypertarget{implementing-a-query-handler}{%
\subsubsection{Implementing a Query
Handler}\label{implementing-a-query-handler}}

Let's look at some examples of query handlers that satisfy the
\passthrough{\lstinline!Query\_handler!} interface. The first example is
a handler that produces unique integer IDs. It works by keeping an
internal counter which it bumps every time it produces a new value. The
input to the query in this case is just the trivial s-expression
\passthrough{\lstinline!()!}, otherwise known as
\passthrough{\lstinline!Sexp.unit!}:
\index{query-handlers/implementation of}

\begin{lstlisting}[language=Caml]
# module Unique = struct
    type config = int [@@deriving sexp]
    type t = { mutable next_id: int }

    let name = "unique"
    let create start_at = { next_id = start_at }

    let eval t sexp =
      match Or_error.try_with (fun () -> unit_of_sexp sexp) with
      | Error _ as err -> err
      | Ok () ->
        let response = Ok (Int.sexp_of_t t.next_id) in
        t.next_id <- t.next_id + 1;
        response
  end
module Unique :
  sig
    type config = int
    val config_of_sexp : Sexp.t -> config
    val sexp_of_config : config -> Sexp.t
    type t = { mutable next_id : config; }
    val name : string
    val create : config -> t
    val eval : t -> Sexp.t -> (Sexp.t, Error.t) result
  end
\end{lstlisting}

We can use this module to create an instance of the
\passthrough{\lstinline!Unique!} query handler and interact with it
directly:

\begin{lstlisting}[language=Caml]
# let unique = Unique.create 0
val unique : Unique.t = {Unique.next_id = 0}
# Unique.eval unique (Sexp.List [])
- : (Sexp.t, Error.t) result = Ok 0
# Unique.eval unique (Sexp.List [])
- : (Sexp.t, Error.t) result = Ok 1
\end{lstlisting}

Here's another example: a query handler that does directory listings.
Here, the config is the default directory that relative paths are
interpreted within:

\begin{lstlisting}[language=Caml]
# module List_dir = struct
    type config = string [@@deriving sexp]
    type t = { cwd: string }

    (** [is_abs p] Returns true if [p] is an absolute path  *)
    let is_abs p =
      String.length p > 0 && Char.(=) p.[0] '/'

    let name = "ls"
    let create cwd = { cwd }

    let eval t sexp =
      match Or_error.try_with (fun () -> string_of_sexp sexp) with
      | Error _ as err -> err
      | Ok dir ->
        let dir =
          if is_abs dir then dir
          else Core.Filename.concat t.cwd dir
        in
        Ok (Array.sexp_of_t String.sexp_of_t (Core.Sys.readdir dir))
  end
module List_dir :
  sig
    type config = string
    val config_of_sexp : Sexp.t -> config
    val sexp_of_config : config -> Sexp.t
    type t = { cwd : config; }
    val is_abs : config -> bool
    val name : config
    val create : config -> t
    val eval : t -> Sexp.t -> (Sexp.t, Error.t) result
  end
\end{lstlisting}

Again, we can create an instance of this query handler and interact with
it directly:

\begin{lstlisting}[language=Caml]
# let list_dir = List_dir.create "/var"
val list_dir : List_dir.t = {List_dir.cwd = "/var"}
# List_dir.eval list_dir (sexp_of_string ".")
- : (Sexp.t, Error.t) result =
Ok
 (yp networkd install empty ma mail spool jabberd vm msgs audit root lib db
  at log folders netboot run rpc tmp backups agentx rwho)
# List_dir.eval list_dir (sexp_of_string "yp")
- : (Sexp.t, Error.t) result = Ok (binding)
\end{lstlisting}

\hypertarget{dispatching-to-multiple-query-handlers}{%
\subsubsection{Dispatching to Multiple Query
Handlers}\label{dispatching-to-multiple-query-handlers}}

Now, what if we want to dispatch queries to any of an arbitrary
collection of handlers? Ideally, we'd just like to pass in the handlers
as a simple data structure like a list. This is awkward to do with
modules and functors alone, but it's quite natural with first-class
modules. The first thing we'll need to do is create a signature that
combines a \passthrough{\lstinline!Query\_handler!} module with an
instantiated query
handler:\index{query-handlers/dispatching to multiple}

\begin{lstlisting}[language=Caml]
# module type Query_handler_instance = sig
    module Query_handler : Query_handler
    val this : Query_handler.t
  end
module type Query_handler_instance =
  sig module Query_handler : Query_handler val this : Query_handler.t end
\end{lstlisting}

With this signature, we can create a first-class module that encompasses
both an instance of the query and the matching operations for working
with that query.

We can create an instance as follows:

\begin{lstlisting}[language=Caml]
# let unique_instance =
    (module struct
      module Query_handler = Unique
      let this = Unique.create 0
  end : Query_handler_instance)
val unique_instance : (module Query_handler_instance) = <module>
\end{lstlisting}

Constructing instances in this way is a little verbose, but we can write
a function that eliminates most of this boilerplate. Note that we are
again making use of a locally abstract type:

\begin{lstlisting}[language=Caml]
# let build_instance
        (type a)
        (module Q : Query_handler with type config = a)
        config
    =
    (module struct
      module Query_handler = Q
      let this = Q.create config
    end : Query_handler_instance)
val build_instance :
  (module Query_handler with type config = 'a) ->
  'a -> (module Query_handler_instance) = <fun>
\end{lstlisting}

Using \passthrough{\lstinline!build\_instance!}, constructing a new
instance becomes a one-liner:

\begin{lstlisting}[language=Caml]
# let unique_instance = build_instance (module Unique) 0
val unique_instance : (module Query_handler_instance) = <module>
# let list_dir_instance = build_instance (module List_dir)  "/var"
val list_dir_instance : (module Query_handler_instance) = <module>
\end{lstlisting}

We can now write code that lets you dispatch queries to one of a list of
query handler instances. We assume that the shape of the query is as
follows:

\begin{lstlisting}
(query-name query)
\end{lstlisting}

where \emph{\passthrough{\lstinline!query-name!}} is the name used to
determine which query handler to dispatch the query to, and
\emph{\passthrough{\lstinline!query!}} is the body of the query.

The first thing we'll need is a function that takes a list of query
handler instances and constructs a dispatch table from it:

\begin{lstlisting}[language=Caml]
# let build_dispatch_table handlers =
    let table = Hashtbl.create (module String) in
    List.iter handlers
      ~f:(fun ((module I : Query_handler_instance) as instance) ->
        Hashtbl.set table ~key:I.Query_handler.name ~data:instance);
    table
val build_dispatch_table :
  (module Query_handler_instance) list ->
  (string, (module Query_handler_instance)) Core_kernel.Hashtbl.t = <fun>
\end{lstlisting}

Now, we need a function that dispatches to a handler using a dispatch
table:

\begin{lstlisting}[language=Caml]
# let dispatch dispatch_table name_and_query =
    match name_and_query with
    | Sexp.List [Sexp.Atom name; query] ->
      begin match Hashtbl.find dispatch_table name with
      | None ->
        Or_error.error "Could not find matching handler"
          name String.sexp_of_t
      | Some (module I : Query_handler_instance) ->
        I.Query_handler.eval I.this query
      end
    | _ ->
      Or_error.error_string "malformed query"
val dispatch :
  (string, (module Query_handler_instance)) Core_kernel.Hashtbl.t ->
  Sexp.t -> Sexp.t Or_error.t = <fun>
\end{lstlisting}

This function interacts with an instance by unpacking it into a module
\passthrough{\lstinline!I!} and then using the query handler instance
(\passthrough{\lstinline!I.this!}) in concert with the associated module
(\passthrough{\lstinline!I.Query\_handler!}).\index{I.Query\_handler module}

The bundling together of the module and the value is in many ways
reminiscent of object-oriented languages. One key difference, is that
first-class modules allow you to package up more than just functions or
methods. As we've seen, you can also include types and even modules.
We've only used it in a small way here, but this extra power allows you
to build more sophisticated components that involve multiple
interdependent types and values.

Now let's turn this into a complete, running example by adding a
command-line interface:

\begin{lstlisting}[language=Caml]
# open Stdio
# let rec cli dispatch_table =
    printf ">>> %!";
    let result =
      match In_channel.(input_line stdin) with
      | None -> `Stop
      | Some line ->
        match Or_error.try_with (fun () ->
          Core_kernel.Sexp.of_string line)
        with
        | Error e -> `Continue (Error.to_string_hum e)
        | Ok (Sexp.Atom "quit") -> `Stop
        | Ok query ->
          begin match dispatch dispatch_table query with
          | Error e -> `Continue (Error.to_string_hum e)
          | Ok s    -> `Continue (Sexp.to_string_hum s)
          end;
    in
    match result with
    | `Stop -> ()
    | `Continue msg ->
      printf "%s\n%!" msg;
      cli dispatch_table
val cli :
  (string, (module Query_handler_instance)) Core_kernel.Hashtbl.t -> unit =
  <fun>
\end{lstlisting}

We can most effectively run this command-line interface from a
standalone program, which we can do by putting the above code in a file
along with following command to launch the interface:

\begin{lstlisting}[language=Caml]
let () =
  cli (build_dispatch_table [unique_instance; list_dir_instance])
\end{lstlisting}

Here's an example of a session with this program:

\begin{lstlisting}[language=bash]
$ dune exec ./query_handler.exe
>>> (unique ())
0
>>> (unique ())
1
>>> (ls .)
(agentx at audit backups db empty folders jabberd lib log mail msgs named
 netboot pgsql_socket_alt root rpc run rwho spool tmp vm yp)
>>> (ls vm)
(sleepimage swapfile0 swapfile1 swapfile2 swapfile3 swapfile4 swapfile5
 swapfile6)
\end{lstlisting}

\hypertarget{loading-and-unloading-query-handlers}{%
\subsubsection{Loading and Unloading Query
Handlers}\label{loading-and-unloading-query-handlers}}

One of the advantages of first-class modules is that they afford a great
deal of dynamism and flexibility. For example, it's a fairly simple
matter to change our design to allow query handlers to be loaded and
unloaded at runtime.\index{query-handlers/loading/unloading of}

We'll do this by creating a query handler whose job is to control the
set of active query handlers. The module in question will be called
\passthrough{\lstinline!Loader!}, and its configuration is a list of
known \passthrough{\lstinline!Query\_handler!} modules. Here are the
basic types:

\begin{lstlisting}[language=Caml]
module Loader = struct
  type config = (module Query_handler) list sexp_opaque
  [@@deriving sexp]

  type t = { known  : (module Query_handler)          String.Table.t
           ; active : (module Query_handler_instance) String.Table.t
           }

  let name = "loader"
\end{lstlisting}

Note that a \passthrough{\lstinline!Loader.t!} has two tables: one
containing the known query handler modules, and one containing the
active query handler instances. The \passthrough{\lstinline!Loader.t!}
will be responsible for creating new instances and adding them to the
table, as well as for removing instances, all in response to user
queries.

Next, we'll need a function for creating a
\passthrough{\lstinline!Loader.t!}. This function requires the list of
known query handler modules. Note that the table of active modules
starts out as empty:

\begin{lstlisting}[language=Caml]
let create known_list =
    let active = String.Table.create () in
    let known  = String.Table.create () in
    List.iter known_list
      ~f:(fun ((module Q : Query_handler) as q) ->
        Hashtbl.set known ~key:Q.name ~data:q);
    { known; active }
\end{lstlisting}

Now we'll start writing out the functions for manipulating the table of
active query handlers. We'll start with the function for loading an
instance. Note that it takes as an argument both the name of the query
handler and the configuration for instantiating that handler in the form
of an s-expression. These are used for creating a first-class module of
type \passthrough{\lstinline!(module Query\_handler\_instance)!}, which
is then added to the active table:

\begin{lstlisting}[language=Caml]
let load t handler_name config =
    if Hashtbl.mem t.active handler_name then
      Or_error.error "Can't re-register an active handler"
        handler_name String.sexp_of_t
    else
      match Hashtbl.find t.known handler_name with
      | None ->
        Or_error.error "Unknown handler" handler_name String.sexp_of_t
      | Some (module Q : Query_handler) ->
        let instance =
          (module struct
             module Query_handler = Q
             let this = Q.create (Q.config_of_sexp config)
           end : Query_handler_instance)
        in
        Hashtbl.set t.active ~key:handler_name ~data:instance;
        Ok Sexp.unit
\end{lstlisting}

Since the \passthrough{\lstinline!load!} function will refuse to
\passthrough{\lstinline!load!} an already active handler, we also need
the ability to unload a handler. Note that the handler explicitly
refuses to unload itself:

\begin{lstlisting}[language=Caml]
let unload t handler_name =
    if not (Hashtbl.mem t.active handler_name) then
      Or_error.error "Handler not active" handler_name String.sexp_of_t
    else if handler_name = name then
      Or_error.error_string "It's unwise to unload yourself"
    else (
      Hashtbl.remove t.active handler_name;
      Ok Sexp.unit
    )
\end{lstlisting}

Finally, we need to implement the \passthrough{\lstinline!eval!}
function, which will determine the query {interface} presented to the
user. We'll do this by creating a variant type, and using the
s-expression converter generated for that type to parse the query from
the user:

\begin{lstlisting}[language=Caml]
type request =
    | Load of string * Sexp.t
    | Unload of string
    | Known_services
    | Active_services
  [@@deriving sexp]
\end{lstlisting}

The \passthrough{\lstinline!eval!} function itself is fairly
straightforward, dispatching to the appropriate functions to respond to
each type of query. Note that we write
\passthrough{\lstinline!<:sexp\_of<string list>>!} to autogenerate a
function for converting a list of strings to an s-expression, as
described in
\href{data-serialization.html\#data-serialization-with-s-expressions}{Data
Serialization With S Expressions}.

This function ends the definition of the
\passthrough{\lstinline!Loader!} module:

\begin{lstlisting}[language=Caml]
let eval t sexp =
    match Or_error.try_with (fun () -> request_of_sexp sexp) with
    | Error _ as err -> err
    | Ok resp ->
      match resp with
      | Load (name,config) -> load   t name config
      | Unload name        -> unload t name
      | Known_services ->
        Ok ([%sexp_of: string list] (Hashtbl.keys t.known))
      | Active_services ->
        Ok ([%sexp_of: string list] (Hashtbl.keys t.active))
end
\end{lstlisting}

Finally, we can put this all together with the command-line interface.
We first create an instance of the loader query handler and then add
that instance to the loader's active table. We can then just launch the
command-line interface, passing it the active table:

\begin{lstlisting}[language=Caml]
let () =
  let loader = Loader.create [(module Unique); (module List_dir)] in
  let loader_instance =
    (module struct
       module Query_handler = Loader
       let this = loader
     end : Query_handler_instance)
  in
  Hashtbl.set loader.Loader.active
    ~key:Loader.name ~data:loader_instance;
  cli loader.Loader.active
\end{lstlisting}

Now build this into a command-line interface to experiment with it:

\begin{lstlisting}
(executable
  (name       query_handler_loader)
  (libraries  core core_kernel ppx_sexp_conv)
  (preprocess (pps ppx_sexp_conv)))
\end{lstlisting}

\begin{lstlisting}[language=bash]
\end{lstlisting}

The resulting command-line interface behaves much as you'd expect,
starting out with no query handlers available but giving you the ability
to load and unload them. Here's an example of it in action. As you can
see, we start out with \passthrough{\lstinline!loader!} itself as the
only active handler:

\begin{lstlisting}
$ ./query_handler_loader.byte
>>> (loader known_services)
(ls unique)
>>> (loader active_services)
(loader)
\end{lstlisting}

Any attempt to use an inactive query handler will fail:

\begin{lstlisting}
>>> (ls .)
Could not find matching handler: ls
\end{lstlisting}

But, we can load the \passthrough{\lstinline!ls!} handler with a config
of our choice, at which point it will be available for use. And once we
unload it, it will be unavailable yet again and could be reloaded with a
different config:

\begin{lstlisting}
>>> (loader (load ls /var))
()
>>> (ls /var)
(agentx at audit backups db empty folders jabberd lib log mail msgs named
 netboot pgsql_socket_alt root rpc run rwho spool tmp vm yp)
>>> (loader (unload ls))
()
>>> (ls /var)
Could not find matching handler: ls
\end{lstlisting}

Notably, the loader can't be loaded (since it's not on the list of known
handlers) and can't be unloaded either:

\begin{lstlisting}
>>> (loader (unload loader))
It's unwise to unload yourself
\end{lstlisting}

Although we won't describe the details here, we can push this dynamism
yet further using OCaml's dynamic linking facilities, which allow you to
compile and link in new code to a running program. This can be automated
using libraries like \passthrough{\lstinline!ocaml\_plugin!}, which can
be installed via OPAM, and which takes care of much of the workflow
around setting up dynamic linking. ~

\hypertarget{living-without-first-class-modules}{%
\subsection{Living Without First-Class
Modules}\label{living-without-first-class-modules}}

It's worth noting that most designs that can be done with first-class
modules can be simulated without them, with some level of awkwardness.
For example, we could rewrite our query handler example without
first-class modules using the following
types:\index{first-class modules/alternatives to}

\begin{lstlisting}[language=Caml]
# type query_handler_instance = { name : string
                                ; eval : Sexp.t -> Sexp.t Or_error.t
  }
type query_handler_instance = {
  name : string;
  eval : Sexp.t -> Sexp.t Or_error.t;
}
# type query_handler = Sexp.t -> query_handler_instance
type query_handler = Sexp.t -> query_handler_instance
\end{lstlisting}

The idea here is that we hide the true types of the objects in question
behind the functions stored in the closure. Thus, we could put the
\passthrough{\lstinline!Unique!} query handler into this framework as
follows:

\begin{lstlisting}[language=Caml]
# let unique_handler config_sexp =
    let config = Unique.config_of_sexp config_sexp in
    let unique = Unique.create config in
    { name = Unique.name
    ; eval = (fun config -> Unique.eval unique config)
    }
val unique_handler : Sexp.t -> query_handler_instance = <fun>
\end{lstlisting}

For an example on this scale, the preceding approach is completely
reasonable, and first-class modules are not really necessary. But the
more functionality you need to hide away behind a set of closures, and
the more complicated the relationships between the different types in
question, the more awkward this approach becomes, and the better it is
to use first-class modules. ~
